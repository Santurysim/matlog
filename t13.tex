\section{Понятие сигнатуры и модели (алгебраической системы) данной сигнатуры. Примеры моделей: стандартная модель
арифметики; кольцо целых чисел; кольца многочленов и матриц порядка n над данным полем; евклидова плоскость в
сигнатуре элементарной геометрии Тарского $(R^{2}; =; B; \cong)$; модель Пуанкаре геометрии Лобачевского.}

\begin{definition}
	Пусть M -- непустое множество. \textit{n-арным предикатом} на M называется произвольное подмножество $Q
	\subseteq M^{n}$.

	\textit{n-aрной функцией} на M называется функция f: $M^{n}\to M$.
	Если Q -- n-арный предикат, то часто пишут $Q(x_1, \ldots , x_n)$ вместо $\langle x_1, \ldots , x_n \rangle \in
	Q$. Аналогично, $f(x_1, \ldots , x_n)$ означает $f (\langle x_1, \ldots , x_n \rangle)$.

	\textit{Константами} называем произвольныe элементы множества M.

	\textbf{Сигнатура} -- это набор из трех множеств имён $\varSigma =  \langle Const, Func, Pred \rangle $, где
	Const -- множество имён констант, Func -- функциональных символов, Pred -- предикатных символов, и функции
	валентности, сопоставляющей каждому предикатному и функциональному символу число его
	аргументов.$\label{formula2}$

	$$arity : Pred_\Sigma \cup Func_\Sigma \rightarrow \mathbb{N} \backslash\{0\}$$
\end{definition}

\definition{
	\textbf{Алгебраическая система (или модель) сигнатуры $\varSigma $} есть непустое множество M вместе с
	отображением, сопоставляющим каждому предикатному символу P из $\varSigma $ некоторый предикат $P_{M}$ на M той же
	валентности, каждому функциональному символу f функцию $f_{M}$ на M той же валентности, и каждой символу $C \in
	Const_{\varSigma} $ константу $c_{M} \in M$. Такое отображение называется \textit{интерпретацией} $\varSigma $ на
	M. Множество M называется \textit{универсумом} или \textit{носителем} данной интерпретации (модели). Модель
	сигнатуры $\varSigma $ с носителем M обозначается \textbf{(M; $\varSigma$)}. $\label{formula3}$
}

\begin{center}
	\textbf{Примеры моделей}
\end{center}

\begin{example} [Стандартная модель арифметики]
	$(\mathbb{N}; =, S, +, \times, 0)$ \\
	Здесь $S(x) \rightleftarrows x+1 $ есть одноместная функция следования на множестве N,
	а все остальные функции и предикаты имеют стандартный смысл.
\end{example}

\begin{example} [Кольцо целых чисел]
	$(\mathbb{Z}; =,+,-,\times,0,1)$. Здесь ''-'' это одноместная функция, отображающая x на -x, а все остальные
	функции и предикаты имеют стандартный смысл.
\end{example}

\begin{example} 
	Любое другое кольцо (с единицей) может рассматриваться как
	модель той же сигнатуры, например
	\begin{itemize}
		\item $\mathbb{Q}[x]$ -- кольцо многочленов над полем Q.
		\item $\mathbb{Z}_{n}$ -- кольцо вычетов по модулю n.
		\item $M_{n}(\mathbb{R})$ -- кольцо матриц порядка n.
	\end{itemize}
\end{example}

\begin{example} [Евклидова плоскость в сигнатуре элементарной геометрии Тарского] $(R^{2};=;B;\cong)$
	\begin{itemize}
		\item $\mathbb{R}^2$ -- множество точек евклидовой плоскости;
		\item $B(a, b ,c)$ -- трёхместный предикат ''точка b лежит на прямой ac между точками a и c'';
		\item $\cong$ -- четырёхместный предикат (записываемый $ab \cong cd$) ''отрезки, задаваемые парами точек ab и
			cd, имеют равные длины''.
	\end{itemize}
\end{example}

\begin{example} [Модель Пуанкаре геометрии Лобачевского]
	$(H^2; =,\cong, B)$, где
	\begin{itemize}
		\item $H^2 \rightleftarrows \{z \in \mathbb{C} : Im(z)>0 \}$ - множество точек верхней евклидовой полуплоскости;
		\item $B(a, b ,c)$ -- трёхместный предикат «точка b лежит между точками a и c на полуокружности (или полупрямой),
			проходящей через a, c и ортогональной вещественной оси»;
		\item $\cong$ -- четырёхместный предикат (записываемый $ab \cong cd$) ''отрезки, задаваемые парами точек ab и cd,
			имеют равные длины в смысле метрики Пуанкаре''.
	\end{itemize}
\end{example}
