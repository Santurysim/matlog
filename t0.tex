\section*{Обозначения}
\addcontentsline{toc}{section}{Обозначения}
\begin{enumerate}
	\item $A,B$ формулы. {$\boldsymbol{A}\equiv \boldsymbol{B}$}-равносильные (эквивалентные) формулы \ref{formula}
	\item Сигнатура $\Sigma$- \ref{formula2}
	\item Модель сигнатуры $\varSigma $ с носителем M обозначается (M; $\varSigma$)-\ref{formula3}
	\item $ FrVar = \{a_{0}, a_{1}, a_{2}, \ldots \},$\\
		$ BdVar = \{v_{0}, v_{1},v_{2}, \ldots \}$,-\ref{formula4}
	\item Обозначим через $ \Sigma(M)$ сигнатуру, получаемую из $  \Sigma$ добавлением новых символов констант
		${\underline {c} : c \in M.  } $ \ref{formula5}
	\item A-формула,M-модель. $ M \vDash A $-формула А истина в модели M \ref{formula6}
	\item Г -множество формул сигнатуры (Г может быть теорией Т) $\Sigma$,$A$- формула той же сигнатуры.
		$\text{Г}\vDash A$ -логически следует \ref{formula7}
	\item $C[P/A]$ означает результат замены всех вхождений $P$ в формулу $C$ на $A$
	\item М-модель, Т-теория.$M\vDash T$-теория $T$ выполнима в модели M \ref{formula8}
	\item $\vdash A$ - Формула $A$ называется \textbf{выводимой} \ref{formula9}
	\item $T \vdash A$- Формула $A$ называется \textbf{выводимой (доказуемой)} в теории $T$ \ref{formula10}
	\item $U \vdash T$-теория $U$ содержит теорию $T$, \ref{formula11}
	\item $U \equiv T$-теории $T$ и $U$ (дедуктивно) эквивалентны \ref{formula12}
\end{enumerate}
