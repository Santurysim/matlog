\section{Семантика логики предикатов первого порядка. Расширение сигнатуры
данной модели константами. Значение замкнутого терма расширенной
сигнатуры в данной модели. Истинностное значение замкнутой формулы
расширенной сигнатуры в данной модели.}

Пусть M -- модель сигнатуры $\Sigma$. Обозначим через $ \Sigma(M)$ сигнатуру, получаемую из $  \Sigma$ добавлением
новых символов констант ${\underline {c} : c \in M. \label{formula5} } $


\begin{definition}

	Пусть t -- замкнутый терм языка $L_{\Sigma (M) }$. Значение терма
	t в модели M есть элемент $ t_{M} \in M  $, определяемый индукцией по построению t.

	(i) Если $a \in M $ то  $\underline{a}_{M} \rightleftharpoons a$;

	(ii) Если $ c \in Const_{\Sigma} $ , то $c_{M} $ есть данная нам интерпретация c.

	(iii) Если t есть $f(t_{1},...,t_{n})$, гдe $f \in \mathrm{Func}_{\Sigma}$, то $ t_{M} =
	f_{M}((t_{1})_M,..,(t_{n})_M )$
\end{definition}
\begin{definition}
	Пусть A -- замкнутая формула языка $L_{\Sigma }$(M). \textit{Истинностное значение} формулы A в модели M
	определяется индукцией по построению A (oтношение $ M \vDash A $ читается ''формула А истина в модели M'')
	\label{formula6}
	\begin{enumerate}
		\item $M \vDash P\left(t_{1}, \ldots, t_{n}\right) \stackrel{\text { def }}{\Longleftrightarrow}
			P_{M}\left(\left(t_{1}\right)_{M}, \ldots,\left(t_{n}\right)_{M}\right)$, если $A=P$ -- атомарная формула;
		\item  $M \vDash(B \rightarrow C) \stackrel{\text { def }}{\Longleftrightarrow}(M \nvDash B \text { или } M
			\vDash C)$
		\item $M \vDash \neg B \stackrel{\text { def }}{\Longleftrightarrow} M \nvDash B$
		\item $M \vDash(B \wedge C) \stackrel{\text { def }}{\Longleftrightarrow}\left(M \vDash B \text{ и } M
			\vDash C\right)$
		\item $M \vDash(B \vee C) \stackrel{\text { def }}{\Longleftrightarrow}(M \vDash B \text { или } M \vDash
			C)$
		\item $M \vDash(\forall x B[a / x]) \stackrel{\text { def }}{\Longleftrightarrow}$ для всех $x \in M \ M
			\vDash B[a /\underline{x}]$
		\item $M \vDash(\exists x B[a / x]) \stackrel{\text { def }}{\Longleftrightarrow}$ существует $x \in M \ M
			\vDash B[a/ \underline{x}]$
	\end{enumerate}
	Если список $b_{1}, \ldots , b_{n}  $ coдержит все свободные переменные формулы A, а
	$x_{1}, \ldots, x_{n} \in M,$ To $M \vDash A\left[b_{1} / \underline{x}_{1}, \ldots, b_{n} /
	\underline{x}_{n}\right]$ сокращенно записываем как $M \vDash$
	$A\left[b_{1} / x_{1}, \ldots, b_{n} / x_{n}\right]$ или даже $M \vDash A\left[x_{1}, \ldots, x_{n}\right]$.

\end{definition}
\begin{remark}
	Нельзя говорить об истинности или ложности незамкнутых формул, поскольку их истинностные значения зависят от выбора
	значений параметров — входящих в формулу свободных переменных. Например, Формула $a+1 = b$  в стандартной
	модели арифметики может быть как истинна, так и ложна, в зависимости от a и b.
\end{remark}
