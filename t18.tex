\section[Выполнимые формулы и множества формул языка первого порядка. Общезначимые и тождественно ложные формулы,
их связь с выполнимыми формулами; примеры. Семантическое следование в логике первого порядка, его связь с понятиями
выполнимости и общезначимости.]{\sloppy Выполнимые формулы и множества формул языка первого порядка. Общезначимые и
тождественно ложные формулы, их связь с выполнимыми формулами; примеры. Семантическое следование в логике первого
порядка, его связь с понятиями выполнимости и общезначимости.}
\begin{definition}
	Формула $A(b_1\dots b_n)$ сигнатуры $\Sigma$ \textbf{выполнима} в модели $(M,\Sigma)$, если для некоторых
	констант $c_1,\dots,c_n$ предложение
	сигнатуры $\Sigma$ истинно. Формула $A$ сигнатуры $\Sigma$ выполнима, если она выполнима в некоторой модели
	$(M,\Sigma)$
\end{definition}
\begin{definition}
	Множество формул Г сигнатуры $\Sigma$ \textbf{выполнимо} в модели M, если существует функция $f: FrVar
	\rightarrow M$ такая, что при подстановке вместо каждой переменной $a_i$  константы $\underline{f(a_i)}$
	сигнатуры $\Sigma(M)$ все формулы истинны в M.\\Такую функцию $f$ будем называть \textbf{выполняющей} оценкой
	для Г.\\ Множество формул Г \textbf{выполнимо}, если Г выполнимо в некоторой модели.
\end{definition}
\begin{definition}
    Формула $A$ \textbf{общезначима} (тождественно истинна), если $\neg A$ не выполнима.\\
    Формула $A$ \textbf{тождественно ложна}, если $A$ не выполнима.
\end{definition}
\begin{definition}
	Пусть Г -- некоторое множество формул сигнатуры $\Sigma$ и $A$ -- формула той же сигнатуры. Говорят, что $A$
	\textbf{логически следует (или семантически следует)} из множества Г (обозначение $\text{Г}\vDash A$), если для
	любой модели $M$ сигнатуры $\Sigma$ формула $A$ истинна в $M$ при любой выполняющей оценке для множества Г.
	\label{formula7}
\end{definition}
Соотношения между понятиями выполнимости, общезначимости и логическим следованием в логике предикатов такие же, как
и в логике высказываний.
\begin{proposition}
    1) $A$ -- общезначима $\Longleftrightarrow \: \varnothing \vDash A$\\
    2)Г выполнимо  $\Longleftrightarrow \:\text{Г}\nvDash \perp$\\
    3)$\text{Г} \vDash A \Longleftrightarrow \: \text{Г}\cup (\neg A)$ не выполнимо
\end{proposition}
\begin{proposition}
	$\left\{B_{1}, \ldots, B_{n}\right\} \vDash A \Longleftrightarrow\left(\bigwedge_{i=1}^{n} B_{i}\right)
	\rightarrow A$ -- общезначима
\end{proposition}
