\section{Аксиома выбора. Лемма Цорна. Теорема Цермело (всякое множество вполне упорядочиваемо)}
\begin{definition}
Пусть S — семейство непустых множеств. Функцией выбора на S называем функцию, сопоставляющую каждому множеству из S некоторый его элемент, то есть функцию f: $S \mapsto \bigcup$ S такую, что $\forall x \in S \ f(x) \in x$.
\end{definition}

\begin{theorem} [Аксиома выбора] 
Для всякого S такого, что $ \varnothing \notin S$, существует функция выбора на S.
\end{theorem}

\begin{theorem} [Теорема Цермело]
Всякое множество можно вполне упорядочить.

(Более строго: для всякого множества X существует бинарное отношение < на X такое, что (X, <) — вполне упорядоченное множество.)
\end{theorem}

\begin{lemma} [Лемма Цорна]
Пусть (X, <) — частично упорядоченное множество, в котором любая цепь $S \subset X$ имеет верхнюю грань. Тогда в (X, <) найдётся максимальный элемент.
\end{lemma}

\begin{theorem}
Любые два множества сравнимы по мощности, то есть для любых множеств A, B найдётся инъекция из A в B или из B в A.
\end{theorem}

\begin{proof}
Действительно, вполне упорядочим множества A и B. Тогда одно из них вложимо в другое как начальный отрезок.
\end{proof}