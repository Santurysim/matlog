\section{Аксиома выбора. Лемма Цорна. Теорема Цермело (всякое множество вполне упорядочиваемо)}
\begin{definition}
	Пусть $S$ -- семейство непустых множеств. \textbf{Функцией выбора на $S$} назывестся функция $f: S\to\cup S$,
	такая, что $\forall X\in S f(X)\in X$
\end{definition}
\begin{axiom}[выбора]
	\label{choiceaxiom}
	Для всякого $S$, такого, что $\varnothing\in S$, существует функция выбора на S.
\end{axiom}
\begin{definition}
	\textbf{Цепь} в упорядоченном множестве $X$ - подмножество, любые два элемента которые сравнимы.

	\textbf{Верхняя грань} цепи - элемент, больший или равный любого элемента цепи.
\end{definition}
\begin{lemma}{Цорна}
	\label{zornlemma}
	Пусть $(X,<)$ -- частично упорядоченное множество, в котором любая цепь $C\subset X$ имеет верхнюю грань. Тогда
	в $X$ существует максимальный элемент.
\end{lemma}
\begin{corollary}
	Если в частично упорядоченном множестве не существует максимального элемента, то найдётся бесконечная цепь.
\end{corollary}
\begin{theorem}[Цермело]
	\label{zermelotheorem}
	Всякое множество может быть вполне упорядочено.
\end{theorem}
\begin{theorem}
	Аксиома выбора, лемма Цорна и теорема Цермело равносильны.
\end{theorem}
% vim: colorcolumn=120 textwidth=115
