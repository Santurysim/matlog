\documentclass[12pt,a4paper,draft]{report}
\usepackage[utf8]{inputenc}
\usepackage[T2A]{fontenc}
\usepackage[russian]{babel}
\usepackage[pdfencoding=auto,psdextra]{hyperref}
\usepackage{amsmath, amssymb, amsthm, amscd}

\hypersetup{bookmarksnumbered,colorlinks=true,linkcolor=blue,filecolor=blue,urlcolor=blue,
	citecolor=blue,pdftitle={Введение в матлог. Билеты.}}

\urlstyle{same}

\textwidth=500pt
\textheight=750pt
\oddsidemargin=20pt
\hoffset=-1.5cm
\topmargin=-25mm

\newtheorem{theorem}{Теорема}[section]
\newtheorem{lemma}{Лемма}[section]
\newtheorem{proposition}{Предложение}[section]
\newtheorem{axiom}{Аксиома}[section]
\newtheorem*{corollary}{Следствие}
\renewcommand{\proofname}{Доказательство}

\theoremstyle{definition}
\newtheorem{definition}{Определение}[section]
\newtheorem*{remark}{Замечание}

\theoremstyle{remark}
\newtheorem*{problem}{Упражнение}
\newtheorem*{denotation}{Обозначение}
\newtheorem*{example}{Пример}
\newtheorem*{examples}{Примеры}


\renewcommand{\thesection}{\arabic{section}}

\title{Введение в математическую логику и теорию алгоритмов.\\Билеты}
\author{Люди}

\righthyphenmin=2

\begin{document}
\hypersetup{pageanchor=false}
\maketitle
\hypersetup{pageanchor=true}
\tableofcontents
\section[Язык теории множеств. Равенство множеств. Аксиомы. Упорядоченные пары, декартово произведение. Бинарные
отношения между множествами, отношение эквивалентности, фактормножество, функции]{Язык теории множеств. Равенство
множеств. Аксиомы равенства, пары, объединения, степени, выделения.\\Упорядоченные пары, декартово произведение,
бинарные отношения между множествами, отношение эквивалентности, фактормножество, функции}

\begin{definition}
	Основными неопределяемыми понятиями теории множеств являются понятие \textbf{множества} и понятие
	\textbf{быть элементом} множества. Неформально, множество понимается как некоторая (конечная или бесконечная)
	совокупность объектов, рассматриваемая как единое целое, отдельный объект. Объекты, входящие в совокупность,
	называются \textbf{элементами} данного множества. Запись $x \in A$ означает, что $x$ есть элемент множества $A$,
	или $x$ принадлежит $A$. Два множества считаются равными, то есть совпадают, если у них одни и те же элементы
	(иногда это определение называют \textbf{аксиомой объемности}):

	$$ x = y \overset{\underset{\mathrm{def}}{}}{\Longleftrightarrow} \forall z ( z \in x \leftrightarrow z \in y ) $$
\end{definition}
\subsection{Некоторые аксиомы теории множеств Цермело–Френкеля}

\begin{enumerate}

	\item \textbf{Аксиома равенства}: Равные множества $x$ и $y$ являются элементами одних и тех же множеств.

		$$ x = y \rightarrow \forall z ( x \in z \leftrightarrow y \in z ) $$ 

	\item \textbf{Аксиома пары}: Для любых $x$ и $y$ найдется множество $ z = {x, y}$, элементами которого являются в
		точности $x$ и $y$.

		$$ \forall x, y \exists z : \forall u (u \in z \leftrightarrow ( u = x \lor u = y)) $$ 

	\item \textbf{Аксиома объединения}: Для любого множества $X$ существует множество $Y = \bigcup X$, содержащее в
		точности те элементы, которые принадлежат хотя бы одному из элементов множества $X$.

		$$ \forall x \exists y: \forall u ( u \in y \leftrightarrow \exists z (u \in z \land z \in x)) $$ 

	\item \textbf{Аксиома степени}: Для любого $X$ существует множество $ Y = \mathcal{P} (X)$ всех подмножеств $X$.

		$$ \forall x \exists y : \forall z ( z \in y \leftrightarrow z \in x) $$ 

	\item \textbf{Схема аксиом выделения}: Для любого свойства $\varphi (x)$ и множества $X$ найдется множество $Y = \{x
		\in X: \varphi (x)\}$, содержащее те и только те элементы $x \in X$, которые удовлетворяют свойству $\varphi$.

		$$ \forall x \exists y: \forall u ( u \in y \leftrightarrow (u \in x \land \varphi (u))) $$ 

\end{enumerate}

\subsection{Другие определения}

\begin{definition}
	\textbf{Упорядоченная пара} -- это сопоставление паре множеств $x$, $y$ некоторое множество $z$,
	обозначаемое $\langle x, y \rangle$, таким образом, чтобы для всех $x_1, x_2, y_1, y_2$

	$$ \langle x_1, y_1 \rangle = \langle x_2, y_2 \rangle \Longleftrightarrow (x_1 = x_2 \land y_1 = y_2) $$
\end{definition}
Один из способов задания (по Куратовскому):  $\langle x, y \rangle \overset{\underset{\mathrm{def}}{}}{=} \{\{x,
y\}, \{x\}\}$
\begin{definition}
	Множество всех упорядоченных пар элементов множеств $A$ и $B$ называется \textbf{декартовым
	произведением} $A$ и $B$:

	$$ A \times B = \{\langle x, y \rangle : x \in A \land y \in B\} $$
	Заметим, что $\langle x, y \rangle \in \mathcal{P} ( \mathcal{P} ( A \cup B))$, которое существует по аксиомам
	объединения и степени, а значит произведение $ A \times B$ существует по аксиоме выделения.
\end{definition}

\begin{definition}\textbf{Бинарным отношением} между множествами $A$ и $B$ называется любое подмножество $R \subset A
	\times B$. Если $A = B$, то говорят о бинарном отношении \textbf{на} множестве $A$. Вместо $\langle x, y \rangle
	\in R$ часто пишут $xRy$. Бинарное отношение $R$ на множестве $A$ называется
	\begin{itemize}
		\item \textbf{рефлексивным}, если $ \forall x \in A$ $xRy$;

		\item \textbf{симметричным}, если $\forall x,y \in A (xRy \rightarrow yRx)$;

		\item \textbf{транзитивным}, если $\forall x,y,z \in A (xRy \land yRz \rightarrow xRz)$.
	\end{itemize}
	Отношение, обладающее всеми тремя этими свойствами называется \textbf{отношением эквивалентности}.
\end{definition}

\begin{definition}Множество всех классов эквивалентности $A$ по отношению $R$ называется \textbf{фактормножеством} и
	обозначается $A / R$:
	$$ A \ R = \{\,x_R\mid x \in A \,\} $$
\end{definition}

\subsection{Функции}

\begin{definition}Отношение $R \subset A \times B$ называется
	\begin{itemize}
		\item \textbf{тотальным}, если $ \forall x \in A$ $\exists y \in B$ $xRy$;

		\item \textbf{сюръективным}, если $ \forall y \in B$ $\exists x \in A$ $xRy$;

		\item \textbf{функциональным}, если $ \forall x \in A$ $\forall y_1, y_2 \in B (xRy_1 \land xRy_2 \rightarrow
			y_1 = y_2)$;

		\item \textbf{инъективным}, если $ \forall y \in B$ $\forall x_1, x_2 \in A (x_1 Ry \land x_2 Ry \rightarrow
			x_1 = x_2)$.
	\end{itemize}
	\textbf{Функцией} $f$ из $A$ в $B$ называется тотальное и функциональное бинарное отношение между $A$ и $B$
	(обозначение $f\colon A \to B$). Слово \textbf{отображение} есть синоним слова функция. Если $f$ -- функция, то
	мы пишем $f(x) = y$ вместо $\langle x, y \rangle \in f$.
\end{definition}

%\section{Натуральные числа. Индуктивные множества, аксиома бесконечности. Формальное определение множества натуральных чисел. Принципы индукции, порядковой индукции, наименьшего числа, их вывод из определения множества натуральных чисел.}

По определению, число 0 отождествляется с пустым множеством $\varnothing$, число 1 с $\{ \varnothing \}$, число 2 с $\{ \varnothing, \{ \varnothing \}\}$ и так далее. Требуется доказать, что это множество.

\textbf{Аксиома бесконечности}: Существует $S$ такое, что $\varnothing \in S$ и для любого $x \in S$ множество $x \cup\{x\} \in S$.

\definition{Множество $S$ называется \textbf{индуктивным}, если $\varnothing \in S$ и $\forall x \in S$ $x \cup\{x\} \in S$.

Аксиома бесконечности утверждает, что существует хотя бы одно индуктивное множество.

\begin{lemma}
Пересечение непустого множества индуктивных множеств индуктивно.
\end{lemma}

\begin{proof}{$\varnothing$ содержится в каждом из множеств, значит содержится и в пересечении. Если $x$ содержится в пересечении, то и $x \cup\{x\}$ содержится.}
\end{proof}

Пусть $S$ — некоторое индуктивное множество, существующее по аксиоме бесконечности. Рассмотрим множество $\mathcal{I}$ всех индуктивных подмножеств $S$.
Обозначим через $\omega$ его пересечение: $\omega:=\cap \mathcal{I}$. Из леммы  получаем, что $\omega$ индуктивно. Кроме того, $\omega$ содержится в любом индуктивном множестве: если $S^{\prime}$ индуктивно, то таковым является $S \cap S^{\prime}$, откуда $S \cap S^{\prime} \in \mathcal{I}$ и следовательно $\omega \subset S \cap S^{\prime}$. Таким образом, мы показали, что $\omega$ есть наименьшее (по включению) индуктивное множество. Такое множество единственно по аксиоме объёмности.

\definition{Наименьшее по включению индуктивное множество называется \textbf{множеством натуральных чисел} (обозначение $\mathbb{N}$ или $\omega$).}

Отметим, что $0 := \varnothing \in \mathbb{N}$ и из того, что $n \in \mathbb{N}$ следует $n \cup\{n\} \in \mathbb{N}$. Натуральное число $n \cup\{n\}$ назовём \textbf{следующим} за $n$ и будем обозначать $S(n) = n + 1$. Функция $n \mapsto n+1$ действует из $\mathbb{N}$ в $\mathbb{N}$. Для $n \in \mathbb{N}$ и любого $m$ определим $m < n$, если и только если $m \in n$. (Ниже мы докажем, что из $m \in n \in \mathbb{N}$ следует $m \in \mathbb{N}$.) Из этого определения мы видим, что для всех $m, n \in \mathbb{N}$ 

$$m<n+1 \leftrightarrow(m<n \vee m=n)$$

\subsection{Принцип математической индукции}

\begin{theorem}
(принцип индукции) Допустим, что некоторое множество $A$ удовлетворяет условиям: $0 \in A$ и $\forall n \in \mathbb{N} \ (n \in A \rightarrow n+1 \in A)$. Тогда $\forall n \in \mathbb{N} \ n \in A$.
\end{theorem}

\begin{proof}{По условию теоремы множество $A \cap \mathbb{N}$ индуктивно. Поскольку $\mathbb{N}$ -- наименьшее индуктивное, мы имеем $\mathbb{N} \subset A$.}
\end{proof}

\begin{theorem}
(порядковая индукция) Пусть множество $A$ удовлетворяет условию $\forall n \in \mathbb{N}(\forall m<n \ m \in A \rightarrow n \in A)$. Тогда $\forall n \in \mathbb{N} \ n \in A$.
\end{theorem}

\begin{proof}{Предположим, что $A$ удовлетворяет условию теоремы. Рассмотрим множество

$$A^{\prime}:=\{x \in \mathbb{N}: \forall y<x \ y \in A\}$$

Тогда $0 \in A$, поскольку $\neg \exists y$ $y<0$, то есть условие $\forall y<0$ $y \in A$ выполняется тривиально. Допустим $n \in \mathbb{N}$ и $n \in A^{\prime}$, тогда $\forall m<n$ $m \in A$. По условию теоремы отсюда следует $n \in A$. Мы утверждаем, что $\forall m < n + 1$ $m \in A$, то есть $n+1 \in A^\prime$. В самом деле, если $m < n + 1$, то $ m < n$ или $ m = n$. В каждом из этих случаев мы уже знаем, что $m \in A$. По предыдущей теореме мы заключаем $\forall n \in \mathbb{N}$ $n \in A^{\prime}$. Осталось вывести отсюда $\forall n \in \mathbb{N}$ $n \in A$. Рассмотрим любое $n \in \mathbb{N}$, тогда $n + 1 \in \mathbb{N}$ и по доказанному $n + 1 \in A^{\prime}$. Поскольку $n < n + 1$, по определению $A^\prime$ отсюда следует $n \in A$.}
\end{proof}

\begin{theorem}
(принцип минимального элемента) Всякое непустое подмножество $A \subset \mathbb{N}$ имеет минимальный элемент.
\end{theorem}

\begin{proof} Допустим противное, то есть $A \neq \varnothing$ и

$$\neg \exists n \in A \ \forall m<n \  m \notin A$$

Рассмотрим $B:=\mathbb{N} \backslash A$ Докажем

$$\forall n \in \mathbb{N}(\forall m<n \  m \in B \rightarrow n \in B)$$

Допустим $n \in \mathbb{N}$ и $\forall m<n \  m \in B$. Тогда $\forall m<n \ m \notin A$. Значит $n \notin A$ в силу первого утверждения, то есть $n \in B$. По теореме 2.2 заключаем, что $\forall n \in \mathbb{N} \ n \in B$. Отсюда следует $A = \varnothing$, противоречие.
\end{proof}

\begin{corollary}
$\forall n \in \mathbb{N} \ n \not<n$.
\end{corollary}

\begin{proof} 
Применим порядковую индукцию. Допустим $\forall m < n \ m \not< m$. Если $ n < n$, то в качестве $m$ можно взять само $n$, тогда получим
$n \not< n$, противоречие. Следовательно, $n \not< n$. 
\end{proof}

\begin{corollary} 
$\forall n \in \mathbb{N} \ \forall x<n \ x \in \mathbb{N}$ 
\end{corollary}

\begin{proof} 
Применим индукцию. $\forall m < 0 \ m \in \mathbb{N}$ тривиально.
Допустим $n \in \mathbb{N}$ и $\forall x<n \ x \in \mathbb{N}$. Тогда очевидно $\forall x<n+1 \ x \in \mathbb{N}$. 
\end{proof}

\begin{corollary} 
$\forall k, m, n \in \mathbb{N} \ (k<m<n \rightarrow k<n)$.
\end{corollary}

\begin{proof} 
Индукция по $n$. Случай $n = 0$ тривиален. Допустим, что утверждение верно для $n$ и имеет место $k<m<n + 1$. Тогда $m < n$ или $m = n$. В первом случае по предположению индукции $k < n$. Во втором случае мы уже знаем, что $k < n = m$. Из $k < n$ следует и
$k < n + 1$. 
\end{proof}

\begin{corollary} 
$\forall m, n \in \mathbb{N} \ \neg(n<m \wedge m<n)$.
\end{corollary}

\begin{proof} 
Если $n<m<n$, то $n<n$, что противоречит следствию 2.3.1. 
\end{proof}

\subsection{Аксиомы Дедекинда-Пеано}

\begin{enumerate}

\item{$ \forall n \in \mathbb{N} \  S(n) \not= 0$}

\item{$ \forall n,m \in \mathbb{N} \ (S(n)=S(m) \leftrightarrow n = m)$}

\item{$ \forall n \in \mathbb{N} \ (n = 0 \lor \exists m (S(m) = n))$}

\item{Принцип индукции}

\end{enumerate}

Данные аксиомы выполняются для определения, данного выше.

\begin{proof}

\begin{enumerate}

\item{Множество $S(n)=n+1$ непусто}

\item{Легко доказывается индукцией по $n$}

\item{Достаточно доказать импликацию слева направо. Допустим $n \cup\{n\}=m \cup\{m\}$. Так как $m \in n \cup\{n\}$ имеем $m \in n$ или
$m = n$. Во втором случае утверждение доказано. Допустим $m \in n$. Так
как $n \in m \cup\{m\}$ имеем $n \in m$ или $n = m$. Первый случай невозможен
по следствию 2.3.4. Остаётся второй.}

\item{По определению.}
\end{enumerate}
\end{proof}
%\section{Рекурсивные определения, определение арифметических операций сложения и умножения.}

Функции натурального аргумента часто определяются по индукции (рекурсии). Для того, чтобы определить значение функции на аргументе $n+1$ предполагается известным значение функции на предыдущем аргументе $n$. Простейшая схема рекурсивного определения функции $f: \mathbb{N} \rightarrow Y$ сводится к следующей теореме. 

\begin{theorem}
Пусть $Y$ — множество, $y_{0} \in Y$ и $h: Y \rightarrow Y$ — любая
функция. Тогда существует единственная функция $f: \mathbb{N} \rightarrow Y$ такая,
что для всех $n \in \mathbb{N}$

\begin{equation}
\left\{\begin{array}{l}
{f(0)=y_{0}} \\
{f(n+1)=h(f(n))}
\end{array}\right.
\end{equation}

\end{theorem}

\begin{proof}

Пусть даны $Y, \ y_0$ и $h$ как в условии теоремы. Рассмотрим множество $F$ всех тех функций $f: m \rightarrow Y$, где $m \in \mathbb{N}$, для которых выполнены условия (1) для любого $n \in m$. Это множество непусто,
поскольку содержит пустую функцию, а также функцию, состоящую из
пары $\left\langle 0, y_{0}\right\rangle$.

Утверждается, что любые две функции $f, g \in F$ совпадают на пересечении своих областей определения. В противном случае рассмотрим минимальный $k \in \mathbb{N}$ такой, что $f(k) \neq g(k)$. Мы имеем $k \neq 0$, поскольку $f(0)=y_{0}=g(0)$. Следовательно $k=s+1$, причем $f(s) = g(s)$, поскольку $k$ -- минимальный. Отсюда $f(k)=f(s+1)=h(f(s))=h(g(s))=g(s+1)=g(k)$, противоречие.

Каждая $g: m \rightarrow Y$ есть подмножество $m \times Y \subset \mathbb{N} \times Y$. Рассмотрим множество $\bigcup F \subset \mathbb{N} \times Y$. Утверждается, что $f:=\cup F$ есть функция $\mathbb{N} \rightarrow Y$. Отношение $\bigcup F$ функционально, поскольку любые два элемента $F$ совпадают на общей области определения. Докажем тотальность, рассуждая от противного. Рассмотрим минимальное $m$ такое, что
$m \notin dom(f)$. Тогда $f: m \rightarrow Y$ и можно продолжить $f$ до функции $f^\prime : m+1 \rightarrow Y$, определив $f^\prime (m) := h(f(m))$. Очевидно, $f^\prime \in F$, поэтому $m \in dom(\bigcup F)$, противоречие. Свойства (1) очевидно выполняются для $f$, тем самым существование $f$ доказано.
Единственность $f$, как в рассуждении выше, легко следует по принципу наименьшего числа.
\end{proof}

Применяя эту теорему мы доказываем, например, существование и единственность функции $f(x) = 2^x$ (предполагая известным определение сложения). Действительно, $f$ рекурсивно определяется равенствами $f(0) = 1$ и $ f(n+1) = f(n) + f(n)$.

Заметим, что на натуральных числах уже определена функция последователя $S(n) = n+1$. Сложение и умножение можно определить рекурсией по второму аргументу. Сложение удовлетворяет равенствам

\begin{equation}
\left\{\begin{array}{l}
{m + 0 = m} \\
{m + (n + 1) = (m + n) + 1}
\end{array}\right.
\end{equation}

Чтобы уложить эту схему в рамки теоремы 3.1 заметим, что функции $f : \mathbb{N} \times X \rightarrow Y$ можно отождествить с функциями $\mathbb{N} \rightarrow Y^X$, то есть с
последовательностями функций $f_n : X \rightarrow Y$. Таким образом, с помощью
теоремы 3.1 надо построить последовательность функций $f_n : \mathbb{N} \rightarrow \mathbb{N}$ такую, что

\begin{equation}
\left\{\begin{array}{l}
{f_0 = id_\mathbb{N}} \\
{f_{n+1} = s \circ f_n}
\end{array}\right.
\end{equation}

Тогда $f_0 (m) = m$ и $f_{n+1} (m) = (s \circ f_n) (m) = s(f_n (m)) = f_n (m) + 1$. То есть, если положить $m+n := f_n (m)$, то выполняются равенства (2).
Аналогично определяется умножение, как единственная функция $\mathbb{N}^2 \rightarrow \mathbb{N}$ для которой

\begin{equation}
\left\{\begin{array}{l}
{m \cdot 0 = 0} \\
{m \cdot (n+1) = (m \cdot n) + m}.
\end{array}\right.
\end{equation}
%\section{Построение множеств целых, рациональных, вещественных чисел в теории множеств.}

\begin{definition}
	Если считать известным определение натурального ряда $\mathbb{N}$, то множество \textbf{целых чисел} $\mathbb{Z}$
	удобно рассматривать как фактормножество. Целое число можно представить разностью двух натуральных чисел $m-n$.
	При этом некоторые пары задают одно и то же число. Поэтому множество целых чисел Z определяется как

	$$
	\mathbb{Z}:=(\mathbb{N} \times \mathbb{N}) /=_{\mathbb{Z}},
	$$

	где отношение эквивалентности $=_{\mathbb{Z}}$ задаётся следующим образом:

	$$
	\left\langle m_{1}, n_{1}\right\rangle=_{\mathbb{Z}}\left\langle m_{2}, n_{2}\right\rangle \stackrel{\text { def
	}}{\Longleftrightarrow} m_{1}+n_{2}=n_{1}+m_{2}.
	$$
\end{definition}

\begin{definition}
	\textbf{Рациональное число} $q = \frac{m}{n}$ можно рассматривать как пару $\left\langle m, n\right\rangle$, где $m
	\in \mathbb{Z}$ и $n \in \mathbb{N} \backslash \{0\}$. Однако некоторые пары задают одно и то же рациональное
	число $q$. Поэтому мы вводим отношение эквивалентности $=_{\mathbb{Q}}$ на $\mathbb{Z} \times(\mathbb{N}
	\backslash\{0\})$ по правилу

	$$
	\left\langle m_1, n_1 \right \rangle=_{\mathbb{Q}} \left \langle m_{2}, n_{2}\right \rangle
	\overset{\underset{\mathrm{def}}{}}{\Longleftrightarrow} m_{1} n_{2}=n_{1} m_{2}.
	$$

	Проверим, что $=_{\mathbb{Q}}$ в самом деле есть отношение эквивалентности. Рефлексивность очевидна, симметричность
	и транзитивность следуют из таковых свойств для целых чисел. Две дроби равны тогда и только тогда, когда
	соответствующие пары эквивалентны. Поэтому рациональные числа можно отождествить с соответствующими классами
	эквивалентности и официальное определение множества рациональных чисел $\mathbb{Q}$ -- это

	$$
	\mathbb{Q}:=\mathbb{Z} \times(\mathbb{N} \backslash\{0\}) /=_{\mathbb{Q}}.
	$$
\end{definition}

Для определения вещественных чисел воспользуемся некоторыми определениями из следующего билета.

\begin{definition}
	Рассмотрим совокупность начальных отрезков в рациональных числах, каждый из которых непустой, собственный (не
	совпадает во всем множеством $\mathbb{Q}$) и не содержит максимума. Каждый такой отрезок назовем
	\textbf{вещественным числом}. На этом множестве отрезков есть естественный порядок (по включению). Такие
	отрезки называются \textbf{дедекиндовыми сечениями}.
\end{definition}

%\section{Вполне упорядоченные множества. Начальные отрезки. Теорема о сравнении вполне упорядоченных множеств. Сумма, произведение вполне упорядоченных множеств.}

\begin{definition}
\textbf{Строгим частичным порядком} на множестве $X$ называем бинарное отношение $<$ на $X$, удовлетворяющее свойствам:

$\circ$ $x < y \land y<z \Rightarrow x<z$ (транзитивность);

$\circ$ $x \not< x$ (иррефлексивность).

Пару $(X, <)$ называем \textbf{частично упорядоченным множеством}. Элементы $x, y \in X$ называются \textbf{сравнимыми}, если $x < y$, или $x = y$, или $y < x$. Частично упорядоченное множество $(X, <)$ называется \textbf{линейно упорядоченным}, или просто \textbf{упорядоченным}, если любые $x, y \in X$ сравнимы.

Пусть $(X, <)$ -- частично упорядоченное множество и $Y \subset X$. 

$\circ$ Элемент $y \in Y$ \textbf{максимальный} в $Y$, если $\forall x \in Y \ y \not< x$.

$\circ$ Элемент $y \in Y$ \textbf{наибольший} в $Y$, если $\forall x \in Y \ x \leqslant y$.

$\circ$ Элемент $x \in X$ есть \textbf{верхняя грань} $Y$, если $\forall y \in Y \ y \leqslant x$.
\end{definition}

\begin{example}
Частично упорядоченные множества:

$\circ$ $(\mathbb{R}, <)$, $(\mathbb{Q}, <)$, $(\mathbb{N}, <)$,

$\circ$ $\mathbb{N}$ с отношением $x$ \textit{есть собственный делитель} $y$;

$\circ$ $(\mathcal{P}(X), \varsubsetneqq)$;

$\circ$ $\mathbb{N}^{*}$ с отношением \textit{последовательность $x$ -- собственное начало последовательности $y$}.

Первые три примера — линейно упорядоченные множества, а последние три — нет.
\end{example}

\begin{definition}
Всякое подмножество $Y \subset X$ частично упорядоченного множества $(X, <)$ можно также рассматривать как частично упорядоченное множество по отношению $<^\prime$ на $Y$:

$$
x<^{\prime} y \Longleftrightarrow(x, y \in Y \text { u } x<y)
$$

(Формально, можно было бы определить $<^\prime$ как $< \cap Y^2$.) В этом случае говорят, что порядок $<^\prime$ является \textbf{ограничением порядка} $<$ на множество $Y$ или \textbf{индуцирован} на $Y$ c $X$.

Множество $Y \subset X$ называется \textbf{цепью}, если любые два элемента $Y$ сравнимы. Другими словами, $Y$ -- цепь, если $Y$ линейно упорядочено в смысле индуцированного отношения порядка. Множество $Y \subset X$ называется \textbf{антицепью}, если любые два элемента $Y$ \textbf{несравнимы}.
\end{definition}

\begin{definition}
Пусть $(X, <_X)$ и $(Y, <_Y)$ -- линейно упорядоченные множества. Отображение $f : X \rightarrow Y$ называется \textbf{сохраняющим порядок} (или \textbf{возрастающим}), если

$$
\forall x_{1}, x_{2} \in X\left(x_{1}<_{X} x_{2} \Rightarrow f\left(x_{1}\right)<_{Y} f\left(x_{2}\right)\right)
$$

\textbf{Изоморфизмом} упорядоченных множеств $X$ и $Y$ называется биекция $f : X \rightarrow Y$, для которой $f$ и обратное отображение $f^{-1}$ сохраняют порядок. $X \cong Y$ означает, что упорядоченные множества $X$ и $Y$ изоморфны, то есть между ними существует изоморфизм.
\end{definition}

\begin{definition}
\textbf{Суммой} $X+Y$ назовём упорядоченное множество $(Z, <_Z)$, где $Z =
X \sqcup Y$ и для любых $z_1, z_2 \in Z$ соотношение $z_1 <_Z z_2$ имеет место в одном
из трех случаев:

$\circ$ $z_1, z_2 \in X$ и $z_1 <_X z_2$,

$\circ$ $z_1, z_2 \in Y$ и $z_1 <_Y z_2$,

$\circ$ $z_1 \in X$ и $z_1 \in Y$.

\textbf{Произведением} $X \cdot Y$ назовём множество $(Z, <_Z)$, где $Z = Y \times X$ и для любых $z_1 = (y_1, x_1) \in Z$ и $z_2 = (y_2, x_2) \in Z$ соотношение $z_1 <_Z z_2$ имеет место, если и только если $y_1 <_Y y_2$ или же $y_1 = y_2$ и $x_1 < x_2$. (Сравнение сначала элементов множества $Y$, а потом уже $X$, выражает ту идею, что $X \cdot Y$ состоит из копий множества $X$, упорядоченных между собой как $Y$, а не наоборот.)
\end{definition}

\begin{definition}
Упорядоченное множество $(X, <)$ называем \textbf{вполне упорядоченным}, если любое непустое подмножество $Y \subset X$ имеет наименьший элемент $y \in Y$. Наименьший элемент $Y$ -- единственный и обозначается $min(Y)$.

\textbf{Начальным отрезком} множества $(X, <)$ называем такое подмножество $Y \subset X$, для которого 

$$
\forall x, y(x \in Y, y<x \Rightarrow y \in Y).
$$

В частности, начальными отрезками $X$ считаем само $X$ и пустое множество.
\end{definition}

\begin{lemma}
Любой собственный начальный отрезок $(X, <)$ имеет вид $\bar{a}=\{x \in X | \ x<a\}$ для некоторого $a \in X$.
\end{lemma}

\begin{proof}
Пусть $Y$ -- собственный начальный отрезок $X$, и пусть $a = min ( X \backslash Y)$. Заметим, что $a \notin Y$ и $\forall x<a \ x \in Y$. Второе влечёт $\bar{a} \subset Y$. С другой стороны, если $\exists y \in Y \ a \leqslant y$, то мы имеем $a \in Y$, поскольку $Y$ -- начальный отрезок. Этого не может быть, значит $Y \subset \bar{a}$.
\end{proof}

\begin{lemma}
Пусть $(X, <)$ вполне упорядочено и $f : X \rightarrow X$ сохраняет порядок. Тогда $\forall x \in X \ f(x) \geqslant x$.
\end{lemma}

\begin{proof}
В противном случае рассмотрим $a = min Y$, где $Y=\{x \in X | f(x) < x\}$. Поскольку $a \in Y$ мы имеем $f(a) < a$. Отсюда следует $f(f(a)) < f(a)$ по монотонности $f$. Но тогда $f(x) < x$ для некоторого $x < a$ (возьмём $x = f(a)$), что противоречит минимальности $a$.
\end{proof}

\begin{theorem}
\begin{enumerate}
\item{Вполне упорядоченное множество не изоморфно никакому своему собственному начальному отрезку.}

\item{Для любых двух вполне упорядоченных множеств одно изоморфно начальному отрезку другого.}
\end{enumerate}
\end{theorem}

\begin{proof}
\begin{enumerate}
\item{Пусть $Y \subset X$ -- собственный начальный отрезок $X$, и $f : X \rightarrow Y$ -- изоморфизм. Тогда по лемме 5.2 имеем $f(x) \geqslant x$ для всех $x \in X$. Но если $a \in X \backslash Y$, то $f(a) \in Y$ и тем самым $f(a) < a$, поскольку $Y$ -- начальный отрезок $X$. Противоречие.}

\item{Рассмотрим бинарное отношение $R \subset X \times Y$ такое, что

$$
x R y \Longleftrightarrow \bar{x} \cong \bar{y}.
$$

Сначала докажем, что отношения $R$, $R^{-1}$ функциональны и сохраняют порядок.
Действительно, если $xRy_1$ и $xRy_2$, то $\bar{x} \cong \bar{y_1}$ и $\bar{x} \cong \bar{y_2}$, значит $\bar{y_1} \cong \bar{y_2}$. Поскольку $Y$ линейно упорядочено, мы имеем $y_1 < y_2$ или $y_2 < y_1$ или $y_1 = y_2$. Если $y_1 < y_2$, то $\bar{y_1}$ -- собственный начальный отрезок $\bar{y_2}$, что противоречит (1). Аналогично, не может быть $y_2 < y_1$, поэтому $y_1 = y_2$. 

Докажем, что $R$ сохраняет порядок. Допустим, что $x_1 < x_2$, $\bar{x_1} \cong \bar{y_1}$ и $\bar{x_2} \cong \bar{y_2}$. Изоморфизм $f : \bar{x_2} \rightarrow \bar{y_2}$ переводит $\bar{x_1}$ в некоторый собственный начальный отрезок $f(\bar{x_1}) \subset \bar{y_2}$. Если при этом $y_2 \leqslant y_1$, то получаем, что $\bar{y_1}$ изоморфно собственному начальному отрезку $f(\bar{x_1}) \cong \bar{x_1}$, что невозможно. Значит, $y_1 < y_2$.

Аналогично устанавливаем, что $x_1Ry$ и $x_2Ry$ влечёт $x_1 = x_2$, и что $R^{-1}$ сохраняет порядок.

Осталось доказать, что хотя бы одна из функций $R$ и $R^{-1}$ определена на всём множестве $X$ или на всём множестве $Y$, соответственно. Предположим противное и рассмотрим наименьший $a \in X$ такой, что $\nexists y \in Y \ aRy$ и наименьший $b \in Y$ такой, что $\nexists x \in X \ xRb$. Тогда $R$ есть изоморфизм начального отрезка $\bar{a} \subset X$ на начальный отрезок $\bar{b} \subset Y$, поскольку на $\bar{a}$ функция $R$ всюду определена, сохраняет порядок, и то же верно для обратной функции $R^{-1}$. Но тогда по определению $R$ мы имеем $aRb$. Противоречие с минимальностью $a$ и $b$.}
\end{enumerate}
\end{proof}
\section{Аксиома выбора. Лемма Цорна. Теорема Цермело (всякое множество вполне упорядочиваемо)}
\begin{definition}
Пусть S — семейство непустых множеств. Функцией выбора на S называем функцию, сопоставляющую каждому множеству из S некоторый его элемент, то есть функцию f: $S \mapsto \bigcup$ S такую, что $\forall x \in S \ f(x) \in x$.
\end{definition}

\begin{theorem} [Аксиома выбора] 
Для всякого S такого, что $ \varnothing \notin S$, существует функция выбора на S.
\end{theorem}

\begin{theorem} [Теорема Цермело]
Всякое множество можно вполне упорядочить.

(Более строго: для всякого множества X существует бинарное отношение < на X такое, что (X, <) — вполне упорядоченное множество.)
\end{theorem}

\begin{lemma} [Лемма Цорна]
Пусть (X, <) — частично упорядоченное множество, в котором любая цепь $S \subset X$ имеет верхнюю грань. Тогда в (X, <) найдётся максимальный элемент.
\end{lemma}

\begin{theorem}
Любые два множества сравнимы по мощности, то есть для любых множеств A, B найдётся инъекция из A в B или из B в A.
\end{theorem}

\begin{proof}
Действительно, вполне упорядочим множества A и B. Тогда одно из них вложимо в другое как начальный отрезок.
\end{proof}
%\section{Вывод леммы Цорна из аксиомы выбора.}

\begin{proof}[Доказательство леммы Цорна]
	Допустим, что (X, <) удовлетворяет условию леммы Цорна, но не имеет максимального элемента. Назовем строгой верхней
	гранью цепи C $\subset X$ такой элемент $x \in X$ что c < x для всех $c \in C$. Тогда можно утверждать, что для
	всякой цепи C в X множество её строгих верхних граней $\psi(C)$ непусто. (Рассмотрим любую верхнюю грань $x$
	цепи $C$. Поскольку элемент $x$ не максимален, найдётся y > x, он и будет строгой верхней гранью C)

	Рассмотрим теперь множество

	\begin{center}
		S = $\{\psi(C)\mid C$ - цепь в X$\}$
	\end{center}

	Заметим, что S будет множеством, поскольку S $\subset \mathcal{P} (X)$. Применяя аксиому выбора к множеству S мы
	можем заключить, что существует функция $\varphi$, сопоставляющая любой цепи C некоторую её строгую верхнюю
	грань $\varphi$(C). (Эта функция является композицией функции $\psi$ и функции выбора для S)

	Теперь мы построим цепь, которая будет настолько велика, что должна выйти за пределы X (это и будет желаемым
	противоречием). Идея состоит в неограниченном удлинении цепи путём применения функции $\varphi$.

	Множество S $\subset$ X называем корректным, если выполняются условия:

	1. (S, <) вполне упорядочено (порядок индуцирован с X);

	2. $\forall x \in S$ $x = \varphi(S_x)$, где $S_x$ означает $\{y \in S \mid y < x\}$

	Заметим, что корректными множествами являются

	\begin{center}
		$ \varnothing$; $\{ \varphi (\varnothing) \}$; $\{ \varphi (\varnothing), \varphi ( \{ \varphi (\varnothing) \})\}$
		и т. д.
	\end{center}

	Докажем следующее вспомогательное утверждение.

	\begin{lemma} (i) Если множества S и T корректны, то одно из них
		есть начальный отрезок другого.

		(ii) Объединение любого семейства корректных множеств корректно.
	\end{lemma}

	\begin{proof}
		(i) Допустим, что ни одно из множеств S и T не является начальным отрезком другого. Общим началом S и T назовём
		такое подмножество J $\subset$ S $\cap$ T,  которое есть начальный отрезок как S, так и T. Заметим, что
		объединение I множества всех общих начал S и T само есть их общее начало. (В самом деле, если x $\in$ I, то для
		некоторого общего начала J имеем $x$ $\in$ J, а тогда $\forall y \in S \ (y < x \Rightarrow y \in J \subset I)$
		и аналогично для T.)

		Если I совпадает с одним из множеств S или T, то (i) доказано. В противном случае рассмотрим $s = min_S$
		(S$\setminus$I) и t = $min_T$ (T$\setminus$I), где min берётся по множествам S и T, соответственно. Тогда $S_s$
		= I = $T_t$. В силу корректности S и T получаем s = $\varphi (S_s)$ = $\varphi (T_t)$ = t, то есть I $\cup
		\{s\}$ есть общее начало T и S, расширяющее I, что не возможно.

		(ii) Пусть $\Sigma$ — семейство корректных множеств и U = $\cup \Sigma$.

		Множество (U, <) линейно упорядочено по утверждению (i). (В самом деле, если x, y $\in$ U, то для некоторых
		корректных множеств S, T $\in \Sigma$ имеем x $\in$ S и y $\in$ T. Возьмём из них большее и воспользуемся его
		линейной упорядоченностью.)

		Каждое S $\in \Sigma$ есть начальный отрезок U. Иначе найдётся x $\in$ S и y < x такой, что y $\in$ U $\setminus$ S.
		Тогда для некоторого корректного T $\in \Sigma$ имеем y $\in T \setminus S$,  значит T не является начальным
		отрезком S. По свойству (i) множество S должно быть начальным отрезком T, что противоречит тому, что y < x
		$\in$ S и y $\notin$ S.

		Докажем, что (U, <) вполне упорядочено. Пусть Y $\subset$ U непусто. Рассмотрим любой y $\in$ Y и корректное
		множество S $\in \Sigma$ такое, что y $\in$ S. Поскольку Y $\cap$ S непусто и вполне упорядочено (как
		подмножество S), существует x = $min_S (Y \cap S) \in S$. Поскольку S есть начальный отрезок U, x также будет
		наименьшим элементом Y в U.

		Осталось проверить, что x = $\varphi (U_x)$ для любого x $\in$ U. Выберем S $\in \Sigma$ такое, что x $\in$ S.
		Заметим, что $U_x = S_x$, поскольку S есть начальный отрезок U. Следовательно, x = $\varphi (S_x) = \varphi
		(U_x)$.
	\end{proof}

	Рассмотрим теперь множество $\Sigma$ всех корректных подмножеств X и положим U = $\cup \Sigma$.  Поскольку U вполне
	упорядочено и, в частности, является цепью, оно имеет строгую верхнюю грань $\varphi$ (U). Тогда U $\cup \{\varphi
	(U) \}$ есть собственное расширение U и является корректным множеством, что невозможно по определению $\Sigma$. Лемма
	Цорна доказана.
\end{proof}

%\section{Вывод теоремы Цермело из леммы Цорна. Вывод аксиомы выбора из теоремы Цермело.}

\begin{proof}[Вывод теоремы Цермело из леммы Цорна.]
Вполне упорядоченное множество (S, $<_S$) назовём вполне упорядоченным подмножеством X, если S $\subset$ X. Для данного множества X рассмотрим совокупность W(X) всех его вполне упорядоченных подмножеств. На W(X) определим отношение строгого частичного порядка $\prec$ следующим образом:

\begin{center}
(S, $<_S$) $\prec$ (T, $<_T$), если и только если S $\subset$ T есть собственный начальный отрезок (T, $<_T$), и $<_S$ совпадает с ограничением $<_T$ на S. 
\end{center}

Докажем, что (W(x), $\prec$) удовлетворяет условию леммы Цорна. Рассмотрим любую цепь C $\subset$ W(X).  Цепи C соответствует возрастающая по включению цепь подмножеств X и возрастающая по включению цепь бинарных отношений на этих множествах. Обозначим через U объединение этой цепи подмножеств X, а через $<_U$ $U$ — объединение соответствующей цепи отношений. Ясно, что $<_U$ $U$ есть отношение линейного порядка на U и каждое (S, $<_S) \in$ C есть начальный отрезок (U, $<_U$). Отсюда получаем, что (U, $<_U$) — вполне упорядоченное подмножество X. Таким образом, (U, $<_U$) есть элемент W(X) и верхняя грань цепи C.

Применяя лемму Цорна получаем, что в (W(x), $\prec$) найдётся некоторый максимальный элемент (M, $<_M$). Тогда M обязано совпадать со всем X: в противном случае мы можем взять a $\in$ X $\setminus$ M и продолжить порядок $<_M$ на большее множество N = M $\cup \{ a \}$ полагая x $<_N$ a для всех x $\in$ M.  (Формально, $<_N$ будет объединением $<_M$ и $\{ <x,a> \mid x \in M \}$.) Тогда (N, $<_N$) будет вполне упорядоченным подмножеством X и (M, $<_M$) $\prec$ (N, $<_N$), что противоречит максимальности (M, $<_M$).
\end{proof}

\begin{proof}[Вывод аксиомы выбора из теоремы Цермело.]
Пусть S - данное семейство непустых множеств. По теореме Цермело множество U = $\cup$ S может быть вполне упорядочено. Для каждого  x $\in$ S имеем x $\subset$ U. Пусть min(x) означает наименьший элемент x в смысле порядка на U. Поскольку $\varnothing \notin$ S, соответствие x $\mapsto$ min(x) является функцией выбора на S.
\end{proof}
%\section{Формулы логики высказываний, понятие подформулы. Истинностное значение формулы при данной оценке
пропозициональных переменных. Таблица истинности формулы. Выполнимые формулы, тавтологии, тождественно ложные
формулы и их взаимосвязь. Алгоритм распознавания выполнимости}

\begin{definition}
	Алфавитом будем называть любое непустое множество. Его элементы называются символами (буквами).
\end{definition}

\begin{definition}
	Словом в алфавите $\sigma$ называется конечная последовательность элементов $\sigma$.
\end{definition}

Пусть задан некоторый алфавит Var символов, называемых пропозициональными переменными. Знаки $\neg, \land, \lor,
\to$(и аналогичные знаки) называются пропозициональными связками или булевыми связками.

\begin{definition}
	Формулы логики высказываний являются словами в алфавите, состоящем из пропозициональных переменных (Интуитивно,
	пропозициональные переменные интерпретируются как высказывания), пропозициональных связок и скобок: ( и ).
\end{definition}

\begin{definition}
	Множество формул $F_m$ логики высказываний порождается из множества Var по следующим правилам:

	1. Если P $\in$ Var, то P — формула.

	2. Если A — формула, то $\neg$ A — формула.

	3. Если A и B — формулы, то (A $\land$ B), (A $\lor$ B), (A $\to$ B) — формулы.
\end{definition}

\begin{definition}
	Подформулами формулы A называются все те формулы, которые входят в любое построение A. Подформула формулы A,
	отличная от самой формулы A, называется собственной подформулой формулы A.
\end{definition}

\begin{definition}
	Обозначим $\mathbb{B}$ $\rightleftarrows$ $\{$И, Л$\}$ $\rightleftarrows$ $\{$0, 1$\}$. Функции f:
	$\mathbb{B}^n$ $\to$ $\mathbb{B}$ называются булевыми функциями.

\end{definition}

\begin{definition}
	Оценкой пропозициональных переменных (или просто оценкой) называется произвольная функция $f: Var \to \mathbb{B}$.
\end{definition}

\begin{definition}
	Истинностное значение (или просто значение) формулы при данной оценке f определяется индукцией по построению
	формулы в соответствии со следующей таблицей ($T =$ И (истина), $F =$ Л (ложь)).

	$$
	\begin{array}{cc|c|c|c|c}
		{A} & {B} & {\neg A} & {A \wedge B} & {A \vee B} & {A \rightarrow B} \\
		\hline F & F & T & F & F & T \\
		F & T & T & F & T & T \\
		T & F & F & F & T & F \\
		T & T & F & T & T & T 
	\end{array}
	$$

\end{definition}

\begin{definition}
	Таблицей истинности (или истинностной таблицей) формулы A над переменными $P_1$,...,$P_n$ называется таблица,
	указывающая значения формулы A при всех возможных оценках переменных $P_1$,...,$P_n$.
\end{definition}

\begin{definition}
	Пропозициональная формула (выражение, построенное из пропозициональных букв при помощи логических
	(пропозициональных) связок), истинная хотя бы при одной оценке пропозициональных переменных, называется
	выполнимой. Множество формул Г назывется выполнимым, если существует оценка f, при которой истинны одновременно
	все формулы из Г.
\end{definition}

\begin{definition}
	Пропозициональная формула, истинная при каждой оценке пропозициональных переменных, называется тавтологией
	(тождественно истинной).
\end{definition}

\begin{definition}
	Пропозициональная формула, ложная при каждой
	оценке пропозициональных переменных, называется тождественно ложной.
\end{definition}

\begin{proposition}
	Следующие условия равносильны:

	(i) Формула $A$ тождественно ложна.

	(ii) Формула $A$ не является выполнимой.

	(iii) Формула $\lnot A$ — тавтология.
\end{proposition}

\begin{definition}
	В приложениях часто встречается задача проверки пропозициональной формулы на выполнимость. Наиболее
	прямолинейный алгоритм её решения состоит в построении всей таблицы истинности формулы, то есть перебора $2^n$
	всех возможных оценок.
\end{definition}

%\section{Связь между формулами логики высказываний от n переменных и булевыми функциями. Теорема о функциональной
полноте.}

\begin{definition}
	Таблицей истинности (или истинностной таблицей) формулы A над переменными $P_1$,...,$P_n$ называется таблица,
	указывающая значения формулы A при всех возможных оценках переменных $P_1$,...,$P_n$.

	Таким образом, таблица истинности формулы A над n переменными задаёт булеву функцию $\varphi_A : \mathbb{B}^n$
	$\to$ $\mathbb{B}$. Функция $\varphi_A$ определяется равенством
	\begin{center}
		$\varphi_A (\vec{x}) = f_{\vec{x}}$ (A),
	\end{center}
	верным для всех наборов $\vec{x} \in \mathbb{B}^n$
\end{definition}

\begin{theorem}[о функциональной полноте]
	Для любой функции $\varphi$: $\mathbb{B}^n$ $\to$ $\mathbb{B}$ найдётся такая формула A от n переменных, что
	$\varphi = \varphi_A$. При этом можно считать, что A содержит лишь связки $\neg$ и $\lor$.
\end{theorem}

\begin{proof}
	Равенство $\varphi = \varphi_A$ означает, что для всех $\vec{x} \in \mathbb{B}^n$
	\begin{center}
		$\varphi (\vec{x}) = \varphi_A (\vec{x}) = f_{\vec{x}} (A)$
	\end{center}
	Для x $\in \mathbb{B}$ положим
	\begin{equation*}
		P^x= 
		\begin{cases}
			P  &\text{ если $x = $И} \\
			\neg P &\text{если $x = $Л}
		\end{cases}
	\end{equation*}
	Для произвольного $\vec{x} = (x_1,...,x_n) \in \mathbb{B}^n$ обозначим
	\begin{center}
		$A_{\vec{x}}$   $\rightleftarrows  \bigwedge_{i=1}^n P_i^{x_i}$
	\end{center}
	Легко видеть, что формула $A_{\vec{x}}$ истинна лишь при оценке $f_{\vec{x}}$. Другими словами, для любой оценки f
	\begin{center}
		f($A_{\vec{x}}$) = И $\Longleftrightarrow$ f = $f_{\vec{x}}$ (1)
	\end{center}
	Для данной функции $\varphi$ пусть список $\vec{x_1}$,...,$\vec{x_m}$ исчерпывает все наборы $\vec{x} \in
	\mathbb{B}$ для которых $\varphi(\vec{x})$ = И, то есть 
	\begin{center}
		$\varphi(\vec{x}) = $И $\Longleftrightarrow$ $\exists j \ \vec{x} = \vec{x}_j$ (2)
	\end{center}
	Положим теперь
	\begin{center}
		A $\rightleftarrows \bigvee_{j=1}^m A_{\vec{x}_j}$
	\end{center}
	тогда: 
	\begin{center}
		$f_{\vec{x}}$ (A) = И $\Longleftrightarrow \exists j \ f_{\vec{x}} (A_{\vec{x}_j})$\\
		$\Longleftrightarrow \exists j \vec{x} = \vec{x}_j$ по (1)\\
		$\Longleftrightarrow \varphi (\vec{x}) =$И по (2)
	\end{center}
	Заметим теперь, что конъюнкция выражается через дизъюнкцию и
	отрицание, поскольку формула A $\land$ B равносильна $\neg(\neg A \lor \neg B)$.  Поэтому, формулы
	$A_{\vec{x}}$ могут быть переписаны без использования знака $\land$. 
\end{proof}

%\section{Равносильность формул логики высказываний, связь с тождественной истинностью. Важнейшие равносильности.
Свойство замены подформулы на равносильную}.
\begin{definition}
	Формулы A и B называются \textbf{равносильными (эквивалентными)}, обозначение $A\equiv B$,
	если при каждой оценке пропозициональных переменных значение A совпадает со значением B.\\
	Другими словами, $\varphi_A=\varphi_B$ \label{formula}
\end{definition}
\begin{example}
	${P}\to{Q}  \equiv \neg Q\to{\neg P} $
\end{example}

\begin{center}
	\textbf{Связь с тождественной истинностью}
\end{center}

\textit{(i) Отношение $\equiv$ рефлексивно,симметрично и
транзитивно. То есть является отношением эквивалентности.}

\textit{(ii) Формулы А и B равносильны тогда и только тогда, когда формула ${A}\leftrightarrow{B}$ является
тавтологией.}
\textit{(iii)Формула A тавтология тогда и только тогда, когда ${A}\equiv \top $.}

\begin{center}
	\textbf{Основные равносильности}

	$A \wedge B  \equiv B \wedge A $ \\
	$A \wedge(B \wedge C)  \equiv(A \wedge B) \wedge C $ \\
	$A \wedge A  \equiv A $ \\
	$A \wedge(B \vee C)  \equiv(A \wedge B) \vee(A \wedge C)$ \\
	$A \vee(A \wedge B)  \equiv A$ \\
	$ \neg(A \wedge B)  \equiv \neg A \vee \neg B$ \\
	$\neg \neg A  \equiv  A $ \\
	{---------------------------------------------}

	$A \vee B  \equiv B \vee A $\\
	$A \vee(B \vee C)  \equiv(A \vee B) \vee C $ \\
	$A \vee A  \equiv A $  \\
	$A \vee(B \wedge C)  \equiv(A \vee B) \wedge(A \vee C) $ \\
	$A \wedge(A \vee B)  \equiv A $ \\
	$\neg(A \vee B)  \equiv \neg A \wedge \neg B $ \\
	$A \rightarrow B  \equiv \neg A \vee B $

\end{center}


\begin{center}
	\textbf{Замена подформулы на равносильную}
\end{center}

Если C и D — формулы, а P — пропозициональная
переменная, то через $C[P/D]$ обозначим результат подстановки формулы D вместо P в формулу C.
Формальное определение даётся с помощью индукции по построению
формулы C.
\begin{center}
	$P[P / D] \rightleftharpoons D$

	$Q[P / D] \rightleftharpoons Q,$ если Q— переменная, отличная от P

	$(\neg A)[P / D] \rightleftharpoons \neg(A[P / D]) $

	$(A \wedge B)[P / D] \rightleftharpoons(A[P / D] \wedge B[P / D])$

	$(A \vee B)[P / D] \rightleftharpoons(A[P / D] \vee B[P / D])$

	$(A \rightarrow B)[P / D] \rightleftharpoons(A[P / D] \rightarrow B[P / D])$

\end{center}

\begin{example}
	Пусть $C=\left(P_{1} \rightarrow P_{2}\right) \rightarrow P_{2} $ и $D=P_{3} \rightarrow P_{2}$

	$C\left[P_{2} / D\right]=\left(P_{1} \rightarrow\left(P_{3} \rightarrow P_{2}\right)\right)
	\rightarrow\left(P_{3} \rightarrow P_{2}\right)$

\end{example}


\begin{theorem}[О подставновке]
	Если $A$ -- тавтология, $B$ -- произвольная формула, а $P$ -- пропозициональная переменная, то $A[P/B]$
	-- тавтология.
\end{theorem}
\begin{proof}
	Рассмотрим произвольную оценку $g$. Обозначим через $g'$ оценку, полученную из $g$ присвоением переменной $P$
	значения $g(B)$. Индукцией по построению $C$ можно доказать, что $g(C[P/B])=g'(C)$  для любой формулы $C$.
	Положим $C=A$. Так как формула A истинна при оценке $g'$, то формула $A[P/B]$ истинна при оценке $g$.
\end{proof}

\begin{theorem}
	Пусть A,B,C -- формулы, а P -- пропозициональная
	переменная. Если $A\equiv B$ , то $A[P/C] \equiv B[P/C]$.
\end{theorem}
\begin{proof}
	Пусть $ {A}\equiv{B} $ . В силу $(ii)$ ,  ${A}\leftrightarrow{B}$  - тавтология. По теореме о подстановке,
	${A}\leftrightarrow B[P/C]$ - тавтология.
	Из определения, что $(A \rightarrow B)[P / D] \rightleftharpoons(A[P / D] \rightarrow B[P / D])$  следует, что
	$(A \leftrightarrow B)[P / C] $ совпадает с 
	$(A[P / C] \leftrightarrow B[P / C])$. В силу (ii) получаем, что $A[P/C] \equiv B[P/C]$.
\end{proof}
\begin{theorem}[Теорема о замене подформулы на эквивалентную]
	Пусть A,B,C формулы, а P -- пропозициональная переменная. Если $A\equiv B$, то $C[P/A]\equiv C[P/B]$.
\end{theorem}
\begin{proof}
	Теорема доказывается индукцией по построению формулы $C$ (c использованием предыдущей теоремы).
\end{proof}
\begin{example}
	Формулы $A$ и $B$ выполнимы, а $A[P1/B]$ нет.  $A=\neg P_{1}$ , $B=P_{2} \vee \neg P_{2}$
\end{example}

%\section{Дизъюнктивные и конъюнктивные нормальные формы. Приведение формул логики высказываний к совершенной
дизъюнктивной (конъюнктивной) нормальной форме. Единственность совершенной дизъюнктивной
нормальной формы.}

\begin{definition}
	\textbf{Литералами} называются переменные и их отрицания.
\end{definition}
\begin{example}
	Формула А является литералом, а $ \neg\neg A$ - нет.
\end{example}

\begin{definition}
	\textbf{Элементарной конъюнкцией} называем формулу вида $\bigwedge^{n}_{i=1}L_{i}$, где $ L_{i} $ - литералы.

	\textbf{Дизъюнктивной нормальной формой (ДНФ)} называем формулу вида $\vee^{m}_{j=1}C_{j}$ , где $ C_{j} $ -
	элементарные коньюнкции.

	\textbf{Элементарной дизъюнкцией} называем формулу вида
	$\vee^{n}_{i=1}L_{i}$, где $ L_{i} $ - литералы.  

	\textbf{Конъюнктивной нормальной формой (КНФ)} называем формулу вида $\wedge^{m}_{j=1}C_{j}$ , где $ C_{j} $ -
	элементарные дизъюнкции.
\end{definition}

\begin{definition}
	Формула A называется \textbf{совершенной ДНФ}, если A— ДНФ и 
	\begin{itemize}
		\item Каждая элементарная конъюнкция имеет вид 
			$ A_{\vec{x}} \rightleftarrows \bigwedge_{i=1}^{n} P_{i}^{x_{i}} $
			для некоторого $\vec{x} = (x_1,\ldots,x_n) \in \mathbb{B}^n$ 
		\item $A=\bigvee_{j=1}^m A_{\vec{x}_j}$, где $\vec{x}_1,\ldots,\vec{x}_m \in \mathbb{B}^{n}$ попарно
			различны и взяты в лексикографическом порядке.
	\end{itemize}
	Аналогично определяется \textbf{совершенная КНФ}, с заменой дизъюнкций
	на конъюнкции и наоборот.
\end{definition}
\begin{remark} 
	Удобно расширить множество формул константами
	$\perp$ (ложь) и $\top$ (истина). Тем самым, формулами считаются и все выражения, построенные с помощью булевых
	связок из переменных и этих констант. Считаем $\perp$ совершенной ДНФ, а $\top$ -- совершенной КНФ.\\
	Совершенные ДНФ и КНФ перестают быть совершенными, если рассматривать их как формулы от более широкого набора
	переменных. Поэтому имеет смысл говорить о совершенных ДНФ и КНФ лишь относительно некоторого фиксированного
	набора переменных.
\end{remark}

\begin{theorem}
	Каждая пропозициональная формула A равносильна некоторой совершенной дизъюнктивной нормальной форме. Причем СДНФ
	любой формулы A единственна.
\end{theorem}

\begin{proof}
	\textbf{Существование.} Если формула A тождественно ложна, в качестве её ДНФ можно взять $\perp$. В противном
	случае достаточно заметить, что формула, построенная в доказательстве теоремы о функциональной полноте для
	функции $\varphi_{A}$ есть совершенная ДНФ.

	\textbf{Единственность.}
	Заметим, что совершенные ДНФ эквивалентных формул (графически) совпадают. И правда, для совершенной ДНФ каждая
	элементарная конъюнкция определяет некоторую выполняющую оценку, а сама ДНФ -- множество всех таких оценок.
	Отсюда и следует единственность.
\end{proof}
{Аналогичная теорема и для СКНФ.}
\begin{theorem}
	Каждая пропозициональная формула A равносильна некоторой совершенной коньюктивной нормальной форме. Причем СКНФ
	любой формулы A единственна.
\end{theorem}

\begin{proof}
	\textbf{Существование.} Если формула A тождественно истинна, в качестве её СКНФ можно взять $\top$. В противном
	случае, построим СДНФ для $\neg A$. Берем отрицание этого СДНФ. Нетрудно заметить, что это СКНФ для нашей
	функции.  

	\textbf{Единственность.}
	Заметим, что совершенные КНФ эквивалентных формул (графически) совпадают.И правда, для совершенной СКНФ каждая
	элементарная дизъюнкция определяет некоторую невыполняющую оценку, а сама КНФ -- множество всех таких оценок.
	Отсюда и следует единственность. (Или просто из единственность СДНФ для отрицания A).
\end{proof}

%\section{Понятие сигнатуры и модели (алгебраической системы) данной сигнатуры. Примеры моделей: стандартная модель
арифметики; кольцо целых чисел; кольца многочленов и матриц порядка n над данным полем; евклидова плоскость в
сигнатуре элементарной геометрии Тарского $(R^{2}; =; B; \cong)$; модель Пуанкаре геометрии Лобачевского.}

\begin{definition}
	Пусть M -- непустое множество. \textit{n-арным предикатом} на M называется произвольное подмножество $Q
	\subseteq M^{n}$.

	\textit{n-aрной функцией} на M называется функция f: $M^{n}\to M$.
	Если Q -- n-арный предикат, то часто пишут $Q(x_1, \ldots , x_n)$ вместо $\langle x_1, \ldots , x_n \rangle \in
	Q$. Аналогично, $f(x_1, \ldots , x_n)$ означает $f (\langle x_1, \ldots , x_n \rangle)$.

	\textit{Константами} называем произвольныe элементы множества M.

	\textbf{Сигнатура} -- это набор из трех множеств имён $\varSigma =  \langle Const, Func, Pred \rangle $, где
	Const -- множество имён констант, Func -- функциональных символов, Pred -- предикатных символов, и функции
	валентности, сопоставляющей каждому предикатному и функциональному символу число его
	аргументов.$\label{formula2}$

	$$arity : Pred_\Sigma \cup Func_\Sigma \rightarrow \mathbb{N} \backslash\{0\}$$
\end{definition}

\begin{definition}
	\textbf{Алгебраическая система (или модель) сигнатуры $\varSigma $} есть непустое множество M вместе с
	отображением, сопоставляющим каждому предикатному символу P из $\varSigma $ некоторый предикат $P_{M}$ на M той же
	валентности, каждому функциональному символу f функцию $f_{M}$ на M той же валентности, и каждой символу $C \in
	Const_{\varSigma} $ константу $c_{M} \in M$. Такое отображение называется \textit{интерпретацией} $\varSigma $ на
	M. Множество M называется \textit{универсумом} или \textit{носителем} данной интерпретации (модели). Модель
	сигнатуры $\varSigma $ с носителем M обозначается \textbf{(M; $\varSigma$)}. $\label{formula3}$
\end{definition}

\begin{center}
	\textbf{Примеры моделей}
\end{center}

\begin{example} [Стандартная модель арифметики]
	$(\mathbb{N}; =, S, +, \times, 0)$ \\
	Здесь $S(x) \rightleftarrows x+1 $ есть одноместная функция следования на множестве N,
	а все остальные функции и предикаты имеют стандартный смысл.
\end{example}

\begin{example} [Кольцо целых чисел]
	$(\mathbb{Z}; =,+,-,\times,0,1)$. Здесь ''-'' это одноместная функция, отображающая x на -x, а все остальные
	функции и предикаты имеют стандартный смысл.
\end{example}

\begin{example} 
	Любое другое кольцо (с единицей) может рассматриваться как
	модель той же сигнатуры, например
	\begin{itemize}
		\item $\mathbb{Q}[x]$ -- кольцо многочленов над полем Q.
		\item $\mathbb{Z}_{n}$ -- кольцо вычетов по модулю n.
		\item $M_{n}(\mathbb{R})$ -- кольцо матриц порядка n.
	\end{itemize}
\end{example}

\begin{example} [Евклидова плоскость в сигнатуре элементарной геометрии Тарского] $(R^{2};=;B;\cong)$
	\begin{itemize}
		\item $\mathbb{R}^2$ -- множество точек евклидовой плоскости;
		\item $B(a, b ,c)$ -- трёхместный предикат ''точка b лежит на прямой ac между точками a и c'';
		\item $\cong$ -- четырёхместный предикат (записываемый $ab \cong cd$) ''отрезки, задаваемые парами точек ab и
			cd, имеют равные длины''.
	\end{itemize}
\end{example}

\begin{example} [Модель Пуанкаре геометрии Лобачевского]
	$(H^2; =,\cong, B)$, где
	\begin{itemize}
		\item $H^2 \rightleftarrows \{z \in \mathbb{C} : Im(z)>0 \}$ - множество точек верхней евклидовой полуплоскости;
		\item $B(a, b ,c)$ -- трёхместный предикат «точка b лежит между точками a и c на полуокружности (или полупрямой),
			проходящей через a, c и ортогональной вещественной оси»;
		\item $\cong$ -- четырёхместный предикат (записываемый $ab \cong cd$) ''отрезки, задаваемые парами точек ab и cd,
			имеют равные длины в смысле метрики Пуанкаре''.
	\end{itemize}
\end{example}

%\section{Язык логики предикатов первого порядка данной сигнатуры. Свободные
и связанные переменные, термы, формулы. Замкнутые формулы. Подстановка терма вместо переменной.}

\definition{
	Язык логики первого порядка $L_{\Sigma }$ определяется его сигнатурой $\varSigma $. Помимо всех
	символов сигнатуры, в алфавит языка $L_{\Sigma}$ входят два фиксированных счётных
	алфавита свободных и связанных переменных

	\begin{center}
		$ FrVar = \{a_{0}, a_{1}, a_{2}, \ldots \},$ \\
		$ BdVar = \{v_{0}, v_{1},v_{2}, \ldots \},  \label{formula4}  $

	\end{center}
	И следующие специальные символы:\\
	\textit{Булевы связки}: $\rightarrow$, $\neg$, $\wedge$ ,$ \vee $;\\
	\textit{Кванторы}: $\forall$ (квантор общности, «для всех»), $\exists$ (квантор существования, «существует»);\\
	\textit{Знаки пунктуации}: «(», «)» (скобки) и «,» (запятая).\\
	Произвольное слово в описанном алфавите называем выражением. Некоторые выражения называются \textit{термами} и
	\textit{формулами}.\\
	Множества термов и формул языка $L_{\Sigma }$ определяются индуктивно.
}

\definition{
	Множество термов $Tm_{\Sigma }$ есть наименьшее множество, замкнутое относительно следующих правил:\\
	1. Свободные переменные и константы -- термы.\\
	2. Если $f$ -- функциональный символ валентности n и $t_{1}, \ldots ,t_{n}  $ -- термы,
	то выражение $f(t_{1}, \ldots , t_{n})$ есть терм.
}


\definition{
	Множество формул $Fm_{\Sigma}$ есть наименьшее множество,
	замкнутое относительно следующих правил:\\
	1. Если P -- предикатный символ валентности n и $t_{1}, \ldots,t_{n}$ -- термы,
	то $P(t_{1}, \ldots , t_{n})$ есть формула (называемая \textit{атомарной формулой}).\\
	2. Если A, B -- формулы, то формулами являются также выражения $A\rightarrow B$, $\neg A$, $A\wedge B$, $ A\vee B $; \\
	3. Если A -- формула, и a -- свободная переменная, то для любой связанной
	переменной x, не входящей в A, выражения $(\forall x A[a/x])$ и $(\exists x A[a/x]) $ -- формулы. 
}

Формулы, в которые не входят кванторы, называются \textit{бескванторными}. Формулы и термы, в которые не входят
свободные переменные, называются \textit{замкнутыми}. Замкнутые формулы также называются \textit{предложениями}.



%\section{Семантика логики предикатов первого порядка. Расширение сигнатуры
данной модели константами. Значение замкнутого терма расширенной
сигнатуры в данной модели. Истинностное значение замкнутой формулы
расширенной сигнатуры в данной модели.}

Пусть M -- модель сигнатуры $\Sigma$. Обозначим через $ \Sigma(M)$ сигнатуру, получаемую из $  \Sigma$ добавлением новых символов констант ${\underline {c} : c \in M. \label{formula5} } $
 
 
\definition{

Пусть t -- замкнутый терм языка $L_{\Sigma (M) }$. Значение терма
t в модели M есть элемент $ t_{M} \in M  $, определяемый индукцией по построению t.\\
(i) Если $a \in M $ то  $\underline{a}_{M} \rightleftharpoons a$;\\
(ii) Если $ c \in Const_{\Sigma} $ , то $c_{M} $ есть данная нам интерпретация c.\\
(iii) Если t есть $f(t_{1},...,t_{n})  $, гдe $f \in Func_{\Sigma}$, то $ t_{M} \rightleftharpoons f_{M}((t_{1})_M,..,(t_{n})_M )$ \\
} 
\begin{definition}
Пусть A -- замкнутая формула языка $L_{\Sigma }$(M). \textit{Истинностное значение} формулы A в модели M определяется индукцией по построению A (oтношение $ M \vDash A $ читается ''формула А истина в модели M'') \label{formula6}  \\
1. $M \vDash P\left(t_{1}, \ldots, t_{n}\right) \stackrel{\text { def }}{\Longleftrightarrow} P_{M}\left(\left(t_{1}\right)_{M}, \ldots,\left(t_{n}\right)_{M}\right),$ если $A=P\left(t_{1}, \ldots, t_{n}\right)-$
атомарная формула;\\
2. $M \vDash(B \rightarrow C) \stackrel{\text { def }}{\Longleftrightarrow}(M \nvDash B \text { или } M \vDash C)$ \\
3. $M \vDash \neg B \stackrel{\text { def }}{\Longleftrightarrow} M \nvDash B$ \\
4. $M \vDash(B \wedge C) \stackrel{\text { def }}{\Longleftrightarrow}\left(M \vDash B \text{ и } M \vDash C\right)$\\
5. $M \vDash(B \vee C) \stackrel{\text { def }}{\Longleftrightarrow}(M \vDash B \text { или } M \vDash C)$ \\
6. $M \vDash(\forall x B[a / x]) \stackrel{\text { def }}{\Longleftrightarrow}$ для всех $x \in M \ M \vDash B[a / \underline{x}]$ \\
7. $M \vDash(\exists x B[a / x]) \stackrel{\text { def }}{\Longleftrightarrow}$ существует $x \in M \ M \vDash B[a / \underline{x}]$ \\
Если список $b_{1}, \ldots , b_{n}  $ coдержит все свободные переменные формулы A, а
$x_{1}, \ldots, x_{n} \in M,$ To $M \vDash A\left[b_{1} / \underline{x}_{1}, \ldots, b_{n} / \underline{x}_{n}\right]$ сокращенно записываем как $M \vDash$
$A\left[b_{1} / x_{1}, \ldots, b_{n} / x_{n}\right]$ или даже $M \vDash A\left[x_{1}, \ldots, x_{n}\right]$.

\end{definition}
\begin{remark}
Нельзя говорить об истинности или ложности незамкнутых
формул, поскольку их истинностные значения зависят от выбора значений параметров — входящих в формулу свободных переменных. Например, Формула $a+1 = b$  в стандартной модели арифметики может
быть как истинна, так и ложна, в зависимости от a и b.
\end{remark}
\section{Предикаты и функции, выразимые в данной модели. Выразимость предиката параллельности прямых в языке
элементарной геометрии и формулировка аксиомы о параллельных.}
\begin{definition}
	Пусть дана сигнатура $\sigma$ и её интерпретация с носителем модели $M$. Рассмотрим произвольную формулу
	$\varphi$ и набор переменных $x_1,\ldots,x_k$, среди которых содержатся все параметры $\varphi$. Получим k-местный
	предикат на $M$. Говорят, что этот предикат \textbf{выражается} формулой $\varphi$.

	Предикаты, для которых существует выражающая их формула, называются \textbf{выразимыми}. Соответствующие им
	области истинности в $M^k$ также называются выразимыми.
\end{definition}
\begin{definition}
	Функция выразима $f$, если предикат $$G(x_1,\ldots,x_n,y) \stackrel{def}{\Longleftrightarrow} f(x_1,\ldots,x_n)
	= y$$ выразим.
\end{definition}
\begin{example}
	Выразим в модели элементарной геометрии $(\mathbb{R}^2;=,\cong,B)$ предикат параллельности прямых. Для этого
	введём следующие предикаты:
	\begin{itemize}
		\item $a \ne b \rightleftharpoons \neg a=b$
		\item $c \in ab$ <<$c$ лежит на прямой $ab$>>: $$c\in ab \rightleftharpoons B(c,a,b)\vee B(a,b,c)\vee
			B(a,c,b)$$
		\item $ab\parallel cd$ <<$ab$ параллельна $cd$>>: 
	\end{itemize}
\end{example}
% vim: textwidth=115 colorcolumn=120

%\section{Гомоморфизмы и изоморфизмы моделей. Теорема о сохранении истинностного значения формулы при изоморфизме.
Автоморфизмы моделей, метод доказательства невыразимости с помощью автоморфизмов. Описание автоморфизмов моделей
$(\mathbb Z;\leqslant)$, $(\mathbb R^2;=,B)$ и $(\mathbb R^2;=,B,\cong)$ и примеры невыразимых предикатов в этих
моделях}
\begin{definition}
	Пусть $M$ и $M'$ -- две модели сигнатуры $\Sigma$. \textbf{Гомоморфизмом} моделей $M$ и $M'$ называется
	отображение $\varphi\colon M\to M'$, такое, что сохраняются все предикаты, функции и константы $\Sigma$:
	$$
		P_M(x_1,\ldots,x_n) \rightarrow P_{M'}(\varphi(x_1),\ldots,\varphi(x_n)),\quad\forall P\in
		\mathrm{Pred}_\Sigma
	$$
	$$
		\varphi(f_M(x_1,\ldots,x_n)) = f_{M'}(\varphi(x_1),\ldots,\varphi(x_n)),\quad\forall f\in\mathrm{Func}_\Sigma
	$$
	$$
		\varphi(c_M) = c_{M'},\quad\forall c\in\mathrm{Const}_\Sigma
	$$
\end{definition}
\begin{definition}
	\textbf{Изоморфизмом} моделей $M$ и $M'$ называется гомоморфизм $\varphi$, для которого существует обратный
	гомоморфизм $\psi\colon M'\to M$, такой, что $\psi\circ\varphi=\mathrm{id}_M$ и $\varphi\circ\psi =
	\mathrm{id}_{M'}$, где
	$\mathrm{id}_M$ -- тождественный гомоморфизм на $M$.
	Две модели \textbf{изоморфны}, если существует изоморфизм этих моделей.
\end{definition}

\begin{theorem}
Если $\varphi: M \rightarrow M^{\prime}$ -- изоморфизм, то для любой формулы $A\left(a_{1}, \ldots, a_{n}\right)$ и
	любых $c_{1}, \ldots, c_{n} \in M$

$$M \vDash A\left[c_{1}, \ldots, c_{n}\right] \Longleftrightarrow M^{\prime} \vDash
	A\left[\varphi\left(c_{1}\right), \ldots, \varphi\left(c_{n}\right)\right].$$
\end{theorem}

\begin{proof}
Индукция по построению $A$.
\end{proof}

\subsection{Доказательство невыразимости с помощью автоморфизмов.}

\begin{definition}
	\textbf{Автоморфизмом} модели называется изоморфизм модели на себя.
\end{definition}


Поскольку все определимые предикаты и функции сохраняются при автоморфизмах модели, для доказательства
невыразимости достаточно построить автоморфизм, не сохраняющий ту или иную функцию или предикат.

\begin{example}
Автоморфизмы в модели $(\mathbb{Z}, \leqslant)$ есть сдвиги на $n \in \mathbb{Z}$. В этой модели не выразима
	функция $+$. Отображение $\varphi: x \rightarrow x+1$ есть автоморфизм $(\mathbb{Z}, \leqslant)$, не
	сохраняющий $+$.
\end{example}

\begin{example}
Автоморфизмами модели $\left(\mathbb{R}^{2} ;=, B \right)$ являются все взаимно однозначные аффинные преобразования
	плоскости и только они. $\cong$ не выражается в модели, т.к. отношение не сохраняется при растяжении вдоль
	одной из осей.
\end{example}

\begin{example}
 Автоморфизмы модели $\left(\mathbb{R}^{2} ;=, B, \cong\right)$ суть все преобразования плоскости, являющиеся
	композицией гомотетии и движения. Не выражается $=$, т.к. не сохраняется при гомотетии.
\end{example}
% vim: colorcolumn=120 textwidth=115

%\section{Выполнимые формулы и множества формул языка первого порядка. Общезначимые и тождественно ложные формулы, их связь с выполнимыми формулами; примеры. Семантическое следование в логике первого порядка, его связь с понятиями выполнимости и общезначимости.}
\begin{definition}
	Формула $A(b_1\dots b_n)$ сигнатуры $\Sigma$ \textbf{выполнима} в модели $(M,\Sigma)$, если для некоторых
	констант $c_1,\dots,c_n$ предложение
	сигнатуры $\Sigma$ истинно. Формула $A$ сигнатуры $\Sigma$ выполнима, если она выполнима в некоторой модели
	$(M,\Sigma)$
\end{definition}
\begin{definition}
	Множество формул Г сигнатуры $\Sigma$ \textbf{выполнимо} в модели M, если существует функция $f: FrVar
	\rightarrow M$ такая, что при подстановке вместо каждой переменной $a_i$  константы $\underline{f(a_i)}$
	сигнатуры $\Sigma(M)$ все формулы истинны в M.\\Такую функцию $f$ будем называть \textbf{выполняющей} оценкой
	для Г.\\ Множество формул Г \textbf{выполнимо}, если Г выполнимо в некоторой модели.
\end{definition}
\begin{definition}
    Формула $A$ \textbf{общезначима} (тождественно истинна), если $\neg A$ не выполнима.\\
    Формула $A$ \textbf{тождественно ложна}, если $A$ не выполнима.
\end{definition}
\begin{definition}
	Пусть Г -- некоторое множество формул сигнатуры $\Sigma$ и $A$ -- формула той же сигнатуры. Говорят, что $A$
	\textbf{логически следует (или семантически следует)} из множества Г (обозначение $\text{Г}\vDash A$), если для
	любой модели $M$ сигнатуры $\Sigma$ формула $A$ истинна в $M$ при любой выполняющей оценке для множества Г.
	\label{formula7}
\end{definition}
Соотношения между понятиями выполнимости, общезначимости и логическим следованием в логике предикатов такие же, как
и в логике высказываний.
\begin{proposition}
    1) $A$ -- общезначима $\Longleftrightarrow \: \varnothing \vDash A$\\
    2)Г выполнимо  $\Longleftrightarrow \:\text{Г}\nvDash \perp$\\
    3)$\text{Г} \vDash A \Longleftrightarrow \: \text{Г}\cup (\neg A)$ не выполнимо
\end{proposition}
\begin{proposition}
	$\left\{B_{1}, \ldots, B_{n}\right\} \vDash A \Longleftrightarrow\left(\bigwedge_{i=1}^{n} B_{i}\right)
	\rightarrow A$ -- общезначима
\end{proposition}

%\section[Равносильности формул языка первого порядка, важнейшие равносильности. Переименовывание связанных
переменных. Предварённая форма формулы.]{\sloppy Равносильность формул языка первого порядка, важнейшие
равносильности. Переименование связанных переменных. Приведение формулы языка первого порядка к предварённой
форме.}
\begin{definition}
	Формулы $A$ и $B$ сигнатуры $\Sigma$ \textbf{равносильны} (обозначение $A\equiv B$), если для любой модели
	$(M,\Sigma)$ и оценки $f$ на $M$
    $$M \vDash f(A) \Longleftrightarrow M \vDash f(B).$$
\end{definition}
Пусть список $b_1 \dots b_n$ содержит все свободные переменные $A$,$B$.
\begin{proposition}
	$A\equiv B$, если и только если в любой модели $M$ формулы $A$ и $B$ определяют один и тот же предикат, то есть
	если $A_M=B_M$ (для данного набора переменных).
\end{proposition}
\begin{proposition}
1) Отношение $\equiv$ рефлексивно,симметрично и транзитивно.\\
2) $A\equiv B$, если и только если формула $A \leftrightarrow B$ общезначима.\\
3) Формула $A$ общезначима тогда и только тогда, когда $A \equiv$\textup{T}.
\end{proposition}
Перечислим основные равносильности с кванторами
\begin{lemma}[замена связанной переменной]
	Если $x,y \in BdVar$ не входят в формулу $A$, то $\forall \:x A[a / x] \equiv \forall y A[a / y]$ u $\exists x
	A[a / x] \equiv \exists y A[a / y] $
\end{lemma}
\begin{proof}
    Для квантора существования рассуждение аналогично.\\
    $\begin{aligned}
    M \vDash \forall x A[a / x] & \Longleftrightarrow M \vDash A[a / c] \text{для всех} c \in M\\
    & \Longleftrightarrow M \vDash \forall y A[a / y]
    \end{aligned}$
\end{proof}
\begin{lemma}
    Если $X\in BdVar$ не входит в формулы $A,B$, то $(\forall x A[a / x] \vee B) \equiv \forall x(A[a / x] \vee B)$
\end{lemma}
\begin{proof}
	 Прежде всего заметим, что правая часть эквивалентности, так же как и левая часть, является формулой. В самом
	 деле, выберем $a'\in FrVar$, не входящую в $A,B$.Тогда $B[a'/x]=B,A[a/x]=A[a/a'][a'/x]$ и тем самым $\forall x
	 \:(A[a/x]\lor B)$ совпадает с  $\forall x\:(A[a/a']\lor B)[a'/x]$ Получаем
     \begin{center}
         $\begin{aligned}
    M \vDash \forall x(A[a / x] \vee B) & \Longleftrightarrow M \vDash(A[a / c] \vee B) \ \text{для всех} \ c \in M \\
    & \Longleftrightarrow(M \vDash B \ \text{или для всех} \ c \in M \ M \vDash A[a / c]) \\
    & \Longleftrightarrow(M \vDash B \ \text{или} \ M \vDash \forall x \  A[a / x]) \\
    & \Longleftrightarrow M \vDash (\forall x \  A[a / x] \vee B)
    \end{aligned}$
     \end{center}
	 Аналогично обосновываются остальные равносильности, входящие в следующую таблицу (где предполагается, что
	 переменные $x,y$ не входят в формулы $A, B$).

$$
\begin{array}{|ccc|ccc|}
\hline \forall x A[a / x] & {\equiv} & {\forall y A[a / y]} & {\exists x A[a / x]} & {\equiv} & {\exists y A[a / y]} \\
{(\forall x A[a / x] \vee B)} & {\equiv} & {\forall x(A[a / x] \vee B)} & {(\exists x A[a / x] \vee B)} & {\equiv}
	& {\exists x(A[a / x] \vee B)} \\
{(\forall x A[a / x] \wedge B)} & {\equiv} & {\forall x(A[a / x] \wedge B)} & {(\exists x A[a / x] \wedge B)} &
	{\equiv} & {\exists x(A[a / x] \wedge B)} \\
{\neg \forall x A[a / x]} & {\equiv} & {\exists x \neg A[a / x]} & {\neg \exists x A[a / x]} & {\equiv} & {\forall
	x \neg A[a / x]} \\
\hline
\end{array}
$$
\end{proof}
\begin{definition}
	Обогатим язык логики первого порядка пропозициональной переменной $P$. Можно считать $P$ нульместным
	предикатным символом. Распостраним на расширенный язык все синтаксические понятия, включая понятие формулы ($P$
	считается атомарной формулой). Запись $C[P/A]$ означает результат замены всех вхождений $P$ в формулу $C$ на
	$A$.Заметим, что $C[P/A]$ всегда является формулой. Для этого достаточно, чтобы связанные переменные $A$ не
	входили в $C$.
\end{definition}
\begin{lemma}
	$C[P/A]$ -- формула, если и только если любое вхождение P в формулу C не находится в области действия квантора
	по переменной $x\in BdVar$, входящей в A.
\end{lemma}
\begin{proof}
     Необходимость этого условия очевидна. Достаточность доказывается простой индукцией по построению формулы $C$.
\end{proof}
\begin{definition}
    Говорим, что разрешена подстановка формулы $A$ вместо $P$ в $C$, если выполнено условие предыдущей леммы.
\end{definition}
\begin{lemma}
	1) Если $A\equiv B$,то $\neg A \equiv \neg B$ Если $A_1\equiv B_1$ и $A_2\equiv B_2$,то $A_1\wedge A_2\equiv
	B_1 \wedge B_2$,$A_1\lor A_2\equiv B_1 \lor B_2$,$A_1\rightarrow A_2\equiv B_1 \rightarrow B_2$\\
	2)Если $A\equiv B$ и $x \in BdVar$ не входит в $A,B$, то $\forall x A[a / x] \equiv \forall x B[a / x]$ и
	$\exists x A[a / x] \equiv \exists x B[a / x]$
\end{lemma}
\begin{theorem}[Замена формулы на эквивалентную]
	Если $A \equiv B$ и разрешена подстановка формул $A,B$ вместо $P$ в $C$, то $C[P/A]\equiv C[P/B]$.
\end{theorem}
\begin{proof}
	\href{http://lpcs.math.msu.su/vml2019/2019_VML_Beklemishev_2_Logic.pdf}{Лекции Л.Д. Беклемишева}, с.~31--32.
\end{proof}
\begin{lemma}
    Пусть $y\in BdVar$ не входит в формулу $B$. Тогда $B[x/y]$ есть формула и $B[x/y] \equiv B$
\end{lemma}
\begin{proof}
	\href{http://lpcs.math.msu.su/vml2019/2019_VML_Beklemishev_2_Logic.pdf}{Лекции Л.Д. Беклемишева}, с.~32
\end{proof}
\begin{theorem}
	Пусть формула $A$ общезначима и разрешена подстановка формулы $C$ вместо $P$ в $A$, тогда общезначима формула
	$A[P/C]$.
\end{theorem}
\begin{proof}
	\href{http://lpcs.math.msu.su/vml2019/2019_VML_Beklemishev_2_Logic.pdf}{Лекции Л.Д. Беклемишева}, с.~32
\end{proof}
\begin{definition}
	Формула $A$ называется \textbf{предварённой}, если $A$ имеет вид $$\mathrm{Q} x_{1} \mathrm{Q} x_{2} \ldots
	\mathrm{Q} x_{n} A_{0}\left[b_{1} / x_{1}, \ldots, b_{n} / x_{n}\right],$$где $Q$ означает квантор $\forall$ или
	$\exists$, а формула $A_0$ бескванторная.
\end{definition}
Далее тут доказательства см. стр.33
\begin{center}
	\href{http://lpcs.math.msu.su/vml2019/2019_VML_Beklemishev_2_Logic.pdf}{Лекции Л.Д. Беклемишева}
\end{center}
\begin{theorem}
    Для любых $\alpha,\beta$ и любых формул $A,B$, не содержащих переменных из $\alpha,\beta$ имеем:\\
    1)$\alpha A \wedge B \equiv \alpha(A \wedge B) ; \alpha A \vee B \equiv \alpha(A \vee B)$\\
	2)$\neg \alpha A \equiv \bar{\alpha} \neg A$, где $\bar{\alpha}$ получается их $\alpha$ заменой всех символов
	$\exists$ на $\forall$ и наоборот.
\end{theorem}
\begin{theorem}
	Для каждой формулы $A$ можно указать эквивалентную ей предварённую формулу $A'$ от тех же свободных переменных.
	Такую формулу $A'$ называем предварённой формой формулы $A$.
\end{theorem}

%\section{Теория первого порядка, её аксиомы и теоремы. Модель данной теории.Понятие выполнимой теории. Примеры теорий: теория строгих частичных порядков, теория отношения эквивалентности, теория простых графов.}
\begin{definition}
    \textbf{Теорией} сигнатуры $\Sigma$ называем произвольное множество $T$ замкнутых формул языка $\mathcal{L}_{\Sigma}$. Элементы $A \in T$ называем нелогическими аксиомами 
\end{definition}
\begin{definition}
    Модель $(M,\Sigma)$ есть \textbf{модель теории  $T$ (теория $T$ выполнима в модели M) $T$} (обозначение $M\vDash T$), если для любой $A \in T \: M\vDash A$\\ \label{formula8}
    Теория называется выполнимой (или совместной), если она имеет хотя бы одну модель.
\end{definition}
\begin{example}
Теория отношения эквивалентности в сигнатуре с единственным бинарным \\предикатным символом $R$ задаётся следующими тремя нелогическими аксиомами:\\
1)$\forall \: x\: (x,x)$\\
2)$\forall \:x,y\: (R(x,y)\rightarrow R(y,x)$\\
3)$\forall \:x,y,z (R(x,y)\wedge R(y,z)\rightarrow R(x,z))$\\
$R$ есть отношение эквивалентности на множестве $M$, если и
только если $(M,R)\vDash T$ где $T$ — теория отношения эквивалентности.
\end{example}
\begin{example}
    Модель $(M,<)$ есть строгий частичный порядок, если в $(M,<)$ истинны следующие предложения:\\
    1)$\forall \:x,y,z\: (x<y\wedge y<z \rightarrow x<z)$\\
    2)$\forall \: \neg x<x$\\
    Это можно считать определением строгих частичных порядков. Аксиомы 1 и 2 задают теорию строгих частичных порядков.
\end{example}
\begin{example}
    Простой граф — это модель вида $(V,E)$ где $V$ — множество
    (называемое множеством вершин графа), а $E$ — бинарный предикат смежности, причём отношение $E$ симметрично и иррефлексивно:\\
    1)$\forall \: x \neg E(x,x)$\\
    2)$\forall \: x,y\: (E(x,y)\rightarrow E(y,x)$\\
    Аксиомы 1 и 2 задают теорию простых графов.
\end{example}
%\section{Теории первого порядка с равенством. Нормальные модели. Теорема о существовании нормальной модели у
выполнимой теории с равенством.Примеры теорий с равенством: теория групп, формальная арифметика.}
Пусть $\Sigma$ -- сигнатура, содержащая выделенный предикатный символ <<$ = $>>.
\begin{definition}
	Нормальной моделью называем модель $(M,\Sigma)$, в которой «=» интерпретируется как равенство $\{\,\langle x,x
	\rangle\mid x\in M \,\}$.
\end{definition}
\begin{definition}
	Аксиомы равенства для $\Sigma$ суть универсальные замыкания следующих формул:
	\begin{enumerate}
		\item аксиомы отношения эквивалентности для <<=>>
		\item $a_{1}=b_{1} \wedge a_{2}=b_{2} \wedge \ldots \wedge a_{n}=b_{n} \rightarrow\left(P\left(a_{1}, \ldots,
	a_{n}\right) \leftrightarrow P\left(b_{1}, \ldots, b_{n}\right)\right)$
		\item $a_{1}=b_{1} \wedge a_{2}=b_{2} \wedge \ldots \wedge a_{n}=b_{n} \rightarrow\left(f\left(a_{1}, \ldots,
	a_{n}\right)=f\left(b_{1}, \ldots, b_{n}\right)\right)$ для всех $f\in Func_\Sigma,P\in Pred_\Sigma$
	\end{enumerate}
\end{definition}
\begin{proposition}
	Если $(M,\Sigma)$ -- нормальная модель, то в $M$ истинны все аксиомы равенства.
\end{proposition}
\begin{definition}
	\textbf{Теорией с равенством} называем теорию в языке с равенством, содержащую все аксиомы равенства.
\end{definition}
\begin{theorem}
	Пусть $T$ -- теория с равенством. Если $T$ выполнима, то $T$ имеет нормальную модель.
\end{theorem}
\begin{proof}
	Пусть $M \vDash T$. Предикат $=_M$ есть отношение эквивалентности на $M$. Положим $M'=M/_{=_M}$
	-- множество классов эквивалентности и пусть $\varphi\colon M\to M'$сопоставляет любому $x\in M$ его класс
	эквивалентности $\varphi(x)\in M'$.Все функции и предикаты сигнатуры $\Sigma$ естественным образом переносятся с
	$M$ на $M'$. Полагаем
	\begin{center}
		$\begin{aligned}
			P_{M^{\prime}}\left(\varphi\left(x_{1}\right), \ldots, \varphi\left(x_{n}\right)\right) &
			\Longleftrightarrow P_{M}\left(x_{1}, \ldots, x_{n}\right) \\
			f_{M^{\prime}}\left(\varphi\left(x_{1}\right), \ldots, \varphi\left(x_{n}\right)\right) &
			\rightleftharpoons \varphi\left(f_{M}\left(x_{1}, \ldots, x_{n}\right)\right) \\
			c_{M^{\prime}} & \rightleftharpoons \varphi\left(c_{M}\right)
		\end{aligned}$
	\end{center}
	Заметим, что в силу истинности аксиом равенства в $M$ все функции и предикаты корректно определены на $M'$, и
	$M'$ -- нормальная модель.\\
	Индукцией по построению формулы $A$ проверяем
	\begin{center}
		$M \vDash A\left[x_{1}, \ldots, x_{n}\right] \Longleftrightarrow M^{\prime} \vDash
		A\left[\varphi\left(x_{1}\right), \ldots, \varphi\left(x_{n}\right)\right]$
	\end{center}
	Отсюда следует $M'\vDash T$
\end{proof}
\begin{example}
	Язык арифметики содержит один константный символ 0, один одноместный функциональный символ S и двухместные
	функциональные символы + и $\cdot$. Единственным (двухместным) предикатным символом языка является равенство. В
	стандартной интерпретации переменные принимают значения в множестве натуральных чисел N, символ 0
	интерпретируется как ноль, S как операция прибавления единицы, а + и $\cdot$ как сложение и умножение
	соответственно. Формула в языке арифметики со свободными переменными задаёт некоторый предикат. Если предикат
	можно выразить некоторой формулой в языке арифметики, то он называется арифметичным. 
\end{example}
\begin{example}
	$(M;=;\cdot;1)$ есть группа, если $M$ есть модель следующей теории (при условии что <<=>> в $M$ понимается как
	равенство):
	\begin{enumerate}
		\item $\forall x,y,z: x\cdot(y\cdot z)=(x\cdot y)\cdot z$
		\item $\forall x \ (1 \cdot x=x \wedge x \cdot 1=x)$
		\item $\forall x \ \exists y \ (x \cdot y=1 \wedge y \cdot x=1)$
	\end{enumerate}
\end{example}

%\section{Аксиомы и правила вывода исчисления предикатов. Выводимость в теории, простейшие свойства выводимости.
Доказуемые, опровержимые,независимые формулы для данной теории.}
Исчисление предикатов сигнатуры $\Sigma$ задаётся следующими схемами аксиом и правилами вывода.\\
Аксиомы:\\
А1: все тавтологии логики высказываний,\\
А2: $\forall x \ A[a/x] \rightarrow A[a/t]$,\\
A3: $A[a/t] \rightarrow \exists x \ A[a/x]$. \\
Схему аксиом A1 более аккуратно мы понимаем следующим образом. Если $A(P_1,\ldots,P_n)$ -- тавтология, то формула
$A[P_1/C_1,\ldots,P_n/C_n]$ есть аксиома исчисления предикатов для любых формул $C_1,\ldots,C_n$ сигнатуры
$\Sigma$\\
В аксиомах A2 и A3 $A$ -- любая формула сигнатуры $\Sigma$ и $t$ -- любой терм (переменная $x$ не входит в $A$).

\textbf{Правила вывода:}\\
R1. $\cfrac{A \quad A \rightarrow B}{B}$ (modus ponens)\\ \medskip 
R2. $\cfrac{A \rightarrow B}{A \rightarrow \forall x B[a / x]}$\\ \medskip 
R3. $\cfrac{B \rightarrow A}{\exists x B[a / x] \rightarrow A}$

В правилах R2 и R3 переменная a не входит в A (и x не входит в B). Правила R2 и R3 называются правилами Бернайса.
\begin{definition}
	\textbf{Выводом} в исчислении предикатов называется конечная
	последовательность формул, каждая из которых либо является аксиомой, либо получается из предыдущих формул по
	одному из правил вывода R1-R3.\\
	Формула $A$ называется \textbf{выводимой} в исчислении предикатов или теоремой исчисления предикатов
	(обозначение $\vdash A$), если существует вывод, в котором последняя формула есть $A$ \label{formula9}
\end{definition}
Будем считать, что \textbf{гипотезы являются замкнутыми формулами}, что соответствует понятию теории как множеству
замкнутых формул.
\begin{definition}
	\textbf{Выводом} в теории $T$ называется конечная последовательность формул, каждая из которых либо принадлежит
	множеству $T$, либо является логической аксиомой вида A1-A3, либо получается из предыдущих формул по одному из
	правил вывода R1-R3.\\
	Формула $A$ называется \textbf{выводимой (доказуемой)} в теории $T$ или \textbf{теоремой} $T$ (обозначение $T
	\vdash A$), если существует вывод в $T$,в котором последняя формула есть $A$ \label{formula10} \\ 
	Формула $A$ \textbf{опровержима} в $T$, если $T \vdash \neg A$\\
	Формула $A$ \textbf{независима} от $T$, если $T \nvdash A$ и $T \nvdash \neg A$ .
\end{definition}
Простейшие свойства отношения выводимости в теории для исчисления предикатов аналогичны свойствам отношения
выводимости из гипотез для исчисления высказываний.\\
1) Если $T \subseteq U$ и $T \vdash A$, то $U \vdash A$ (монотонность)\\
2) Если $T \vdash A$, то существует такое конечное множество $T_0 \subseteq T$, что $T_0 \vdash A$ (компактность)\\
3) Если $T \vdash A$ для каждой аксиомы $B \in T$ имеет место $U \vdash B$, то $U \vdash A$ (транзитивность).
\begin{definition}
	Пусть $T, \ U$ -- теории сигнатуры $\Sigma$.\\
	Теория $U$ содержит $T$, если для любой $A\in T\: U \vdash A$ (обозначение $U \vdash T$)\\
	Теории $T$ и $U$ (дедуктивно) эквивалентны, если   $U \vdash T$ и $T \vdash U$ (обозначение $U \equiv T$)
\end{definition}

%\section{Теорема о дедукции для исчисления предикатов.}
Обозначение: $T,A \vdash B$ вместо $T \cup \{A\} \vdash B$
\begin{theorem}
	Для любой теории $T$ и замкнутой формулы $A$
	\begin{center}
		$T, A \vdash B \Longleftrightarrow T \vdash A \rightarrow B$
	\end{center}
\end{theorem}
\begin{proof}
	Индукция по длине вывода из гипотез $T,A \vdash B$\\
	Если $B$ является аксиомой или принадлежит $T$, то искомый вывод выглядит так:
	\begin{center}
		$B$\\
		$B \rightarrow(A \rightarrow B)$\\
		$A \rightarrow B$   (MP)
	\end{center}
	Если формула $B$ совпадает с $A$, то $A \rightarrow B$ есть тавтология $A \rightarrow A$, то есть аксиома A1\\
	Если $B$ получена из некоторых предыдущих формул по правилу вывода
	modus ponens, то эти формулы имеют вид $C$ и $C\rightarrow B$ Согласно предположению индукции $T \vdash (A
	\rightarrow C)$ и $T \vdash (A \rightarrow (C \rightarrow B))$. Искомый вывод формулы $B$
	из множества гипотез  $T \cup \{A\}$ стоит из этих двух выводов и следующих
	формул:
	\begin{center}
		$\begin{array}{l}
			{(A \rightarrow(C \rightarrow B)) \rightarrow((A \rightarrow C) \rightarrow(A \rightarrow B))} \\
			{(A \rightarrow C) \rightarrow(A \rightarrow B)} \\
			{A \rightarrow B}
		\end{array}$
	\end{center}
	Допустим $B=(C \rightarrow \forall x D[a / x])$ получена из $C \rightarrow D$ по R2. По предположению индукции
	\begin{center}
		$T\vdash A \rightarrow(C\rightarrow D)$
	\end{center}
	Надо построить вывод
	\begin{center}
		$T\vdash A \rightarrow(C\rightarrow \forall\:x\:D[a/x])$
	\end{center}
	Рассмотрим тавтологию
	\begin{center}
		($P \rightarrow(Q\rightarrow R))\leftrightarrow (P\wedge Q\rightarrow R)$
	\end{center}
	Подставляя $A$ место $P$, $C$ вместо $Q$ и $D$ вместо $R$ получаем, что формула
	\begin{center}
		($A \rightarrow(C\rightarrow D))\leftrightarrow (A\wedge C\rightarrow D)$
	\end{center}
	выводима в исчислении предикатов.\\
	Таким образом, вывод $A \rightarrow(C\rightarrow D)$ в $T$ можно продолжить:
	\begin{center}
		$A \rightarrow(C \rightarrow D)$\\
		$(A \rightarrow(C \rightarrow D)) \rightarrow(A \wedge C \rightarrow D) \quad(\mathrm{A} 1)$\\
		$(A \wedge C) \rightarrow D$(MP)\\
		$(A \wedge C) \rightarrow \forall x D[a / x]$ (R2, $A$ замкнута)\\ $A \rightarrow(C \rightarrow \forall x D[a / x])$ (аналогично)
	\end{center}
	Правило R3 рассматривается аналогично.
\end{proof}

%\section{Общезначимость аксиом исчисления предикатов. Теорема о корректности исчисления предикатов.}
\begin{definition}
Теория $T$ \textbf{противоречива}, если существует формула $A$
такая, что $T\vdash A$ и  $T\vdash \neg A$.В противном случае теория $T$ называется непротиворечивой.
\end{definition}
\begin{corollary}[Из теоремы о дедукции получаем следующее следствие].\\ Пусть формула $T$ замкнута. Тогда теория $T \cup \{A\}$ противоречива $\Longleftrightarrow T \vdash \neg A$
\end{corollary}
Следующая теорема называется теоремой о корректности исчисления предикатов.
\begin{theorem}
Если $M\vDash T$ и $T \vdash B(b_1,\dots,b_n$), то $M \vDash B([b_1]/x_1,\dots,[b_n]/x_n)$ \\для любых $x_1\dots x_n \in M$
\end{theorem}
\begin{proof}
Индукция по длине вывода формулы $B$ в $T$. Если $B\in T$, то $M\vDash B$, поскольку $M\vDash T$.\\\medskip 
Рассмотрим случай, когда $B$ -- логическая аксиома вида A3, то есть $B=(A[a / t] \rightarrow \exists x A[a / x])$. Можно считать $B$ (после подстановки констант вместо свободных переменных) замкнутой формулой, а $t$ — замкнутым термом сигнатуры $\Sigma(M)$.\\\medskip 
Допустим $M \vDash A[a / t]$. Пусть $c \rightleftharpoons t_{M}$, тогда $M \vDash A[a / \underline{c}]$, а значит и $M \vDash \exists\:x A[a/x]$ (см. определение истинности формул). Тем самым доказано, что $M \vDash A[a / t] \rightarrow \exists x A[a / x]$\\\medskip 
Аксиомы вида A2 рассматриваются аналогично\\
Рассмотрим случай, когда $B$ — аксиома A1, то есть $B$ имеет вид $B_{0}\left[P_{1} / C_{1}, \ldots, P_{n} / C_{n}\right]$,где $B(P_1,\dots,P_n)$ — тавтология. Считаем $C_1,\dots,C_n$ замкнутыми формулами сигнатуры $\Sigma(M)$ и докажем $M\vDash B$.\\\medskip 
Допустим $M\nvDash B$. Рассмотрим оценку $f$ пропозициональных переменных $P_1,\dots,P_n$ такую, что\\\medskip 
$f\left(P_{i}\right)=\text{И} \stackrel{\text { def }}{\Longleftrightarrow} M \vDash C_{i}$\\\medskip 
Тогда для любой пропозициональной формулы  $D(P_1,\dots,P_n)$ индукцией по построению $D$ легко доказывается эквивалентность\\
$f(D)=\text{и} \Longleftrightarrow M \vDash D\left[P_{1} / C_{1}, \ldots, P_{n} / C_{n}\right]$\\\medskip 
В частности, для $D=B_0$ получаем $f(B_0)=$Л поскольку $M\nvDash B$.Это противоречит предположению о том, что $B_0$ — тавтология.\\\medskip 
Рассмотрим теперь случай, когда $B$ получена по одному из правил вывода R1–R3.\\\medskip 
Если $A$ получена из $A$ и $A\rightarrow B$ по правилу modus ponens, мы имеем по предположению индукции $M \vDash A_{\mathbf{H}} M \vDash A \rightarrow B$ (считая $A$ и $B$ замкнутыми формулами сигнатуры $\Sigma(M)$). Тогда $M \vDash B$ в силу определения истинности для импликации.\\ \medskip 
Допустим $B=(A \rightarrow \forall x C[a / x])$ получена из $A \rightarrow C$ по правилу R2. Считаем $B$ замкнутой формулой в сигнатуре $\Sigma(M)$. По предположению индукции $M\vDash A\rightarrow C[a/\underline{c}]$ для всех $c\in M$.Если $M\nvDash A$, то очевидно $M \vDash A \rightarrow \forall x C[a / x]$.
Иначе $M\vDash C[a/\underline{c}]$ для всех $c\in M$ и тем самым $M \vDash \forall x C[a / x]$.\\\medskip 
Правило R3 рассматривается аналогично
\end{proof}
\begin{corollary}
    Если $\vdash A$ то $A$ общезначима
\end{corollary}
\begin{corollary}
    Если теория $T$ имеет модель, то $T$ непротиворечива.
\end{corollary}
\begin{corollary}
    Если существует модель $M$ теории $T$ для которой $M\nvDash A$,то $M\nvdash A$.
\end{corollary}
%\section{Теорема Гёделя о полноте исчисления предикатов (без доказательства), её
три эквивалентные формулировки (с доказательством эквивалентности).\\
Теорема Гёделя–Мальцева о компактности для логики предикатов.}
\begin{theorem}[Гёделя о полноте, без доказательства]
    Приведем три различные формулировки и докажем их эквивалентность\\
    A -- замкнутая формула, М -- модель, T -- теория
	\begin{enumerate}
		\item Всякая непротиворечивая теория $T$ выполнима, то есть имеет модель $M\models T$
		\item Если $ T \nvdash A$, то найдется модель $M\models T$, для которой $M \nvDash A$.
		\item $T\models A$ влечет $T \vdash A$.
	\end{enumerate}
\end{theorem}
\begin{proof}[Доказательство равносильности формулировок]
    $\boldsymbol{(1)}\Rightarrow \boldsymbol{(2)}$
    
	Если $T\nvdash A$ то по теореме о дедукции ${T} \cup \{\neg A\}$ непротиворечива.
    Следовательно, \\$T \cup \{\neg A\}$ имеет модель $M$
    
    $\boldsymbol{(2)}\Rightarrow \boldsymbol{(3)}$

	Пусть $T\vDash A$, если бы при этом $T \nvdash A$, то по 2-му условию существовала
	бы модель $T$,такая что $A$ в ней ложна. Но $A$ истинна в $T$,а значит и в $M$-противоречие.

    $\boldsymbol{(3)}\Rightarrow \boldsymbol{(1)}$

	Возьмем $A = B \wedge \neg B$. Тогда $T\nvdash A$, следовательно $T \nvDash A$ и у теории $T$ должна быть
	модель, опровергающая $A$
\end{proof}
\begin{theorem}[Мальцева о компактности]
	1.Теория $T$ выполнима $ \Longleftrightarrow$ любое конечное подмножество
	$T_0\subset T$ выполнимо\\
	2. $T\models A$   $ \Longleftrightarrow$ существует такое конечное множество
	$T_0\subset T$, что $T_0\models A$  
\end{theorem}
\begin{proof}
Теорема о компактности вытекает из теоремы о полноте и свойства компактности отношения выводимости.\\
Предположим противное, т.е. что для любого конечного $T_0\subseteq T$ выполняется
$T_0\nvDash A$. Тогда для каждого конечного существует модель, для которой предложение
$\neg A$  истинно. По теореме компактности существует модель, в которой все предложения из
$T$ и предложение $\neg A$ истинны. Но тогда вопреки условию $T\nvDash A$.
\end{proof}

%\section{Нестандартные модели арифметики, их существование. Понятие галактики. Описание отношения порядка на элементах данной галактики. Плотность порядка на множестве галактик.}
\begin{example}
    Пусть $(\mathbb{N};=;S;+;\times;0)$ — стандартная модель арифметики и $ Th(\mathbb{N}) $ есть множество \textbf{всех} истинных в $\mathbb{N}$ предложений.\\
    Добавим к сигнатуре новую константу $c$ и рассмотрим теорию
    \begin{center}
    $T \rightleftharpoons Th(\mathbb{N}) \cup \{\neg c=0,\neg c=S0,\neg c=SS0,\ldots \}$
    \end{center}
    Терм $\overline{n} \rightleftharpoons SS...S0$ (n раз) называем нумералом. Нумералы служат именами натуральных чисел. 
\end{example}
\begin{proposition}
Каждая конечная подтеория $\boldsymbol{T_0}\subseteq \boldsymbol{T}$ выполнима
\end{proposition}
\begin{proof}
$\boldsymbol{T_0}$ содержит лишь конечное число аксиом вида
$c \neq \overline{n_1} ,\ldots, c \neq \overline{n_k}$. Интерпретируем константу $c$ стандартной модели как любое число $m > n_1,\ldots,n_k$
\end{proof}
По теореме о компактности существует (нормальная) модель. Модель $\boldsymbol{T}\models \boldsymbol{A}$ обладает следующими свойствами:\\
1) $\mathbb{N}$ изоморфна начальному сегменту $\boldsymbol{N}$
Вложение $\mathbb{N} \rightarrow \boldsymbol{N}$ задается функцией $\phi: n \longrightarrow \overline{M}$\\
2) $\boldsymbol{N} \models Th(\mathbb{N})$\\
3) $M\ncong \mathbb{N}$, в частности есть  $c_M \in \boldsymbol{M}$ -- "бесконечно большое", т.к. оно отлично от всех $n \in \mathbb{N}$
\\\\
Формула $a<b\rightleftharpoons \exists x\:( x \neq 0 \wedge  a+x=b)$  определяет порядок в $\mathbb{N}$. Для данной
формулы в $\mathbb{N}$ выполнены аксиомы строгого линейного порядка и следующие
предложения:\\
1)$\forall \: x \:(0<x \lor x=0) $\\
2)$\forall \: x \: \exists \: y \: (x<y\wedge \forall \: z\: (z<y\rightarrow z=x \lor z<x)$\\
3)$\forall\: y\: (y \neq 0 \rightarrow \exists \:x\:(x<y \wedge \forall\: z\:(z<y \rightarrow z=x \lor z<x)$\\
Следовательно, те же аксиомы выполнены и в $\boldsymbol{M}$ поэтому предикат $<_M$ на $\boldsymbol{M}$ представляет собой строгий линейный порядок с наименьшим элементом
0. При этом каждый элемент имеет последователя, и каждый элемент, кроме 0, имеет непосредственного предшественника
\begin{definition}
    Элементы $x,y \in \boldsymbol{M}$ \textbf{близки}, если для некоторого $n \in \mathbb{N}$ выполнено: $y=SS...S(x)$ или $x=SS...S(y)$ (n раз символ S)\\
    Легко проверить, что это отношение эквивалентности\\
    Классы эквивалентности по отношению близости называем галактиками.\\
    Будем говорить, что элементы x и y лежат «в одной галактике», если между ними конечное число элементов (они близки).
\end{definition}
\begin{proposition}
    Если $\boldsymbol{G}$ -- галактика в $\boldsymbol{M}$, $\boldsymbol{G} \neq \mathbb{N}$, то порядок $(\boldsymbol{G},<_M)$ изоморфен $(\mathbb{Z},<)$ 
\end{proposition}
    Пусть $\mathcal{G}$ есть множество всех галактик в $\boldsymbol{M}$. Определим $G_{1}<_{M} G_{2}$, если
для любых $x \in G_{1}, \ y \in G_{2}\: x<_{M} y$
\begin{theorem}
    Порядок  $\left(\mathcal{G},<_{M}\right)$ есть плотный порядок без наибольшего элемента, наименьшим элементом которого является $\mathbb{N}$
\end{theorem}
\begin{proof}
    Если $\boldsymbol{G_1}<\boldsymbol{G_2}$, возьмем четные $x_1 \in \boldsymbol{G_1},x_2 \in \boldsymbol{G2}$ и рассмотрим $y=(x1+x2)/2$\\
    Если $y \in \boldsymbol{G_1}$, то $(x1+x1)/2=x+\overline{n}, \ n\in \mathbb{N}$, тогда $2x_1+2\overline{n}=x_1+x_2$, откуда $x_1+\overline{n}=x_2$, то есть $x_2\in \boldsymbol{G_2}$ -- противоречие\\
    Аналогично, $y \notin \boldsymbol{G_2}$
\end{proof}
%\section{Элементарная теория данной модели. Подмодель, элементарная подмодель данной модели. Теорема
Лёвенгейма–Сколема (для счётной сигнатуры). Cуществование счётных моделей теории множеств ZFC, элементарных теорий
полей $\mathbb{R}$ и $\mathbb{C}$ и элементарной геометрии.}
Пусть $St_\varSigma$ -- множество предложений сигнатуры $\varSigma$
\begin{definition}
	Элементарная теория модели $M$ есть множество $T h(M) \rightleftharpoons \left\{\,A \in \mathrm{St}_{\Sigma}\mid M
	\vDash A\,\right\}$
\end{definition}
\begin{definition}
	Модели $M$ и $N$ сигнатуры $\varSigma$ элементарно эквивалентны ($M\equiv N$), если в $M$ и в $N$ истинны одни
	и те же предложения из $\varSigma$, т.е. $Th(M) \equiv Th(N)$.
\end{definition}
\begin{proposition}
	Если $M \cong N$, то $M \equiv N$
\end{proposition}
\begin{proof} \textcolor{mygray}{
		\textbf{Sketch of the proof} \\(сереньким, потому что не было на лекциях)\\
		Естественно доказывать это утверждение по индукции. Для
		этого его надо обобщить на произвольные формулы (не только замкнутые). Вот это обобщение: пусть $\alpha: M
		\rightarrow N$ — изоморфизм, а $F$ — произвольная формула нашей сигнатуры. Тогда она истинна в $M$ при
		оценке $\pi$ тогда и только тогда, когда она истинна в $N$ при $\alpha \circ \pi$\\
		$M  \models F\left(a_{1}, \ldots, a_{n}\right) \Leftrightarrow N  \models\left(\alpha\left(a_{1}\right),
		\ldots, \alpha\left(a_{n}\right)\right)$ для любых элементов $a_{1}, \dots, a_{n}$ множества $M$
		После такого обобщения доказательство по индукции становится
		очевидным
	}
\end{proof}
\begin{definition}
	$(N ; \Sigma)$  есть подмодель модели $(M ; \Sigma)$ если $N \subseteq M$ и для всех $P \in
	\operatorname{Pred}_{\Sigma}, f \in \mathrm{Func}_{\Sigma}, c \in \text { Const }_{\Sigma}$ $P_{N}=P_{M}\:
	\lceil N, c_{M} \in N, N$ замкнуто относительно $f_{M}$ и $f_{N}=f_{M}\lceil N$
\end{definition}
\begin{example}
	Если $\left(G ;=, \cdot, 1,(\cdot)^{-1}\right)$ — группа, то  подмодели $G$ суть подгруппы группы $G$. Если же
	$G$  рассматривается как модель $(G ;=, \cdot, 1)$, то её подмоделями будут подполугруппы с единицей группы
	$G$.
\end{example}
\begin{definition}
	Подмодель $(N ;\Sigma)$ модели $(M ;\Sigma)$ элементарна, (обозначение $N \preccurlyeq M$) если для всех $A \in
	\mathrm{Fm}_{\Sigma}$ $\forall \vec{x} \in N(N \vDash A[\vec{x}] \Longleftrightarrow M \vDash A[\vec{x}])$
\end{definition}
\begin{proposition}
	$N \preccurlyeq M$ влечет $N \equiv M$.
\end{proposition}
\begin{example}
	Если M — модель $Th(\mathbb{N})$,то $\mathbb{N}$ изоморфна некоторой элементарной подмодели  $N \preccurlyeq M$.
\end{example}
\begin{proof}
	Вложение $\varphi: \mathbb{N} \rightarrow M$ действует по формуле $\varphi(n) \rightleftharpoons(\bar{n})_{M}$.
	Докажем, что $N \rightleftharpoons \varphi(\mathbb{N})$
	есть подмодель $M$. Ясно, что подмножество $\varphi(\mathbb{N}) \subseteq M$ замкнуто относительно функции $S_M$ и
	$S_{M}(\varphi(n))=\varphi\left(S_{\mathrm{N}}(n)\right)$. Рассмотрим теперь функцию сложения $+$. Надо
	установить, что $\phi(\mathbb{N})$ замкнуто относительно $+_M$ и 
	\\$\varphi(n)+_{M} \varphi(m)=\varphi(n+m)$\\
	Это вытекает из равенства $\bar{n}+\bar{m}=\overline{n+m}$, которое является истинным в $\mathbb{N}$
	и потому входит в $Th(\mathbb{N})$ . Функция умножения рассматривается аналогично.
	Таким образом,$\varphi: \mathbb{N} \rightarrow N$  — изоморфизм.\\
	Элементарность вложения следует из цепочки эквивалентностей, верной для любых $n_{1}, \dots, n_{k} \in
	\mathbb{N}$:\\
	$N \vDash A\left[\varphi\left(n_{1}\right), \ldots, \varphi\left(n_{k}\right)\right] \Longleftrightarrow
	\mathbb{N} \vDash A\left[n_{1}, \ldots, n_{k}\right] \Longleftrightarrow \mathbb{N} \vDash A(\overline{n_{1}},
	\ldots, \overline{n_{k}})\Longleftrightarrow M \vDash A(\overline{n_{1}}, \ldots, \overline{n_{k}})
	\Longleftrightarrow M \vDash A\left[\varphi\left(n_{1}\right), \ldots, \varphi\left(n_{k}\right)\right]$\\
	Третья эквивалентность следует из $M \vDash T h(\mathbb{N})$
\end{proof}
Пусть $\Sigma$ - счётная сигнатура\\
\begin{theorem}
	Всякая модель  $(N ;\Sigma)$ имеет (конечную или) счётную элементарную подмодель.
\end{theorem}
\begin{proof}
	Построим последовательность счётных подмножеств модели $M$\\
	$N_{0} \subseteq N_{1} \subseteq N_{2} \subseteq \ldots$\\
	следующим образом:\\
	1)$N_0$— любое непустое счётное подмножество $M$\\
	2)Для каждой формулы $A[a, \vec{b}]$ и набора $\vec{y} \in N_{k}$,если $M \vDash \exists v A[v, \vec{y}]$,
	выберем $x \in M$
	такой, что $M \vDash A[x, \vec{y}]$. Добавим все такие x к $N_k$ и получим $N_{k+1}$ .\\
	Положим $N \rightleftharpoons \bigcup_{k \geq 0} N_{k}$.По построению множества $N$ получаем следующее
	свойство.
	\begin{lemma}
		Для любой формулы $a$ и всех $\vec{y} \in N$\\
		$M \vDash \exists v A[v, \vec{y}] \Longleftrightarrow \exists x \in N \quad M \vDash A[x, \vec{y}]$
	\end{lemma}
	\begin{lemma}
		$N$ есть подмодель $M$
	\end{lemma}
	\begin{proof}
		Пусть $\vec{x} \in N,\:f\in Func_\Sigma$ Поскольку $M \vDash \exists v \: f(\vec{x})=v$, имеем $y\in N$
		такой, что $M \vDash f(\vec{x})=y$, т.е. $f_M(\vec{x})\in N$
	\end{proof}
	Индукцией по построению $A$ теперь покажем\\
	$\forall \: \vec{y}\in N (N \vDash A[\vec{y}] \Longleftrightarrow M \vDash A[\vec{y}])$\\
	1)Для атомарных формул $A$ следует из того, что $N$ — подмодель $M$\\
	\textcolor{mygray}{Если $P\in Pred_\Sigma$ валентности n и $t1, . . . ,tn$ —
	термы, то $P(t1, . . . ,tn)$ есть формула,называемая атомарной формулой}\\
	2)Для $A=\neg B, B \wedge C, B \vee C$ вытекает из предположения индукции.\\
	3) Допустим $A=\exists v B[a / v]$. Тогда\\
	$M \vDash \exists v B[a / v, \vec{y}] \Longleftrightarrow \exists x \in N \quad M \vDash B[x,
	\vec{y}]\Longleftrightarrow \exists x \in N \quad N \vDash B[x, \vec{y}] \Longleftrightarrow N \vDash \exists v
	B[a / v, \vec{y}]$
\end{proof}
\begin{corollary}
	Всякая непротиворечивая теория в счётной сигнатуре имеет (конечную или) счётную модель.\\
	1)Существуют счётные модели $Th(\mathbb{R})$ и $Th(\mathbb{C})$\\
	2)Существует счётная модель элементарной геометрии\\
	3)Если теория множеств $ZFC$ непротиворечива, то существует\\ и счётная модель $ZFC$.\\
	\textcolor{mygray}{Система аксиом Цермело — Френкеля,см билет}
\end{corollary}
\begin{theorem}\textbf{Теорема Лёвенгейма–Сколема (для счётной сигнатуры)}
	Пусть $(M ;\Sigma)$— бесконечная модель в счётной сигнатуре и $\alpha \leq |M|$
	— бесконечная мощность. Тогда найдётся подмодель $N \preceq M$ такая,что $|N|=\alpha$.
\end{theorem}
\begin{proof}
	Та же конструкция, но начинаем с любого подмножества $N_0\subseteq M$
	мощности $\alpha$ . Поскольку сигнатура счётна, нетрудно показать по индукции, что все множеств $N_K$,так же
	как и их объединение $N$,имеют мощность $\alpha$ 
\end{proof}

%\section{Бесконечная модель счётной сигнатуры имеет элементарное расширение
любой большей мощности. (Теорема Мальцева о повышении мощности)}
\begin{theorem} 
	Для любой бесконечной модели $(M ;\Sigma)$, где $\Sigma$ -- счетная сигнатура с равенством, и мощности $\lambda
	\ge |M|$ найдётся модель $(N ;\Sigma)$ такая, что $M \preccurlyeq N$ и $|N|=\lambda$
\end{theorem}
\begin{proof}
	Возьмем $ X \supseteq M,\:|X|=\lambda$.Рассмотрим сигнатуру $\Sigma_X \Longleftrightarrow \Sigma
	\cup\{\underline{c}: c \in X\}$ и теорию $T:=\mathrm{Th}\left(M;\Sigma_{X}\right) \cup\{\underline{c} \neq
	\underline{d}: c, d \in X, c \neq d\}$. Каждая конечное подмножество теории $T$ совместно. По теореме о
	компактности $T$ имеет нормальную модель $N$. Но функция $\varphi\colon c \mapsto(\underline{c})_{N}$
	инъективна в силу аксиом $T$, следовательно $|N|\ge |M|=\lambda$. Т.к. $N\vDash Th(M;\Sigma_X)$, то
	$\varphi(M)$ есть подмодель $N$ изоморфная $M$ и $\varphi(M)\preccurlyeq N$. Это рассуждение совершенно
	аналогично уже разобранному более детально примеру из билета 27.

	Если $|N|=\lambda$, то теорема доказана. Если $|N|>\lambda$, воспользуемся конструкцией из теоремы
	Лёвенгейма-Сколема и построим модель $N_1$ такую, что $M \subseteq N_{1} \preccurlyeq N$ и
	$\left|N_{1}\right|=\lambda$. Тогда $N_1$- требуемая модель, поскольку для любой формулы $A$
	$$\forall \vec{x} \in M\left(N_{1} \vDash A[\vec{x}] \Longleftrightarrow N \vDash A[\vec{x}]
	\Longleftrightarrow M \vDash A[\vec{x}]\right),$$ то есть $M \preccurlyeq N_1$.
\end{proof}
\begin{corollary}
	Если теория $T$ имеет бесконечную модель, то $T$ имеет модели любой бесконечной мощности (Мощность
	модели-мощность предметной области).\\ 
	Множество $Th(\mathbb{N})$ всех предложений , истинных в стандартной модели арифметики, имеет модели любой
	бесконечной мощности.
\end{corollary}

%\section{Основные понятия теории алгоритмов. Пошаговый характер выполнения алгоритма. Частичная функция,
вычисляемая данным алгоритмом; область определения и область значений вычислимой функции.}
\begin{definition}{
	\textbf{Алгоритм} -- это точное предписание, которое задает вычислительный процесс, начинающийся с
	произвольного исходного данного и направленный на получение полностью определяемого этим исходным данным
	результата. Множество возможных исходных данных всякого алгоритма есть некоторый тип конструктивных объектов.
	Главной особенностью любого типа конструктивных объектов является наличие вычислимой взаимнооднозначной
	нумерации. Исполнение алгоритма представляет собой конечную или бесконечную последовательность шагов, каждый из
	которых завершается за конечное время. Если на некотором шаге выполнение алгоритма на исходном данном
	завершается и получен результат, что алгоритм применим к cовокупности исходных данных, к которым применим
	данный алгоритм, называется \textbf{областью применимости этого алгоритма}. (Определение от Плиско)
  }
\end{definition}
Функция назывется частичной, если она, возможно, не тотальна (определена не везде). Частичная функция $f\colon x\to
y$ называется \textbf{вычислимой}, если существует алгоритм, такой что $\dom(f)$ совпадает с областью
применимости и $\forall \:x\in\dom(f)$ есть результат применения алгортима к x.\\
Множество $A\subseteq X$ называется \textbf{разрешимым}, если вычислима его \textbf{характеристическая} функция
\begin{center}
    $\chi_{A}(n)=\left\{\begin{array}{ll}
    {1,} & {n \in A} \\
    {0,} & {n \notin A}
    \end{array}\right.$
\end{center}
Множество $A\subseteq X$ называется \textbf{полуразрешимым}, если вычислима его полухарактеристическая функция 
    \begin{center}
        $\chi_{A}(n)=\left\{\begin{array}{ll}
        {1,} & {n \in A} \\
        \text{не определено иначе}
        \end{array}\right.$
    \end{center}
Множество $A\subseteq X$ называется \textbf{перечислимым}, если $A=\varnothing$ или $A$ -- множество значений всюду
определенной вычислимой функции $f\colon\mathbb{N}\rightarrow X$\\.
\textcolor{mygray}{пример решения задач на эквивалентность определений
\url{http://lpcs.math.msu.su/~zolin/vmlta/pdf/2019-2020_Logic_Zolin_Seminar_7_screen.pdf}}

%\section{Машина Тьюринга. Вычисление словарных и числовых функций на машинах Тьюринга. Тезис Чёрча- Тьюринга.}
% ОБРАТИ ВНИМАНИЕ НА ТО, ЧТО В САМОМ НАЧАЛЕ ФАЙЛА ПОДКЛЮЧАЮСЯ + ТРОЕ (для
% графики); НИЧЕГО НЕ МЕНЯТЬ В ЭТОМ ГОВНОКОДЕ, рискуешь споткнуться о костыли,
% серьёзно,просто не трогай, я не знаю, как оно едет, но мне всё равно, уже 4
% утра.
\subsection*{Машина Тьюринга.}
\par Неформально опишем машину Тьюринга: пусть есть некоторое устройство, на
котором находится ''бесконечная'' лента, заполненная символами ''$\#$''
(пробелами) почти всюду.
На ленте в любой конечный момент времени записано конечное количество информации.

(тут должна быть таблица с лентой, но она уехала вниз)

\begin{wraptable}{r}{7cm}
	\begin{tabular}{ c|c|c|c|c|c|c|c|c } 
		\hline
		$\#$ & $\#$ & $S_{0}$ & $S_{1}$ & $S_{2}$ & ... & ... & $\#$ & $\#$ \\  \hline
	\end{tabular}
	\\
	\begin{tikzpicture}
		\filldraw[color=white!100, fill=none, very thick](-5,-5) circle (0.4);
		\filldraw[color=black!100, fill=none, very thick](-3.5,-6) circle (0.35);
		\put(-100,-155){\vector(0,1){20}}
		\put(-104,-173){$q$}
	\end{tikzpicture}
\end{wraptable}

\par У устройства есть так называемая ''считывающая головка'',  находящаяся в одном из конечного множества
состояний $Q$ = \{ $q_{0}$, $q_{1}$, $q_{2}$, ...\}, где $q_{0}$ - конечное состояние, $q_{1}$ -- начальное
состояние. Можно понимать $Q$ как оперативную память. Головка считывает символы алфавита $\Sigma$ := \{$S_{0}$,
$S_{1}$,..., $S_{n}$\}, который называется ленточным алфавитом. 
\par Для машины есть ''инструкции'', которые говорят, что машине делать, когда ''она находится в состоянии $q_{i}$
и обозревает символ $S_{j}$''. Например, пусть $q_{i}S_{j} \to q_{i}S_{j}R$. Это значит, что наша машина,
находясь в состоянии $q_{i}S_{j}$, перейдёт в состояние $q_{i}S_{j}$ и считывающая головка сдвинется вправо (за это
отвечает $R$). Таким образом, любую команду мы можем записать пятёркой букв. 
\par Под программой $P$ будем подразумевать конечное множество команд такое, что для каждой пары $q_{i}S_{j}$ есть
ровно одна команда в P. Договариваются, что $q_{0}$ в команде писать необязательно: машина пришла в конечное
состояние и ей уже ничего не надо делать. Иногда считается, что если какая-то пара $q_{i}S_{j}$ не встречается в
инструкциях, то в этом случае машина останавливается. Существует  разновидность машин Тьюринга, в которых
нарушается однозначность соответствия пар состояния и команды. Они называются недетерминированными, но мы их в
курсе рассматривать не будем.
\par Под конфигурацией понимаем мгновенное описание того, что у нас происходит с машиной; будем записывать её как
тройку $XqY$, где $X, Y \subset \Sigma^*, q \in Q$. Головка машины смотрит на первый символ слева в слове $Y$.
\par Будем обозначать машину Тьюринга $M:=(Q, \Sigma, P, q_{0}, q_{1})$, где $Q$ -- конечное множество состояний,
$\Sigma$ -- ленточный алфавит, $P$ -- программа, $q_{0}$ -- конечное состояние, $q_{1}$ -- начальное состояние.
\begin{example}
	\par $\Sigma$ = \{\#, 0, 1 \}, $Q$ = \{$q_{0}$, $q_{1}$, $q_{2}$\}.  Программа:
	\par
	\begin{tabular}{ c c c} 
		$q_{1}\#$ & $\to\ $& $q_{1}\#R$ \\ $q_{1}0$ & $\to\ $& $q_{1}1R$ \\ $q_{1}1$ & $\to\ $& $q_{1}0R$ \\ $q_{2}\#$ & $\to\ $&
		$q_{0}\#L$ \\ $q_{2}0$ & $\to\ $& $q_{2}1R$ \\ $q_{2}1$ & $\to\ $& $q_{2}0R$ \\ 
	\end{tabular}
	\par
	\begin{wraptable}{r}{7cm}
		\begin{tabular}{ c|c|c|c|c|c|c|c|c } 
			\hline
			$\#$ & $\#$ & $0$  & $1$ & $0$ & $0$ & $0$ & $\#$ & $\#$ \\  \hline
		\end{tabular}
		\\
		\begin{tikzpicture}
			\filldraw[color=white!100, fill=none, very thick](-5,-5) circle (0.4);
			\filldraw[color=black!100, fill=none, very thick](-3.5,-6) circle (0.35);
			\put(-100,-155){\vector(0,1){20}}
			\put(-104,-173){$q_{1}$}
		\end{tikzpicture}
	\end{wraptable}

	(а тут лента, которая еще ниже, а не эта)

	\par Что произойдёт при работе такой программы?  Из состояния $q_{1}$ машина перейдёт в состояние $q_{1}$, заменит
	первый символ слова, $0$, на $1$ и головка сдвинется вправо. Дальше из состояния $q_{1}$ машина перейдёт в
	состояние $q_{1}$, заменит второй символ слова, $1$, на $0$ и головка сдвинется вправо. Перейдёт ли машина
	когда-нибудь в состояние $q_{2}$? Нет. Она никогда не завершит работу (то есть не перейдёт в состояние
	$q_{0}$). Как же исправить программу, чтобы этого не происходило? 
	\par Давайте заменим $q_{1}0 \to\ q_{2}1R$, $q_{1}1 \to\ q_{2}0R$. Можно убедиться, что тогда программа будет
	работать, как мы и хотели: сначала получит слово 10111, затем завершит работу, встретив символ $\#$.
\end{example}
\subsection*{Вычисление словарных и числовых функций на машинах Тьюринга.}
\par Возникает естественный вопрос: а какие функции мы можем вычислять с помощью машин Тьюринга? Давайте
разбираться. Для начала, рассмотрим несколько примеров.
\par 
\begin{example}
	\par 1) Сложение можно реализовать, например, так: запишем числа $n-1$, $m-1$ черточками (число $n-1$ запишем $n$
	черточками, потому что мы хотим, чтобы 0 соответствовала одна черточка). Когда головка дойдёт до пробела,
	разделяющего эти числа, иначе говоря ''слова'' из чёрточек, она заменит его на черточку, а последнюю черточку
	из $m$ заменит на $\#$. Тогда количество черт не поменяется, однако же машина сможет прочитать сумму:
	\par $||...| \#|||...| \to ||...||||...|\#$
	\par $n, \#, m \to n,\ 1,\ m-1, \#$
	\par 2) Сравнивать два числа тоже очень просто: будем вычеркивать по одной четрочке из каждого числа. Как только
	одно из чисел будет состоять из одной черточки, то есть мы получили ноль, смотрим на другое: если к этому
	моменту в нём остались две или более чёрточки, то оно больше.
	\par 3) Умножить числа теперь не предствляется трудным: будем складывать столько раз, сколько потребуется. 
\end{example}
\par 
\begin{definition}
	Пусть $f : \Sigma^* \to\ \Sigma^*$ -- частичная функция. $f$ \textbf{вычислима машиной Тьюринга} $M=(Q,
	\Sigma_{M}, P, q_{0}, q_{1})$, если $\Sigma \subset\ \Sigma_{M}$ (то есть алфавит $\Sigma$ является
	подалфавитом $\Sigma_{M}$),\ $\# \notin \Sigma$ ("пробела"\ нет в алфавите $\Sigma$) и $M$ переводит
	конфигурацию $q_{1}X \mapsto q_{0}f(X)$, где $X, f(X) \in \Sigma^*$.
\end{definition}
\par Если $x \notin dom(f)$, то $M$ на $q_{1}X$ не остановится.
\begin{definition} $f$ \textbf{вычислима}, если существует машина Тьюринга, которая её вычисляет.
\end{definition}

\begin{remark}
	\par С таким же успехом мы можем рассматривать $f: \mathbb{N} \times \mathbb{N} \to \mathbb{N}$. 
\end{remark}
\begin{remark}
	\par Композиция $f(g(x))$ вычислимых функций $f$, $g$ вычислима.
\end{remark}
\subsection*{Тезис Чёрча-Тьюринга}
\begin{theorem}[\textbf{Тезис Чёрча-Тьюринга}] Для любой алгоритмически вычислимой функции существует вычисляющая
	её значения машина Тьюринга.
\end{theorem}
Другими словами, любая интуитивно вычислимая функция является вычислимой по Тьюрингу.

%\section{Разрешимые множества. Свойства объединения, пересечения и дополнения разрешимых множеств.}
\begin{definition}
	Множество $A \subset \Sigma^*$ \textbf{разрешимо}, если можно вычислить характеристическую функцию:
	\begin{equation*}
		\chi_{A}(x) = 
		\begin{cases}
			1 &\text{если $x \in A$}\\
			0 &\text{если $x \notin A$}
		\end{cases}
	\end{equation*}
\end{definition}
\begin{example}
	$\varnothing$ (тождественно нулевая функция), $\mathbb{N}$ (тождественная функция), $2\mathbb{N}$
\end{example}

\begin{proposition}
	Семейство разрешимых множеств замкнуто относительно объединения, пересечения и дополнения, то есть если $A, B$
	разрешимы, то $A \cup B$, $A \cap B$, $\mathbb{N} \setminus A$ разрешимы.
\end{proposition}
\begin{proof}
	Если $A$ разрешимо, то можно вычислить характеристическую функцию $\chi_{A}(x)$, тогда
	функция для дополнения $\mathbb{N} \setminus A$ есть инвертированная $\chi_{A}(x)$.
\end{proof}

%\section{Перечислимые множества.Теорема об эквивалентных определениях перечислимого множества.}
\begin{definition}
	Множество $A \subset \Sigma^*$ \textbf{перечислимо}, если $A=rng(f)$, где $f$ вычислима.
\end{definition}
Интуитивно это значит следующее: $f\colon\mathbb{N}\to\Sigma^*$ переводит какие-то натуральные числа в $f(n)$, при
этом мы получаем перечисление $x_0, x_2, x_2,\ldots$ значений функции, которое и будет являться нашим
множеством $A$.

\begin{proposition}
	Из разрешимости следует перечислимость.
\end{proposition}
\begin{proof}
	Действительно, если $A$ разрешимо, то можно вычислить характеристическую функцию $\chi_{A}(x)$, тогда положим
	$\chi^*_{A}(x)$ := $1$, если вычислили  $\chi_{A}(x)$ и получили $1$. В другом случае зациклимся. Построенная
	функция является вычислимой, определена на всех элементах множества $A$ и не определена иначе. Значит, $A$
	перечислимо.
\end{proof}
\begin{theorem}[Эквивалентные определения перечислимого множества]
	Следующие утверждения равносильны:
	\begin{enumerate}
		\item $A$ перечислимо;
		\item $A = \varnothing $ или $A = rng(f)$, где f -- тотально вычислимая функция;
		\item $A = \dom(f)$, где $f$ - вычислимая;
		\item Вычислима функция
			\begin{equation*}
				\chi^*_{A}(x) = 
				\begin{cases}
					\text{1, если $x \in A$}\\
					\text{не определена, иначе}
				\end{cases}
			\end{equation*}
		\item $A = \{\,x\mid\exists y\,\langle x,y\rangle\in B\,\}$, где $B$ - разрешимо.
	\end{enumerate}
\end{theorem}
\begin{proof}
	2) $\Rightarrow$ 1), 4) $\Rightarrow$ 3) очевидны.

	3) $\Rightarrow$ 4): $A=\dom(f)$. Рассмотрим машину Тьюринга $M$, соответствующую $f$.Сделаем новую машину
	Тьюринга $M'$, которая делает следующее: $M'$ работает как $M$, если же $M$ останавлвается, то мы стираем то,
	что было на ленте и пишем там $f$. $M'$ вычислит $\chi^*_{A}(x)$.

	3) $\Rightarrow$ 5): Рассмотрим машину $M_f$, которая вычисляет $f, \dom(f)=A$ и рассмотрим множество $B =
	\{\,\langle x,y\rangle\in A\times\mathbb{N}\mid M_{f}(x)\mbox{ останавливается через}\,\leqslant y\mbox{ шагов
	}\,\}$. $B$ - разрешимо. Действительно, характеристическую функцию $B$ можно вычислить следующим образом:
	выполним $y+1$ шаг $f(x)$. Если при этом машина остановится, то выдаём $1$, иначе $0$. Требуемое множество
	построено.

	5) $\Rightarrow$ 2): Пусть $A = \{\,x\mid \exists y\,\langle x,y\rangle \in B\,\}$, $B$ - разрешимо.
	Допустим, что $A\neq\varnothing$. Выберем $a_0\in A$ и построим функцию
	$\varphi: \mathbb{N} \to \mathbb{N}$ такую, что 
	\begin{equation*}
		\varphi(n) = 
		\begin{cases}
			a_{0}\text{, если }n=\langle x,y\rangle\notin B&\\
			l(n)\text{, если }n=\langle x,y\rangle \in B.&
		\end{cases}
	\end{equation*}

	Она вычислима. $x \in A \iff \exists n (l(n)=x \land n \in B) \Rightarrow$ вычислима.
	
	1) $\Rightarrow$ 4) $A = rng(f)$, дана машина $M_f$. Для данного $n=0,1,2,\ldots$ выполним $l(n)$ шагов
	вычисления функции $f(r(n))$ (теперь проекция на вторую координату), то есть $M_{f}(r(n))$, где $n = \langle l(n),
	r(n)\rangle$. Если какое-то вычисление завершено и $f(r(n))=x$, то возьмём 1 на нём в качестве
	ответа. Схема обхода на картинке.
\end{proof}

%\section[Пересечение и объединение перечислимых множеств. Теорема о графике вычислимой функции. Теорема
Чёрча-поста]{Свойства пересечения и объединения перечислимых\\множеств. Теорема о графике вычислимой функции.
Теорема Чёрча-Поста (критерий разрешимости).}
\begin{quote}{\textbf{Свойства: }}
	Семейство перечислимых множеств замкнуто относительно объединения и пересечения, то есть если $A, B$ перечислимы,
	то $A \cup B$, $A \cap B$, перечислимы.
	\par Для объединения: $A$ перечислимо, тогда $A=rng(f)$, где $f$ -- функция натурального аргумента. Тогда $A = \{
		f(0), f(1), f(2), \dots \}$ и существует программа, которая печатает элементы этих множеств. Запустим обе
	программы для $A, B$ и смотрим, что выдалось. Результаты объединяем.
	\par Для пересечения: Запускаем обе программы и ждём накопления результатов. Если в одной программе появился, то
	появится и во второй.
\end{quote}
\begin{theorem}{(о графике вычислимой функции)}
	\par$f$ вычислима $\iff$ график $G_{f} = \{<x,y> | f(x) = y\}$ перечислим.
	\begin{proof}
		\par $\Longrightarrow$ Если мы можем вычислить $f(x)$, то можем вычислить $\chi^*_{G_f}(x)$.
		\par $\Longleftarrow$ Перечисляем график до тех пор, пока не увидим $<x,y> \in G_{f}$. Если встретим её, то
		положим $f(x) = y$. Если не встретим, то не определена.
	\end{proof}
\end{theorem}
\par \begin{theorem}{(Поста)}
	\par$A, \ \mathbb{N}\setminus A$ разрешимы $\iff A$ и $\mathbb{N}\setminus A$ перечислимы.
	\begin{proof}
		\par$\Longrightarrow$ Очевидно, потому что из разрешимости следует перечислимость.
		\par$\Longleftarrow$ Воспользуемся эквивалентным определением перечислимости: множество $A$ перечислимо, если
		вычислима \begin{equation*}
			\chi^*_{A}(x) = 
			\begin{cases}
				\text{1, если $x \in A$}\\
				\text{не определена, иначе}
			\end{cases}
		\end{equation*}
		Аналогично и для множества $\mathbb{N}\setminus A$. 
		\parВозьмём слово $x$ и запустим две машины: одна будет вычислять $\chi^*_{A}$, другая --
		$\chi^*_{\mathbb{N}\setminus A}$. За конечное время какая-то из машин выдаст результат на любом слове. Пусть мы
		хотим проверить разрешимость множества $A$, тогда если результат на слове $x$ выдала $M_{\chi^*_{A}}$, положим
		$\chi_{A}(x) = 1$. Если же результат выдала машина $M_{\chi^*_{\mathbb{N}\setminus A}}$, то $\chi_{A}(x) = 0$.
		Вот мы и вычислили $\chi_{A}$, доказав разрешимость $A$. Разрешимость $\mathbb{N}\setminus A$ доказывается
		аналогично.
	\end{proof}
\end{theorem}

\begin{corollary}
	Множество теорем исчисления предикатов перечислимо (указание: использовать теорему Гёделя о полноте).
\end{corollary}
\par Факт: если теория $T$ перечислима и полна,то $T$ разрешима.
\par Действительно, из перечислимости $T$ следует перечислимость множества опровержимых и выполнимых формул
(теорем). Любое предложение либо является теоремой, либо её отрицанием. Следовательно, множество теорем и его
дополнение перечеслимо. Тогда по теореме Поста $T$ разрешима.

%\section[Кодирование машин Тьюринга. Построение универсальной машины Тьюринга.]{\sloppy Кодирование машин Тьюринга.
Построение универсальной машины Тьюринга.}
\parПусть $M=(Q, \Sigma, P, q_{0}, q_{1})$ -- машина Тьюринга, где $Q = \{q_{0}, \dots, q_{s}\}$, $\Sigma =
\{a_{0}, \dots, a_{r}\}$ и $P = \{P_{0}, \dots,  P_{rs)}\}$ -- команды. 
\parМы хотим построить такую машину Тьюринга, которая будет эмулировать работу других машин. Для этого нам понадобятся:
\par1) $\Pi = \{\rightarrow, N, L, R, q, a, 1\}$ -- алфавит программ.
\parКак будем записывать символы команд? $q_{i} \rightarrow q\underbrace{11...1}_{i} = q1^i$, $a_{i} \rightarrow
a\underbrace{11...1}_{i} = a1^i$. Тогда слово $x \in \Sigma^*$ перейдёт в $Code(x)$ -- конкатенацию всех букв.
Пример: $Code(a_{2}, a_{0}, a_{1}) = a11aa1$.
\parКак будет выглядеть команда? $ P: q_{i}a_{k} \rightarrow q_{j}a_{l}\nu$, где $\nu = \{L, N, K \}$. $Code(P) =
q1^i a1^k \rightarrow q1^j a1^l \nu$
\parКак быть с программой? $Code(M) := Code(P_{0})Code(P_{1}) \dots Code(P_{rs})$.
\parПо коду $Code(M)$ однозначно распознаётся $M$: если есть код, то мы знаем все встречающиеся символы (записаны в
коде) и все их пары. Разобьём символы на команды -- вот и восстановили машину. Это значит, что отображение $M
\mapsto Code(M)$ инъективно.
\par2) Понять, что значит, что $M$ вычисляет функцию.
\begin{definition}$M$ \textbf{вычисляет}$f : \Delta^* \rightarrow \Delta^*, \Delta \subset \Sigma, \# \notin
	\Delta$, если $\forall x \in \Delta^*$:
	\par1) i$x\notin\dom(f)$, то из конфигурации $q_{1}x$ машина не останавливается.
	\par2) $x \in\dom(f)$, то останавливается и на месте $x$ написано $f(x) = y$, слева и справа от $y$ символы не
	из $\Delta$, а головка внутри или перед $y$.
\end{definition}
Наше вычисление всё-таки оставляет ''мусор'' -- символы других алфавитов, оставшиеся на ленте. Как получить
чистое вычисление? Давайте просто дойдём до разделяющего символа до $Code(M)$ и сотрём всё до этого, затем идём к
разделяющему символу после слова $y$ и стираем всё после него.
\begin{definition}\textbf{Универсальная машина Тьюринга} $U_{\Delta}$ с рабочим алфавитом
	$\Pi\cup\Delta\cup\{\$\} \forall x \in \Delta^*$ вычисляет результат работы машины $M$ на входе $x$, то есть
	функцию $Code(M)\$x \rightarrow M_{\Delta}(x)$.
\end{definition}
Как организована работа такой машины? Мы ходим по $Code(M)$ и изменяем слово $x$, в конечном итоге придём к
конфигурации $XqY$ -- конфигурации последнего состояния. 

\begin{tabular}{ c|c|c|c|c|c|c|c|c } 
	\hline
	$\#$ & $P_{0}$ & $...$  & $P_{rs}$ & $\$$ & $a$ & $...$ & $1$ & $\#$ \\  \hline
\end{tabular}

%\section{Универсальные функции. Построение универсальной вычислимой функции для класса всех одноместных вычислимых функций $\mathbb{N} \rightarrow \mathbb{N}$}

\begin{definition}
Функция $U$ двух натуральных аргументов является универсальной для класса вычислимых функций одного аргумента, если для каждого n функция $U_n: x \mapsto U(n,x)$ («сечение» функции U при фиксированном n) является вычислимой и если все вычислимые функции одного аргумента встречаются среди $U_n$. 
\end{definition}

\begin{theorem}
Существует вычислимая функция двух аргументов, являющаяся универсальной функцией для класса вычислимых функций одного аргумента.
\end{theorem}

\begin{proof}
Запишем все программы, вычисляющие функции одного аргумента, в вычислимую последовательность $p_0, \ p_1$ и т.д. (например, в порядке возрастания их длины). Положим $U(i,x)$ равным результату работы i-ой программы на входе x. Тогда функция U и будет искомой вычислимой универсальной функцией. Сечение $U_i$ будет вычислимой функцией, вычисляемой программой $p_i$. Алгоритм, вычисляющий саму функцию U, есть по существу интерпретатор для используемого языка программирования (он применяет первый аргумент ко второму, если отождествить программу и её номер).
\end{proof}

\begin{theorem}
Не существует вычислимой всюду определённой функции двух аргументов, универсальной для класса всех вычислимых всюду определённых функций одного аргумента.
\end{theorem}

\begin{proof}
Воспользуемся «диагональной конструкцией». Пусть U -- произвольная вычислимая всюду определённая функция двух аргументов. Рассмотрим диагональную функцию $u(n) = U(n,n)$. Очевидно, на аргументе n функция u совпадает с функцией $U_n$, а функция $d(n) = u(n) + 1$ отличается от $U_n$. Таким образом, вычислимая всюду определённая функция d(n) отличается от всех сечений $U_n$, и потому функция U не является универсальной.
\end{proof}
%\section{Пример вычислимой функции, не имеющей всюду определённого вычислимого продолжения. Пример неразрешимого перечислимого множества. Алгоритмическая неразрешимость проблемы остановки машин Тьюринга.}
\begin{theorem}
Существует вычислимая функция d с натуральными аргументами и значениями, от которой никакая вычислимая функция f не может всюду отличаться: для любой вычислимой функции f найдётся такое число n, что f(n) = d(n)(либо оба значения f(n) и d(n) не определены, либо оба определены и равны).
\end{theorem}
\begin{proof}
Такова диагональная функция $d(n) = U(n,n)$ (здесь U — вычислимая функция двух аргументов, универсальная для класса вычислимых функций одного аргумента). Любая вычислимая функция f есть $U_n$ при некотором n и потому $f(n) = U_n(n) = U(n,n) = d(n)$.
\end{proof}
\begin{theorem}[Теорема-пример 1]
Существует вычислимая функция, не имеющая всюду определённого вычислимого продолжения.
\end{theorem}
\begin{proof}
Такова, например, функция $D(n) = d(n) + 1$, где d — функция из предыдущей теоремы. В самом деле, любое её всюду определённое продолжение всюду отличается от d (в тех местах, где функция d определена, функция D на единицу больше d и потому любое продолжение функции D отличается от d; там, где d не определена, любая всюду определённая функция отличается от d).
\end{proof}
\begin{theorem}[Теорема-пример 2]
Существует перечислимое неразрешимое множество.
\end{theorem}
\begin{proof}
Рассмотрим вычислимую функцию f(x), не имеющую всюду определённого вычислимого продолжения. Её область определения F будет искомым множеством. В самом деле, F перечислимо (по одному из определений перечислимости). Если бы F было разрешимо, то функция
\begin{equation*}
g(x) = 
 \begin{cases}
   f(x) &\text{если $x \in F$}\\
   0 &\text{если $x \notin F$}
 \end{cases}
\end{equation*} 
была бы вычислимым всюду определённым продолжением функции f (при вычислении g(x) мы сначала проверяем, лежит ли x в F, если лежит, то вычисляем f(x)).
\end{proof}

Полезно проследить, какое именно множество в итоге оказалось перечислимым и неразрешимым. Легко понять, что это множество тех n, при которых U(n,n) определено. Если вспомнить конструкцию функции U, то это множество тех n, при которых n-я программа останавливается на n. Поэтому говорят, что «проблема самоприменимости» (применимости программы к своему номеру) неразрешима. Заметим, что отсюда следует, что и область определения всей универсальной функции U является перечислимым неразрешимым множеством пар. (Если бы проблема выяснения применимости программы к произвольному аргументу была бы разрешима, то и её частный случай — применимость программы к себе — был бы разрешим.) Эту более общую задачу (узнать, остановится ли данная программа на данном входе) называют иногда «проблемой остановки».
%\section{Главные универсальные функции. Главность вычислимой универсальной функции, построенной по нумерации машин
Тьюринга.}

\begin{definition}
	Вычислимая универсальная функция $F:\mathbb{N}^2 \rightarrow \mathbb{N}$ называется главной, если для любой
	вычислимой $g:\mathbb{N}^2 \rightarrow \mathbb{N}$ найдётся тотальная вычислимая функция $s:\mathbb{N}
	\rightarrow \mathbb{N}$ такая, что $\forall e,x$ $g(e,x) \simeq F(s(e),x)$.
\end{definition}

\begin{theorem}
	Существует главная универсальная функция.
\end{theorem}
\begin{proof}
	Покажем, что описанное в доказательстве теоремы 35.1 (см. билет 35) построение универсальной функции даёт
	главную универсальную функцию. Напомним, что мы перечисляли все программы $p_0, p_1, p_2,...$ какого-то
	естественного языка программирования в порядке возрастания их длин и полагали $F(n,x)$ равным результату
	применения программы $p_n$ к входу x. Пусть теперь есть какая-то другая вычислимая функция g двух аргументов.
	Нам надо по любому натуральному e получить программу функции $g_e$, то есть функции, которая получится, если в
	g зафиксировать первый аргумент равным e. Ясно, что такую программу получить легко — надо в программе для g
	заменить первый аргумент на определение константы (или использовать программу для g в качестве подпрограммы, а
	в основной программе вызывать g с фиксированным первым аргументом).
\end{proof}

\begin{theorem}
	Вычислимая универсальная функция $F:\mathbb{N}^2 \rightarrow \mathbb{N}$, построенная по универсальной машине
	Тьюринга U, является главной.
\end{theorem}
\begin{proof}
	Пусть $\bigtriangleup = \{1\}$ и Машина Тьюринга $M$ вычисляет $g(e,x)$ в унарной записи, то есть
	$M_\bigtriangleup (\overline{\langle e,x \rangle}) \simeq \overline{g(e,x)}$. Сопоставим Машине Тьюринга $M$
	машину $M[n]$, которая для данного входа x вычисляет $\overline{\langle n,x \rangle}$, а далее работает как M.
	Преобразование $n\mapsto\mathrm{Code}(M[n])$ ($\mathrm{Code}()$ - код соответствующей команды) является
	тотальной вычислимой функцией. Имеем $M_\bigtriangleup (\overline{\langle e,x \rangle}) \simeq
	M[e]_\bigtriangleup (\overline{x}) \simeq U_\bigtriangleup (\mathrm{Code}(M[e]) \$ \overline{x}) $. Вспомним,
	что $F(i,n) = |U_\bigtriangleup (\mathrm{word}_{\Pi} (i) \$ \overline{n})|$. Отсюда $g(e,x) \simeq F(s(e),x)$,
	где $s(e) = \mathrm{word}_{\Pi}^{-1} (Code(M[e])) $.

\end{proof}

%\section{Задача распознавания свойств вычислимых функций по их программам. Теорема Райса–Успенского.}

\begin{theorem}
Пусть U — произвольная главная универсальная функция. Тогда множество тех n, при которых функция U является нигде не определённой, неразрешимо.
\end{theorem}
\begin{proof}
Используем метод, называемый «сведением» — покажем, что если бы это множество было разрешимым, то и вообще любое перечислимое множество было бы разрешимым (что, неверно). Пусть K — произвольное перечислимое неразрешимое множество. Рассмотрим такую вычислимую функцию V двух аргументов:
\begin{equation*}
V(n,x) = 
 \begin{cases}
   0 &\text{если $n \in K$}\\
   \text{не определено если $n \notin K$}
 \end{cases}
\end{equation*} 
Как видно, второй аргумент этой функции фиктивен, и она по существу совпадает с полухарактеристической функцией множества K от первого аргумента. Очевидно, эта функция имеет сечения двух типов: при $n \in K$ сечение V появляется нулевой функцией, при $n \notin K$ нигде не определённой функцией. Так как функция U является главной, существует вычислимая всюду определённая функция s, для которой $V(n,x) = U(s(n),x)$ при всех n и x, т. е.$V_n = U_s(n)$. Поэтому при $n \in K$ значение $s(n)$ является U-номером нулевой функции, а при $n \notin K$ значение $s(n)$ является U-номером нигде не определённой функции. Поэтому если бы множество U-номеров нигде не определённой функции разрешалось бы некоторым алгоритмом, то мы бы могли применить этот алгоритм к $s(n)$ и узнать, принадлежит ли число n множеству K или нет. Таким образом, множество K было бы разрешимым в противоречии с нашим предположением.
\end{proof}

\begin{theorem}[Теорема Райса-Успенского]
Пусть $\alpha $ — произвольное нетривиальное свойство вычислимых функций (нетривиальность означает, что есть как функции, ему удовлетворяющие, так и функции, ему не удовлетворяющие). Пусть U — главная универсальная функция. Тогда не существует алгоритма, который по U-номеру вычислимой функции проверял бы, обладает ли она свойством $\alpha$. Другими словами, множество $\{n | U_n \in \alpha \}$ неразрешимо.
\end{theorem}
\begin{proof}
Посмотрим, принадлежит ли нигде не определённая функция (обозначим её $\zeta$) классу $A$, и возьмём произвольную функцию $\xi$ «с другой стороны» (если $\zeta \in A$, то $\xi \notin A$ и наоборот). Далее действуем как раньше, но только вместо нулевой функции возьмём функцию $\xi$: 
\begin{equation*}
V(n,x) = 
 \begin{cases}
   \xi (x) &\text{если $n \in K$}\\
   &\text{не определено если $n \notin K$}
 \end{cases}
\end{equation*} 
Как и раньше, функция V будет вычислимой (для данных n и x мы ожидаем появления n в множестве K, после чего вычисляем $\xi (x)$. При $n \in K$ функция $V_n$ совпадает с $\xi$, при $n \notin K$ -- с $\zeta$. Таким образом, проверяя свойство $V_n \in \alpha$, можно было бы узнать, принадлежит ли число n множеству K или нет (а это невозможно).
\end{proof}
\end{document}
