\section{Свойства пересечения и объеделения перечислимых множеств. Теорема о графике вычислимой функции. Теорема Чёрча-Поста (критерий разрешимости).}
\begin{quote*}{\textbf{Свойства: }}
Семейство перечислимых множеств замкнуто относительно объединения и пересечения, то есть если $A, B$ перечислимы, то $A \cup B$, $A \cap B$, перечислимы.
\par Для объединения: $A$ перечислимо, тогда $A=rng(f)$, где $f$ -- функция натурального аргумента. Тогда $A = \{ f(0), f(1), f(2), \dots \}$ и существует программа, которая печатает элементы этих множеств. Запустим обе программы для $A, B$ и смотрим, что выдалось. Результаты объединяем.
\par Для пересечения: Запускаем обе программы и ждём накопления результатов. Если в одной программе появился, то появится и во второй.
\end{quote*}
\begin{theorem}{(о графике вычислимой функции)}
\par$f$ вычислима $\iff$ график $G_{f} = \{<x,y> | f(x) = y\}$ перечислим.
\begin{proof}
\par $\Longrightarrow$ Если мы можем вычислить $f(x)$, то можем вычислить $\chi^*_{G_f}(x)$.
\par $\Longleftarrow$ Перечисляем график до тех пор, пока не увидим $<x,y> \in G_{f}$. Если встретим её, то положим $f(x) = y$. Если не встретим, то не определена.
\end{proof}
\end{theorem}
\par \begin{theorem}{(Поста)}
\par$A, \ \mathbb{N}\setminus A$ разрешимы $\iff A$ и $\mathbb{N}\setminus A$ перечислимы.
\begin{proof}
\par$\Longrightarrow$ Очевидно, потому что из разрешимости следует перечислимость.
\par$\Longleftarrow$ Воспользуемся эквивалентным определением перечислимости: множество $A$ перечислимо, если вычислима \begin{equation*}
    \chi^*_{A}(x) = 
    \begin{cases}
    \text{1, если $x \in A$}\\
    \text{не определена, иначе}
    \end{cases}
\end{equation*}
Аналогично и для множества $\mathbb{N}\setminus A$. 
\parВозьмём слово $x$ и запустим две машины: одна будет вычислять $\chi^*_{A}$, другая -- $\chi^*_{\mathbb{N}\setminus A}$. За конечное время какая-то из машин выдаст результат на любом слове. Пусть мы хотим проверить разрешимость множества $A$, тогда если результат на слове $x$ выдала $M_{\chi^*_{A}}$, положим $\chi_{A}(x) = 1$. Если же результат выдала машина $M_{\chi^*_{\mathbb{N}\setminus A}}$, то $\chi_{A}(x) = 0$. Вот мы и вычислили $\chi_{A}$, доказав разрешимость $A$. Разрешимость $\mathbb{N}\setminus A$ доказывается аналогично.
\end{proof}
\end{theorem}

\begin{corollary}
Множество теорем исчисления предикатов перечислимо (указание: использовать теорему Гёделя о полноте).
\end{corollary}
\parФакт: если теория $T$ перечислима и полна,то $T$ разрешима.
\parДействительно, из перечислимости $T$ следует перечислимость множества опровержимых и выполнимых формул (теорем). Любое предложение либо является теоремой, либо её отрицанием. Следовательно, множество теорем и его дополнение перечеслимо. Тогда по теореме Поста $T$ разрешима.
