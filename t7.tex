\section{Вывод леммы Цорна из аксиомы выбора.}

\begin{proof}[Доказательство леммы Цорна]
Допустим, что (X, <) удовлетворяет условию леммы Цорна, но не имеет максимального элемента. Назовем строгой верхней
	гранью цепи C $\subset X$ такой элемент $x \in X$ что c < x для всех $c \in C$. Тогда можно утверждать, что для
	всякой цепи C в X множество её строгих верхних граней $\psi(C)$ непусто. (Рассмотрим любую верхнюю грань $x$
	цепи $C$. Поскольку элемент $x$ не максимален, найдётся y > x, он и будет строгой верхней гранью C)

Рассмотрим теперь множество

\begin{center}
S = $\{\psi(C)\mid C$ - цепь в X$\}$
\end{center}

Заметим, что S будет множеством, поскольку S $\subset \mathcal{P} (X)$. Применяя аксиому выбора к множеству S мы
	можем заключить, что существует функция $\varphi$, сопоставляющая любой цепи C некоторую её строгую верхнюю
	грань $\varphi$(C). (Эта функция является композицией функции $\psi$ и функции выбора для S)

Теперь мы построим цепь, которая будет настолько велика, что должна выйти за пределы X (это и будет желаемым
	противоречием). Идея состоит в неограниченном удлинении цепи путём применения функции $\varphi$.

Множество S $\subset$ X называем корректным, если выполняются условия:

1. (S, <) вполне упорядочено (порядок индуцирован с X);

2. $\forall x \in S$ $x = \varphi(S_x)$, где $S_x$ означает $\{y \in S \mid y < x\}$

Заметим, что корректными множествами являются

\begin{center}
$ \varnothing$; $\{ \varphi (\varnothing) \}$; $\{ \varphi (\varnothing), \varphi ( \{ \varphi (\varnothing) \})\}$
	и т. д.
\end{center}

 Докажем следующее вспомогательное утверждение.

\begin{lemma} (i) Если множества S и T корректны, то одно из них
есть начальный отрезок другого.

(ii) Объединение любого семейства корректных множеств корректно.
\end{lemma}

\begin{proof}
(i) Допустим, что ни одно из множеств S и T не является начальным отрезком другого. Общим началом S и T назовём
	такое подмножество J $\subset$ S $\cap$ T,  которое есть начальный отрезок как S, так и T. Заметим, что
	объединение I множества всех общих начал S и T само есть их общее начало. (В самом деле, если x $\in$ I, то для
	некоторого общего начала J имеем $x$ $\in$ J, а тогда $\forall y \in S \ (y < x \Rightarrow y \in J \subset I)$
	и аналогично для T.)

Если I совпадает с одним из множеств S или T, то (i) доказано. В противном случае рассмотрим $s = min_S$
	(S$\setminus$I) и t = $min_T$ (T$\setminus$I), где min берётся по множествам S и T, соответственно. Тогда $S_s$
	= I = $T_t$. В силу корректности S и T получаем s = $\varphi (S_s)$ = $\varphi (T_t)$ = t, то есть I $\cup
	\{s\}$ есть общее начало T и S, расширяющее I, что не возможно.

(ii) Пусть $\Sigma$ — семейство корректных множеств и U = $\cup \Sigma$.

Множество (U, <) линейно упорядочено по утверждению (i). (В самом деле, если x, y $\in$ U, то для некоторых
	корректных множеств S, T $\in \Sigma$ имеем x $\in$ S и y $\in$ T. Возьмём из них большее и воспользуемся его
	линейной упорядоченностью.)

Каждое S $\in \Sigma$ есть начальный отрезок U. Иначе найдётся x $\in$ S и y < x такой, что y $\in$ U $\setminus$ S.
	Тогда для некоторого корректного T $\in \Sigma$ имеем y $\in T \setminus S$,  значит T не является начальным
	отрезком S. По свойству (i) множество S должно быть начальным отрезком T, что противоречит тому, что y < x
	$\in$ S и y $\notin$ S.

Докажем, что (U, <) вполне упорядочено. Пусть Y $\subset$ U непусто. Рассмотрим любой y $\in$ Y и корректное
	множество S $\in \Sigma$ такое, что y $\in$ S. Поскольку Y $\cap$ S непусто и вполне упорядочено (как
	подмножество S), существует x = $min_S (Y \cap S) \in S$. Поскольку S есть начальный отрезок U, x также будет
	наименьшим элементом Y в U.

Осталось проверить, что x = $\varphi (U_x)$ для любого x $\in$ U. Выберем S $\in \Sigma$ такое, что x $\in$ S.
	Заметим, что $U_x = S_x$, поскольку S есть начальный отрезок U. Следовательно, x = $\varphi (S_x) = \varphi
	(U_x)$.
\end{proof}

Рассмотрим теперь множество $\Sigma$ всех корректных подмножеств X и положим U = $\cup \Sigma$.  Поскольку U вполне
упорядочено и, в частности, является цепью, оно имеет строгую верхнюю грань $\varphi$ (U). Тогда U $\cup \{\varphi
(U) \}$ есть собственное расширение U и является корректным множеством, что невозможно по определению $\Sigma$. Лемма
Цорна доказана.
\end{proof}
