\section{Бесконечная модель счётной сигнатуры имеет элементарное расширение
любой большей мощности.(Теорема Мальцева о повышении мощности)}
\begin{theorem} 
	Для любой бесконечной модели $(M ;\Sigma)$, где $\Sigma$ -- счетная сигнатура с равенством, и мощности $\lambda
	\ge |M|$ найдётся модель $(N ;\Sigma)$ такая, что $M \preccurlyeq N$ и $|N|=\lambda$
\end{theorem}
\begin{proof}
	Возьмем $ X \supseteq M,\:|X|=\lambda$.Рассмотрим сигнатуру $\Sigma_X \Longleftrightarrow \Sigma
	\cup\{\underline{c}: c \in X\}$ и теорию \\ $T:=Th\left(M ; \Sigma_{X}\right) \cup\{\underline{c} \neq
	\underline{d}: c, d \in X, c \neq d\}$\\
	Каждая конечное подмножество теории $T$ совместно. По теореме о компактности $T$ имеет нормальную модель $N$.
	Но функция $\varphi\colon c \mapsto(\underline{c})_{N}$ инъективна в силу аксиом $T$, следовательно $|N|\ge
	|M|=\lambda$. Т.к. $N\vDash Th(M;\Sigma_X)$, то $\varphi(M)$ есть подмодель $N$ изоморфная $M$ и $\varphi(M)
	\preccurlyeq N$. Это рассуждение совершенно аналогично уже разобранному более детально примеру из билета 27.\\
	Если $|N|=\lambda$, то теорема доказана. Если $|N|>\lambda$, воспользуемся конструкцией из последней теоремы
	билета 27 и построим модель $N_1$ такую, что $M \subseteq N_{1} \preccurlyeq N$ и
	$\left|N_{1}\right|=\lambda$\\
	Тогда $N_1$- требуемая модель, поскольку для любой формулы $A$\\
	$$\forall \vec{x} \in M\left(N_{1} \vDash A[\vec{x}] \Longleftrightarrow N \vDash A[\vec{x}]
	\Longleftrightarrow M \vDash A[\vec{x}]\right),$$ то есть $M \preccurlyeq N_1$.
\end{proof}
\begin{corollary}
	Если теория $T$ имеет бесконечную модель, то $T$ имеет модели любой бесконечной мощности (Мощность
	модели-мощность предметной области).\\ 
	Множество $Th(\mathbb{N})$ всех предложений , истинных в стандартной модели арифметики, имеет модели любой
	бесконечной мощности.
\end{corollary}
