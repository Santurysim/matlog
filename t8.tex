\section{Вывод теоремы Цермело из леммы Цорна. Вывод аксиомы выбора из теоремы Цермело.}

\begin{proof}[Вывод теоремы Цермело из леммы Цорна.]
Вполне упорядоченное множество (S, $<_S$) назовём вполне упорядоченным подмножеством X, если S $\subset$ X. Для данного множества X рассмотрим совокупность W(X) всех его вполне упорядоченных подмножеств. На W(X) определим отношение строгого частичного порядка $\prec$ следующим образом:

\begin{center}
(S, $<_S$) $\prec$ (T, $<_T$), если и только если S $\subset$ T есть собственный начальный отрезок (T, $<_T$), и $<_S$ совпадает с ограничением $<_T$ на S. 
\end{center}

Докажем, что (W(x), $\prec$) удовлетворяет условию леммы Цорна. Рассмотрим любую цепь C $\subset$ W(X).  Цепи C соответствует возрастающая по включению цепь подмножеств X и возрастающая по включению цепь бинарных отношений на этих множествах. Обозначим через U объединение этой цепи подмножеств X, а через $<_U$ $U$ — объединение соответствующей цепи отношений. Ясно, что $<_U$ $U$ есть отношение линейного порядка на U и каждое (S, $<_S) \in$ C есть начальный отрезок (U, $<_U$). Отсюда получаем, что (U, $<_U$) — вполне упорядоченное подмножество X. Таким образом, (U, $<_U$) есть элемент W(X) и верхняя грань цепи C.

Применяя лемму Цорна получаем, что в (W(x), $\prec$) найдётся некоторый максимальный элемент (M, $<_M$). Тогда M обязано совпадать со всем X: в противном случае мы можем взять a $\in$ X $\setminus$ M и продолжить порядок $<_M$ на большее множество N = M $\cup \{ a \}$ полагая x $<_N$ a для всех x $\in$ M.  (Формально, $<_N$ будет объединением $<_M$ и $\{ <x,a> \mid x \in M \}$.) Тогда (N, $<_N$) будет вполне упорядоченным подмножеством X и (M, $<_M$) $\prec$ (N, $<_N$), что противоречит максимальности (M, $<_M$).
\end{proof}

\begin{proof}[Вывод аксиомы выбора из теоремы Цермело.]
Пусть S - данное семейство непустых множеств. По теореме Цермело множество U = $\cup$ S может быть вполне упорядочено. Для каждого  x $\in$ S имеем x $\subset$ U. Пусть min(x) означает наименьший элемент x в смысле порядка на U. Поскольку $\varnothing \notin$ S, соответствие x $\mapsto$ min(x) является функцией выбора на S.
\end{proof}