\section{Равносильность формул логики высказываний, связь с тождественной истинностью. Важнейшие равносильности. Свойство замены подформулы
на равносильную}.
\definition{Формулы A и B называются \textbf{равносильными (эквивалентными)}, обозначение {$\boldsymbol{A}\equiv \boldsymbol{B}$}, если при каждой оценке пропозициональных переменных значение A совпадает со значением B.\\ Другими словами, $\varphi_{A}=\varphi_{B} $} \label{formula}
\begin{example}
  ${P}\to{Q}  \equiv \neg Q\to{\neg P} $
\end{example}
 
\begin{center}
 \textbf{Связь с тождественной истинностью}
\end{center}

{

}
\textit{(i) Отношение $\equiv$ рефлексивно,симметрично и
транзитивно. То есть является отношением эквивалентности.}
{

}
\textit{(ii) Формулы А и B равносильны тогда и только тогда, когда формула ${A}\leftrightarrow{B}$ является тавтологией.}{

}
\textit{(iii)Формула A тавтология тогда и только тогда, когда
${A}\equiv \top $.}

\begin{center}
 \textbf{Основные равносильности}
{

}
$A \wedge B  \equiv B \wedge A $ \\
$A \wedge(B \wedge C)  \equiv(A \wedge B) \wedge C $ \\
$A \wedge A  \equiv A $ \\
$A \wedge(B \vee C)  \equiv(A \wedge B) \vee(A \wedge C)$ \\
$A \vee(A \wedge B)  \equiv A$ \\
$ \neg(A \wedge B)  \equiv \neg A \vee \neg B$ \\
$\neg \neg A  \equiv  A $ \\
{---------------------------------------------}

$A \vee B  \equiv B \vee A $\\
$A \vee(B \vee C)  \equiv(A \vee B) \vee C $ \\
$A \vee A  \equiv A $  \\
$A \vee(B \wedge C)  \equiv(A \vee B) \wedge(A \vee C) $ \\
$A \wedge(A \vee B)  \equiv A $ \\
$\neg(A \vee B)  \equiv \neg A \wedge \neg B $ \\
$A \rightarrow B  \equiv \neg A \vee B $

\end{center}
 






\begin{center}
 \textbf{Замена подформулы на равносильную}
\end{center}
{ Если C и D — формулы, а P — пропозициональная
переменная, то через C[P/D] обозначим результат подстановки формулы D вместо P в формулу C.
Формальное определение даётся с помощью индукции по построению
формулы C.
}
\begin{center}
{

}
$P[P / D] \rightleftharpoons D$ {

}
$Q[P / D] \rightleftharpoons Q,$ если Q— переменная, отличная от P {

}
$(\neg A)[P / D] \rightleftharpoons \neg(A[P / D]) $ {

}
$(A \wedge B)[P / D] \rightleftharpoons(A[P / D] \wedge B[P / D])$ {

}
$(A \vee B)[P / D] \rightleftharpoons(A[P / D] \vee B[P / D])$ {

}
$(A \rightarrow B)[P / D] \rightleftharpoons(A[P / D] \rightarrow B[P / D])${

}

\end{center}

\begin{example}
Пусть $C=\left(P_{1} \rightarrow P_{2}\right) \rightarrow P_{2} $ и $D=P_{3} \rightarrow P_{2}$
{

}
$C\left[P_{2} / D\right]=\left(P_{1} \rightarrow\left(P_{3} \rightarrow P_{2}\right)\right) \rightarrow\left(P_{3} \rightarrow P_{2}\right)  $

\end{example}


\begin{theorem}[О подставновке]
Если A -- тавтология, B -- произвольная формула, а P -- пропозициональная переменная, то A[P/B] -- тавтология.
\end{theorem}
\begin{proof}
Рассмотрим произвольную оценку g. Обозначим через g' оценку, полученную из g присвоением переменной P значения
g(B). Индукцией по построению C можно доказать, что
g(C[P/B])=g'(C)  для любой формулы C. Положим C=A. Так как формула A истинна при оценке g', то формула A[P/B] истинна при оценке g .
\end{proof}

\begin{theorem}
Пусть A,B,C -- формулы, а P -- пропозициональная
переменная. Если ${A}\equiv{B}$ , то $A[P/C] \equiv B[P/C]$.
\end{theorem}
\begin{proof}
Пусть $ {A}\equiv{B} $ . В силу $(ii)$ ,  ${A}\leftrightarrow{B}$  - тавтология. По теореме о подстановке, ${A}\leftrightarrow{B}[P/C]$ - тавтология.
Из определения, что $(A \rightarrow B)[P / D] \rightleftharpoons(A[P / D] \rightarrow B[P / D])$  следует, что $(A \leftrightarrow B)[P / C] $ совпадает с 
$(A[P / C] \leftrightarrow B[P / C])$. В силу, (ii) получаем, что $A[P/C] \equiv B[P/C]$.
\end{proof}


\begin{theorem}[Теорема о замене подформулы на эквивалентную]
 Пусть A,B,C формулы, а P -- пропозициональная переменная. Если ${A}\equiv{B}$, то $ { C[P/A]} \equiv  {C[P/B]} $.

\end{theorem}
\begin{proof}
Теорема доказывается индукцией по построению формулы C(c использованием предыдущей теоремы).
\end{proof}

\begin{example}
Формулы A и B выполнимы, а A[P1/B] нет.  $A=\neg P_{1}$ , $B=P_{2} \vee \neg P_{2}$
\end{example}
