\section{Общезначимость аксиом исчисления предикатов. Теорема о корректности исчисления предикатов.}
\begin{definition}
Теория $T$ \textbf{противоречива}, если существует формула $A$
такая, что $T\vdash A$ и  $T\vdash \neg A$.В противном случае теория $T$ называется непротиворечивой.
\end{definition}
\begin{corollary}[Из теоремы о дедукции получаем следующее следствие].\\ Пусть формула $T$ замкнута. Тогда теория $T \cup \{A\}$ противоречива $\Longleftrightarrow T \vdash \neg A$
\end{corollary}
Следующая теорема называется теоремой о корректности исчисления предикатов.
\begin{theorem}
Если $M\vDash T$ и $T \vdash B(b_1,\dots,b_n$), то $M \vDash B([b_1]/x_1,\dots,[b_n]/x_n)$ \\для любых $x_1\dots x_n \in M$
\end{theorem}
\begin{proof}
Индукция по длине вывода формулы $B$ в $T$. Если $B\in T$, то $M\vDash B$, поскольку $M\vDash T$.\\\medskip 
Рассмотрим случай, когда $B$ -- логическая аксиома вида A3, то есть $B=(A[a / t] \rightarrow \exists x A[a / x])$. Можно считать $B$ (после подстановки констант вместо свободных переменных) замкнутой формулой, а $t$ — замкнутым термом сигнатуры $\Sigma(M)$.\\\medskip 
Допустим $M \vDash A[a / t]$. Пусть $c \rightleftharpoons t_{M}$, тогда $M \vDash A[a / \underline{c}]$, а значит и $M \vDash \exists\:x A[a/x]$ (см. определение истинности формул). Тем самым доказано, что $M \vDash A[a / t] \rightarrow \exists x A[a / x]$\\\medskip 
Аксиомы вида A2 рассматриваются аналогично\\
Рассмотрим случай, когда $B$ — аксиома A1, то есть $B$ имеет вид $B_{0}\left[P_{1} / C_{1}, \ldots, P_{n} / C_{n}\right]$,где $B(P_1,\dots,P_n)$ — тавтология. Считаем $C_1,\dots,C_n$ замкнутыми формулами сигнатуры $\Sigma(M)$ и докажем $M\vDash B$.\\\medskip 
Допустим $M\nvDash B$. Рассмотрим оценку $f$ пропозициональных переменных $P_1,\dots,P_n$ такую, что\\\medskip 
$f\left(P_{i}\right)=\text{И} \stackrel{\text { def }}{\Longleftrightarrow} M \vDash C_{i}$\\\medskip 
Тогда для любой пропозициональной формулы  $D(P_1,\dots,P_n)$ индукцией по построению $D$ легко доказывается эквивалентность\\
$f(D)=\text{и} \Longleftrightarrow M \vDash D\left[P_{1} / C_{1}, \ldots, P_{n} / C_{n}\right]$\\\medskip 
В частности, для $D=B_0$ получаем $f(B_0)=$Л поскольку $M\nvDash B$.Это противоречит предположению о том, что $B_0$ — тавтология.\\\medskip 
Рассмотрим теперь случай, когда $B$ получена по одному из правил вывода R1–R3.\\\medskip 
Если $A$ получена из $A$ и $A\rightarrow B$ по правилу modus ponens, мы имеем по предположению индукции $M \vDash A_{\mathbf{H}} M \vDash A \rightarrow B$ (считая $A$ и $B$ замкнутыми формулами сигнатуры $\Sigma(M)$). Тогда $M \vDash B$ в силу определения истинности для импликации.\\ \medskip 
Допустим $B=(A \rightarrow \forall x C[a / x])$ получена из $A \rightarrow C$ по правилу R2. Считаем $B$ замкнутой формулой в сигнатуре $\Sigma(M)$. По предположению индукции $M\vDash A\rightarrow C[a/\underline{c}]$ для всех $c\in M$.Если $M\nvDash A$, то очевидно $M \vDash A \rightarrow \forall x C[a / x]$.
Иначе $M\vDash C[a/\underline{c}]$ для всех $c\in M$ и тем самым $M \vDash \forall x C[a / x]$.\\\medskip 
Правило R3 рассматривается аналогично
\end{proof}
\begin{corollary}
    Если $\vdash A$ то $A$ общезначима
\end{corollary}
\begin{corollary}
    Если теория $T$ имеет модель, то $T$ непротиворечива.
\end{corollary}
\begin{corollary}
    Если существует модель $M$ теории $T$ для которой $M\nvDash A$,то $M\nvdash A$.
\end{corollary}