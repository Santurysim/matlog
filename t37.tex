\section{Главные универсальные функции. Главность вычислимой универсальной функции, построенной по нумерации машин Тьюринга.}

\begin{definition}
Вычислимая универсальная функция $F:\mathbb{N}^2 \rightarrow \mathbb{N}$ называется главной, если для любой вычислимой $g:\mathbb{N}^2 \rightarrow \mathbb{N}$ найдётся тотальная вычислимая функция $s:\mathbb{N} \rightarrow \mathbb{N}$ такая, что $\forall e,x$ $g(e,x) \simeq F(s(e),x)$.
\end{definition}

\begin{theorem}
Существует главная универсальная функция.
\end{theorem}
\begin{proof}
Покажем, что описанное в доказательстве теоремы 35.1 (см. билет 35) построение универсальной функции даёт главную универсальную функцию. Напомним, что мы перечисляли все программы $p_0, p_1, p_2,...$ какого-то естественного языка программирования в порядке возрастания их длин и полагали $F(n,x)$ равным результату применения программы $p_n$ к входу x. Пусть теперь есть какая-то другая вычислимая функция g двух аргументов. Нам надо по любому натуральному e получить программу функции $g_e$, то есть функции, которая получится, если в g зафиксировать первый аргумент равным e. Ясно, что такую программу получить легко — надо в программе для g заменить первый аргумент на определение константы (или использовать программу для g в качестве подпрограммы, а в основной программе вызывать g с фиксированным первым аргументом).
\end{proof}

\begin{theorem}
Вычислимая универсальная функция $F:\mathbb{N}^2 \rightarrow \mathbb{N}$, построенная по универсальной машине Тьюринга U, является главной.
\end{theorem}
\begin{proof}
Пусть $\bigtriangleup = \{1\}$ и МТ M вычисляет $g(e,x)$ в унарной записи, то есть $M_\bigtriangleup (\overline{\langle e,x \rangle}) \simeq \overline{g(e,x)}$. Сопоставим МТ M машину $M[n]$, которая для данного входа x вычисляет $\overline{\langle n,x \rangle}$, а далее работает как M. Преобразование $n \mapsto Code(M[n])$ ($Code()$ - код соответствующей команды) является тотальной вычислимой функцией.
Имеем $M_\bigtriangleup (\overline{\langle e,x \rangle}) \simeq M[e]_\bigtriangleup (\overline{x}) \simeq U_\bigtriangleup (Code(M[e]) \$ \overline{x}) $. Вспомним, что $F(i,n) \rightleftharpoons \mid U_\bigtriangleup (word_{\Pi} (i) \$ \overline{n}) \mid $. Отсюда $g(e,x) \simeq F(s(e),x)$, где $s(e) = word_{\Pi}^{-1} (Code(M[e])) $.

\end{proof}