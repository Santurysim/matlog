\section[Формулы логики высказываний, понятие подформулы. Истинностное значение формулы при данной оценке
пропозициональных переменных. Таблица истинности формулы. Выполнимые формулы, тавтологии, тождественно ложные
формулы и их взаимосвязь. Алгоритм распознавания выполнимости.]{\sloppy Формулы логики высказываний, понятие
подформулы. Истинностное значение формулы при данной оценке пропозициональных переменных. Таблица истинности
формулы. Выполнимые формулы, тавтологии, тождественно ложные формулы и их взаимосвязь. Алгоритм распознавания
выполнимости.}

\begin{definition}
	Алфавитом будем называть любое непустое множество. Его элементы называются символами (буквами).
\end{definition}

\begin{definition}
	Словом в алфавите $\sigma$ называется конечная последовательность элементов $\sigma$.
\end{definition}

Пусть задан некоторый алфавит Var символов, называемых пропозициональными переменными. Знаки $\neg, \land, \lor,
\to$(и аналогичные знаки) называются пропозициональными связками или булевыми связками.

\begin{definition}
	Формулы логики высказываний являются словами в алфавите, состоящем из пропозициональных переменных (Интуитивно,
	пропозициональные переменные интерпретируются как высказывания), пропозициональных связок и скобок: ( и ).
\end{definition}

\begin{definition}
	Множество формул $F_m$ логики высказываний порождается из множества Var по следующим правилам:

	1. Если P $\in$ Var, то P — формула.

	2. Если A — формула, то $\neg$ A — формула.

	3. Если A и B — формулы, то (A $\land$ B), (A $\lor$ B), (A $\to$ B) — формулы.
\end{definition}

\begin{definition}
	Подформулами формулы A называются все те формулы, которые входят в любое построение A. Подформула формулы A,
	отличная от самой формулы A, называется собственной подформулой формулы A.
\end{definition}

\begin{definition}
	Обозначим $\mathbb{B}$ $\rightleftarrows$ $\{$И, Л$\}$ $\rightleftarrows$ $\{$0, 1$\}$. Функции f:
	$\mathbb{B}^n$ $\to$ $\mathbb{B}$ называются булевыми функциями.

\end{definition}

\begin{definition}
	Оценкой пропозициональных переменных (или просто оценкой) называется произвольная функция $f: Var \to \mathbb{B}$.
\end{definition}

\begin{definition}
	Истинностное значение (или просто значение) формулы при данной оценке f определяется индукцией по построению
	формулы в соответствии со следующей таблицей ($T =$ И (истина), $F =$ Л (ложь)).

	$$
	\begin{array}{cc|c|c|c|c}
		{A} & {B} & {\neg A} & {A \wedge B} & {A \vee B} & {A \rightarrow B} \\
		\hline F & F & T & F & F & T \\
		F & T & T & F & T & T \\
		T & F & F & F & T & F \\
		T & T & F & T & T & T 
	\end{array}
	$$

\end{definition}

\begin{definition}
	Таблицей истинности (или истинностной таблицей) формулы A над переменными $P_1$,...,$P_n$ называется таблица,
	указывающая значения формулы A при всех возможных оценках переменных $P_1$,...,$P_n$.
\end{definition}

\begin{definition}
	Пропозициональная формула (выражение, построенное из пропозициональных букв при помощи логических
	(пропозициональных) связок), истинная хотя бы при одной оценке пропозициональных переменных, называется
	выполнимой. Множество формул Г назывется выполнимым, если существует оценка f, при которой истинны одновременно
	все формулы из Г.
\end{definition}

\begin{definition}
	Пропозициональная формула, истинная при каждой оценке пропозициональных переменных, называется тавтологией
	(тождественно истинной).
\end{definition}

\begin{definition}
	Пропозициональная формула, ложная при каждой
	оценке пропозициональных переменных, называется тождественно ложной.
\end{definition}

\begin{proposition}
	Следующие условия равносильны:

	(i) Формула $A$ тождественно ложна.

	(ii) Формула $A$ не является выполнимой.

	(iii) Формула $\lnot A$ — тавтология.
\end{proposition}

\begin{definition}
	В приложениях часто встречается задача проверки пропозициональной формулы на выполнимость. Наиболее
	прямолинейный алгоритм её решения состоит в построении всей таблицы истинности формулы, то есть перебора $2^n$
	всех возможных оценок.
\end{definition}
