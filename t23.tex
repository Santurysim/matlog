\section{Теорема о дедукции для исчисления предикатов.}
Обозначение: $T,A \vdash B$ вместо $T \cup \{A\} \vdash B$
\begin{theorem}
    Для любой теории $T$ и замкнутой формулы $A$
    \begin{center}
        $T, A \vdash B \Longleftrightarrow T \vdash A \rightarrow B$
    \end{center}
\end{theorem}
\begin{proof}
Индукция по длине вывода из гипотез $T,A \vdash B$\\
Если $B$ является аксиомой или принадлежит $T$, то искомый вывод выглядит так:
    \begin{center}
    $B$\\
    $B \rightarrow(A \rightarrow B)$\\
    $A \rightarrow B$   (MP)
    \end{center}
Если формула $B$ совпадает с $A$, то $A \rightarrow B$ есть тавтология $A \rightarrow A$, то есть аксиома A1\\
Если $A$ получена из некоторых предыдущих формул по правилу вывода
modus ponens, то эти формулы имеют вид $C$ и $C\rightarrow B$ Согласно предположению индукции $T \vdash (A \rightarrow C)$ и.$T \vdash (A \rightarrow (C \rightarrow B))$. Искомый вывод формулы $B$
из множества гипотез  $T \cup \{A\}$ стоит из этих двух выводов и следующих
формул:
\begin{center}
    $\begin{array}{l}
    {(A \rightarrow(C \rightarrow B)) \rightarrow((A \rightarrow C) \rightarrow(A \rightarrow B))} \\
    {(A \rightarrow C) \rightarrow(A \rightarrow B)} \\
    {A \rightarrow B}
    \end{array}$
\end{center}
Допустим $B=(C \rightarrow \forall x D[a / x])$ получена из $C \rightarrow D$ по R2. По предположению индукции
\begin{center}
    $T\vdash A \rightarrow(C\rightarrow D)$
\end{center}
Надо построить вывод
\begin{center}
    $T\vdash A \rightarrow(C\rightarrow \forall\:x\:D[a/x])$
\end{center}
Рассмотрим тавтологию
\begin{center}
    ($P \rightarrow(Q\rightarrow R))\leftrightarrow (P\wedge Q\rightarrow R)$
\end{center}
Подставляя $A$ место $P$, $C$ вместо $Q$ и $D$ вместо $R$ получаем, что формула
\begin{center}
     ($A \rightarrow(C\rightarrow D))\leftrightarrow (A\wedge C\rightarrow D)$
\end{center}
выводима в исчислении предикатов.\\
Таким образом, вывод $A \rightarrow(C\rightarrow D)$ в $T$ можно продолжить:
\begin{center}
    $A \rightarrow(C \rightarrow D)$\\
$(A \rightarrow(C \rightarrow D)) \rightarrow(A \wedge C \rightarrow D) \quad(\mathrm{A} 1)$\\
$(A \wedge C) \rightarrow D$(MP)\\
$(A \wedge C) \rightarrow \forall x D[a / x]$ (R2, $A$ замкнута)\\ $A \rightarrow(C \rightarrow \forall x D[a / x])$ (аналогично)
\end{center}
Правило R3 рассматривается аналогично.
\end{proof}