\section{Язык логики предикатов первого порядка данной сигнатуры. Свободные
и связанные переменные, термы, формулы. Замкнутые формулы. Подстановка терма вместо переменной.}

\definition{
Язык логики первого порядка $L_{\Sigma }$ определяется его сигнатурой $\varSigma $. Помимо всех
символов сигнатуры, в алфавит языка $L_{\Sigma}$ входят два фиксированных счётных
алфавита свободных и связанных переменных

 \begin{center}
$ FrVar = \{a_{0}, a_{1}, a_{2}, \ldots \},$ \\
$ BdVar = \{v_{0}, v_{1},v_{2}, \ldots \},  \label{formula4}  $

\end{center}
И следующие специальные символы:\\
\textit{Булевы связки}: $\rightarrow$, $\neg$, $\wedge$ ,$ \vee $;\\
\textit{Кванторы}: $\forall$ (квантор общности, «для всех»), $\exists$ (квантор существования, «существует»);\\
\textit{Знаки пунктуации}: «(», «)» (скобки) и «,» (запятая).\\
Произвольное слово в описанном алфавите называем выражением. Некоторые выражения называются \textit{термами} и \textit{формулами}.\\
Множества термов и формул языка $L_{\Sigma }$ определяются индуктивно.
}

\definition{
Множество термов $Tm_{\Sigma }$ есть наименьшее множество, замкнутое относительно следующих правил:\\
1. Свободные переменные и константы -- термы.\\
2. Если $f$ -- функциональный символ валентности n и $t_{1}, \ldots ,t_{n}  $ -- термы,
то выражение $f(t_{1}, \ldots , t_{n})$ есть терм.
}


\definition{
Множество формул $Fm_{\Sigma}$ есть наименьшее множество,
замкнутое относительно следующих правил:\\
1. Если P -- предикатный символ валентности n и $t_{1}, \ldots,t_{n}$ -- термы,
то $P(t_{1}, \ldots , t_{n})$ есть формула (называемая \textit{атомарной формулой}).\\
2. Если A, B -- формулы, то формулами являются также выражения $A\rightarrow B$, $\neg A$, $A\wedge B$, $ A\vee B $; \\
3. Если A -- формула, и a -- свободная переменная, то для любой связанной
переменной x, не входящей в A, выражения $(\forall x A[a/x])$ и $(\exists x A[a/x]) $ -- формулы. 
}

Формулы, в которые не входят кванторы, называются \textit{бескванторными}. Формулы и термы, в которые не входят свободные переменные, называются \textit{замкнутыми}. Замкнутые формулы также называются \textit{предложениями}.


