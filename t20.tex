\section{Теория первого порядка, её аксиомы и теоремы. Модель данной теории.Понятие выполнимой теории. Примеры
теорий: теория строгих частичных порядков, теория отношения эквивалентности, теория простых графов.}
\begin{definition}
	\textbf{Теорией} сигнатуры $\Sigma$ называем произвольное множество $T$ замкнутых формул языка
	$\mathcal{L}_{\Sigma}$. Элементы $A \in T$ называем нелогическими аксиомами 
\end{definition}
\begin{definition}
	Модель $(M,\Sigma)$ есть \textbf{модель теории  $T$ (теория $T$ выполнима в модели M) $T$} (обозначение
	$M\vDash T$), если для любой $A \in T \: M\vDash A$\\ \label{formula8}
    Теория называется выполнимой (или совместной), если она имеет хотя бы одну модель.
\end{definition}
\begin{example}
Теория отношения эквивалентности в сигнатуре с единственным бинарным предикатным символом $R$ задаётся следующими
	тремя нелогическими аксиомами:
	\begin{enumerate}
		\item $\forall \: x\: R(x,x)$
		\item $\forall \:x,y\: (R(x,y)\rightarrow R(y,x))$
		\item $\forall \:x,y,z (R(x,y)\wedge R(y,z)\rightarrow R(x,z))$
	\end{enumerate}
$R$ есть отношение эквивалентности на множестве $M$, если и
только если $(M,R)\vDash T$ где $T$ — теория отношения эквивалентности.
\end{example}
\begin{example}
    Модель $(M,<)$ есть строгий частичный порядок, если в $(M,<)$ истинны следующие предложения:
	\begin{enumerate}
		\item $\forall \:x,y,z\: (x<y\wedge y<z \rightarrow x<z)$
		\item $\forall \: \neg x<x$
	\end{enumerate}
    Это можно считать определением строгих частичных порядков. Аксиомы 1 и 2 задают теорию строгих частичных порядков.
\end{example}
\begin{example}
    Простой граф — это модель вида $(V,E)$ где $V$ — множество
	(называемое множеством вершин графа), а $E$ — бинарный предикат смежности, причём отношение $E$ симметрично и
	иррефлексивно:
	\begin{enumerate}
		\item $\forall \: x \neg E(x,x)$
		\item $\forall \: x,y\: (E(x,y)\rightarrow E(y,x))$
	\end{enumerate}
    Аксиомы 1 и 2 задают теорию простых графов.
\end{example}
