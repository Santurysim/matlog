\section{Гомоморфизмы и изоморфизмы моделей. Теорема о сохранении истинностного значения формулы при изоморфизме.
Автоморфизмы моделей, метод доказательства невыразимости с помощью автоморфизмов. Описание автоморфизмов моделей
$(\mathbb Z;\leqslant)$, $(\mathbb R^2;=,B)$ и $(\mathbb R^2;=,B,\cong)$ и примеры невыразимых предикатов в этих
моделях}
\begin{definition}
	Пусть $M$ и $M'$ -- две модели сигнатуры $\Sigma$. \textbf{Гомоморфизмом} моделей $M$ и $M'$ называется
	отображение $\varphi\colon M\to M'$, такое, что сохраняются все предикаты, функции и константы $\Sigma$:
	$$
		P_M(x_1,\ldots,x_n) \rightarrow P_{M'}(\varphi(x_1),\ldots,\varphi(x_n)),\quad\forall P\in
		\mathrm{Pred}_\Sigma
	$$
	$$
		\varphi(f_M(x_1,\ldots,x_n)) = f_{M'}(\varphi(x_1),\ldots,\varphi(x_n)),\quad\forall f\in\mathrm{Func}_\Sigma
	$$
	$$
		\varphi(c_M) = c_{M'},\quad\forall c\in\mathrm{Const}_\Sigma
	$$
\end{definition}
\begin{definition}
	\textbf{Изоморфизмом} моделей $M$ и $M'$ называется гомоморфизм $\varphi$, для которого существует обратный
	гомоморфизм $\psi\colon M'\to M$, такой, что $\psi\circ\varphi=\mathrm{id}_M$ и $\varphi\circ\psi =
	\mathrm{id}_{M'}$, где
	$\mathrm{id}_M$ -- тождественный гомоморфизм на $M$.
	Две модели \textbf{изоморфны}, если существует изоморфизм этих моделей.
\end{definition}

\begin{theorem}
Если $\varphi: M \rightarrow M^{\prime}$ -- изоморфизм, то для любой формулы $A\left(a_{1}, \ldots, a_{n}\right)$ и
	любых $c_{1}, \ldots, c_{n} \in M$

$$M \vDash A\left[c_{1}, \ldots, c_{n}\right] \Longleftrightarrow M^{\prime} \vDash
	A\left[\varphi\left(c_{1}\right), \ldots, \varphi\left(c_{n}\right)\right].$$
\end{theorem}

\begin{proof}
Индукция по построению $A$.
\end{proof}

\subsection{Доказательство невыразимости с помощью автоморфизмов.}

\begin{definition}
	\textbf{Автоморфизмом} модели называется изоморфизм модели на себя.
\end{definition}


Поскольку все определимые предикаты и функции сохраняются при автоморфизмах модели, для доказательства
невыразимости достаточно построить автоморфизм, не сохраняющий ту или иную функцию или предикат.

\begin{example}
Автоморфизмы в модели $(\mathbb{Z}, \leqslant)$ есть сдвиги на $n \in \mathbb{Z}$. В этой модели не выразима
	функция $+$. Отображение $\varphi: x \rightarrow x+1$ есть автоморфизм $(\mathbb{Z}, \leqslant)$, не
	сохраняющий $+$.
\end{example}

\begin{example}
Автоморфизмами модели $\left(\mathbb{R}^{2} ;=, B \right)$ являются все взаимно однозначные аффинные преобразования
	плоскости и только они. $\cong$ не выражается в модели, т.к. отношение не сохраняется при растяжении вдоль
	одной из осей.
\end{example}

\begin{example}
 Автоморфизмы модели $\left(\mathbb{R}^{2} ;=, B, \cong\right)$ суть все преобразования плоскости, являющиеся
	композицией гомотетии и движения. Не выражается $=$, т.к. не сохраняется при гомотетии.
\end{example}
% vim: colorcolumn=120 textwidth=115
