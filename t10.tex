\section{Связь между формулами логики высказываний от n переменных и булевыми функциями. Теорема о функциональной
полноте.}

\begin{definition}
	Таблицей истинности (или истинностной таблицей) формулы A над переменными $P_1$,...,$P_n$ называется таблица,
	указывающая значения формулы A при всех возможных оценках переменных $P_1$,...,$P_n$.

	Таким образом, таблица истинности формулы A над n переменными задаёт булеву функцию $\varphi_A : \mathbb{B}^n$
	$\to$ $\mathbb{B}$. Функция $\varphi_A$ определяется равенством
	\begin{center}
		$\varphi_A (\vec{x}) = f_{\vec{x}}$ (A),
	\end{center}
	верным для всех наборов $\vec{x} \in \mathbb{B}^n$
\end{definition}

\begin{theorem}[о функциональной полноте]
	Для любой функции $\varphi$: $\mathbb{B}^n$ $\to$ $\mathbb{B}$ найдётся такая формула A от n переменных, что
	$\varphi = \varphi_A$. При этом можно считать, что A содержит лишь связки $\neg$ и $\lor$.
\end{theorem}

\begin{proof}
	Равенство $\varphi = \varphi_A$ означает, что для всех $\vec{x} \in \mathbb{B}^n$
	\begin{center}
		$\varphi (\vec{x}) = \varphi_A (\vec{x}) = f_{\vec{x}} (A)$
	\end{center}
	Для x $\in \mathbb{B}$ положим
	\begin{equation*}
		P^x= 
		\begin{cases}
			P  &\text{ если $x = $И} \\
			\neg P &\text{если $x = $Л}
		\end{cases}
	\end{equation*}
	Для произвольного $\vec{x} = (x_1,...,x_n) \in \mathbb{B}^n$ обозначим
	\begin{center}
		$A_{\vec{x}}$   $\rightleftarrows  \bigwedge_{i=1}^n P_i^{x_i}$
	\end{center}
	Легко видеть, что формула $A_{\vec{x}}$ истинна лишь при оценке $f_{\vec{x}}$. Другими словами, для любой оценки f
	\begin{center}
		f($A_{\vec{x}}$) = И $\Longleftrightarrow$ f = $f_{\vec{x}}$ (1)
	\end{center}
	Для данной функции $\varphi$ пусть список $\vec{x_1}$,...,$\vec{x_m}$ исчерпывает все наборы $\vec{x} \in
	\mathbb{B}$ для которых $\varphi(\vec{x})$ = И, то есть 
	\begin{center}
		$\varphi(\vec{x}) = $И $\Longleftrightarrow$ $\exists j \ \vec{x} = \vec{x}_j$ (2)
	\end{center}
	Положим теперь
	\begin{center}
		A $\rightleftarrows \bigvee_{j=1}^m A_{\vec{x}_j}$
	\end{center}
	тогда: 
	\begin{center}
		$f_{\vec{x}}$ (A) = И $\Longleftrightarrow \exists j \ f_{\vec{x}} (A_{\vec{x}_j})$\\
		$\Longleftrightarrow \exists j \vec{x} = \vec{x}_j$ по (1)\\
		$\Longleftrightarrow \varphi (\vec{x}) =$И по (2)
	\end{center}
	Заметим теперь, что конъюнкция выражается через дизъюнкцию и
	отрицание, поскольку формула A $\land$ B равносильна $\neg(\neg A \lor \neg B)$.  Поэтому, формулы
	$A_{\vec{x}}$ могут быть переписаны без использования знака $\land$. 
\end{proof}
