\section{Аксиомы и правила вывода исчисления предикатов. Выводимость в теории, простейшие свойства выводимости.
Доказуемые, опровержимые,независимые формулы для данной теории.}
Исчисление предикатов сигнатуры $\Sigma$ задаётся следующими схемами аксиом и правилами вывода.\\
Аксиомы:\\
А1: все тавтологии логики высказываний,\\
А2: $\forall x \ A[a/x] \rightarrow A[a/t]$,\\
A3: $A[a/t] \rightarrow \exists x \ A[a/x]$. \\
Схему аксиом A1 более аккуратно мы понимаем следующим образом. Если $A(P_1,\ldots,P_n)$ -- тавтология, то формула
$A[P_1/C_1,\ldots,P_n/C_n]$ есть аксиома исчисления предикатов для любых формул $C_1,\ldots,C_n$ сигнатуры
$\Sigma$\\
В аксиомах A2 и A3 $A$ -- любая формула сигнатуры $\Sigma$ и $t$ -- любой терм (переменная $x$ не входит в $A$).

\textbf{Правила вывода:}\\
R1. $\cfrac{A \quad A \rightarrow B}{B}$ (modus ponens)\\ \medskip 
R2. $\cfrac{A \rightarrow B}{A \rightarrow \forall x B[a / x]}$\\ \medskip 
R3. $\cfrac{B \rightarrow A}{\exists x B[a / x] \rightarrow A}$

В правилах R2 и R3 переменная a не входит в A (и x не входит в B). Правила R2 и R3 называются правилами Бернайса.
\begin{definition}
	\textbf{Выводом} в исчислении предикатов называется конечная
	последовательность формул, каждая из которых либо является аксиомой, либо получается из предыдущих формул по
	одному из правил вывода R1-R3.\\
	Формула $A$ называется \textbf{выводимой} в исчислении предикатов или теоремой исчисления предикатов
	(обозначение $\vdash A$), если существует вывод, в котором последняя формула есть $A$ \label{formula9}
\end{definition}
Будем считать, что \textbf{гипотезы являются замкнутыми формулами}, что соответствует понятию теории как множеству
замкнутых формул.
\begin{definition}
	\textbf{Выводом} в теории $T$ называется конечная последовательность формул, каждая из которых либо принадлежит
	множеству $T$, либо является логической аксиомой вида A1-A3, либо получается из предыдущих формул по одному из
	правил вывода R1-R3.\\
	Формула $A$ называется \textbf{выводимой (доказуемой)} в теории $T$ или \textbf{теоремой} $T$ (обозначение $T
	\vdash A$), если существует вывод в $T$,в котором последняя формула есть $A$ \label{formula10} \\ 
	Формула $A$ \textbf{опровержима} в $T$, если $T \vdash \neg A$\\
	Формула $A$ \textbf{независима} от $T$, если $T \nvdash A$ и $T \nvdash \neg A$ .
\end{definition}
Простейшие свойства отношения выводимости в теории для исчисления предикатов аналогичны свойствам отношения
выводимости из гипотез для исчисления высказываний.\\
1) Если $T \subseteq U$ и $T \vdash A$, то $U \vdash A$ (монотонность)\\
2) Если $T \vdash A$, то существует такое конечное множество $T_0 \subseteq T$, что $T_0 \vdash A$ (компактность)\\
3) Если $T \vdash A$ для каждой аксиомы $B \in T$ имеет место $U \vdash B$, то $U \vdash A$ (транзитивность).
\begin{definition}
	Пусть $T, \ U$ -- теории сигнатуры $\Sigma$.\\
	Теория $U$ содержит $T$, если для любой $A\in T\: U \vdash A$ (обозначение $U \vdash T$)\\
	Теории $T$ и $U$ (дедуктивно) эквивалентны, если   $U \vdash T$ и $T \vdash U$ (обозначение $U \equiv T$)
\end{definition}
