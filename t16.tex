\section{Предикаты и функции, выразимые в данной модели. Выразимость предиката параллельности прямых в языке
элементарной геометрии и формулировка аксиомы о параллельных.}
\begin{definition}
	Пусть дана сигнатура $\Sigma$ и её интерпретация с носителем модели $M$. Рассмотрим произвольную формулу
	$\varphi$ и набор переменных $x_1,\ldots,x_k$, среди которых содержатся все параметры $\varphi$. Получим k-местный
	предикат на $M$. Говорят, что этот предикат \textbf{выражается} формулой $\varphi$.

	Предикаты, для которых существует выражающая их формула, называются \textbf{выразимыми}. Соответствующие им
	области истинности в $M^k$ также называются выразимыми.
\end{definition}
\begin{definition}
	Функция $f$ выразима, если предикат $$G(x_1,\ldots,x_n,y) \stackrel{def}{\Longleftrightarrow} f(x_1,\ldots,x_n)
	= y$$ выразим.
\end{definition}
\begin{example}
	Выразим в модели элементарной геометрии $(\mathbb{R}^2;=,\cong,B)$ предикат параллельности прямых и аксиому
	параллельности прямых. Для этого введём следующие предикаты:
	\begin{itemize}
		\item $a \ne b \rightleftharpoons \neg a=b$
		\item $c \in ab$ <<$c$ лежит на прямой $ab$>>: $$c\in ab \rightleftharpoons B(c,a,b)\vee B(a,b,c)\vee
			B(a,c,b)$$
		\item $ab\parallel cd$ <<$ab$ параллельна $cd$>>: $$ab\parallel cd\rightleftharpoons (a \ne b\wedge c \ne
			d\wedge \forall x\neg ((x\in ab)\wedge (x\in cd)))$$
		\item аксиома параллельности прямых:
			\begin{quote}
				Через точку $x$ вне прямой $ab$ можно провести прямую, параллельную $ab$, и притом только одну.
			\end{quote}
			$$\forall a,b,x\,(a\ne b \wedge\neg x\in ab \to \forall u,v\,(xu\parallel ab \wedge xv\parallel ab\to
			v\in xu))$$
	\end{itemize}
\end{example}
% vim: textwidth=115 colorcolumn=120
