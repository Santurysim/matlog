\section{Нестандартные модели арифметики, их существование. Понятие галактики. Описание отношения порядка на элементах данной галактики. Плотность порядка на множестве галактик.}
\begin{example}
    Пусть $(\mathbb{N};=;S;+;\times;0)$ — стандартная модель арифметики и $ Th(\mathbb{N}) $ есть множество \textbf{всех} истинных в $\mathbb{N}$ предложений.\\
    Добавим к сигнатуре новую константу $c$ и рассмотрим теорию
    \begin{center}
    $T \rightleftharpoons Th(\mathbb{N}) \cup \{\neg c=0,\neg c=S0,\neg c=SS0,\ldots \}$
    \end{center}
    Терм $\overline{n} \rightleftharpoons SS...S0$ (n раз) называем нумералом. Нумералы служат именами натуральных чисел. 
\end{example}
\begin{proposition}
Каждая конечная подтеория $\boldsymbol{T_0}\subseteq \boldsymbol{T}$ выполнима
\end{proposition}
\begin{proof}
$\boldsymbol{T_0}$ содержит лишь конечное число аксиом вида
$c \neq \overline{n_1} ,\ldots, c \neq \overline{n_k}$. Интерпретируем константу $c$ стандартной модели как любое число $m > n_1,\ldots,n_k$
\end{proof}
По теореме о компактности существует (нормальная) модель. Модель $\boldsymbol{T}\models \boldsymbol{A}$ обладает следующими свойствами:\\
1) $\mathbb{N}$ изоморфна начальному сегменту $\boldsymbol{N}$
Вложение $\mathbb{N} \rightarrow \boldsymbol{N}$ задается функцией $\phi: n \longrightarrow \overline{M}$\\
2) $\boldsymbol{N} \models Th(\mathbb{N})$\\
3) $M\ncong \mathbb{N}$, в частности есть  $c_M \in \boldsymbol{M}$ -- "бесконечно большое", т.к. оно отлично от всех $n \in \mathbb{N}$
\\\\
Формула $a<b\rightleftharpoons \exists x\:( x \neq 0 \wedge  a+x=b)$  определяет порядок в $\mathbb{N}$. Для данной
формулы в $\mathbb{N}$ выполнены аксиомы строгого линейного порядка и следующие
предложения:\\
1)$\forall \: x \:(0<x \lor x=0) $\\
2)$\forall \: x \: \exists \: y \: (x<y\wedge \forall \: z\: (z<y\rightarrow z=x \lor z<x)$\\
3)$\forall\: y\: (y \neq 0 \rightarrow \exists \:x\:(x<y \wedge \forall\: z\:(z<y \rightarrow z=x \lor z<x)$\\
Следовательно, те же аксиомы выполнены и в $\boldsymbol{M}$ поэтому предикат $<_M$ на $\boldsymbol{M}$ представляет собой строгий линейный порядок с наименьшим элементом
0. При этом каждый элемент имеет последователя, и каждый элемент, кроме 0, имеет непосредственного предшественника
\begin{definition}
    Элементы $x,y \in \boldsymbol{M}$ \textbf{близки}, если для некоторого $n \in \mathbb{N}$ выполнено: $y=SS...S(x)$ или $x=SS...S(y)$ (n раз символ S)\\
    Легко проверить, что это отношение эквивалентности\\
    Классы эквивалентности по отношению близости называем галактиками.\\
    Будем говорить, что элементы x и y лежат «в одной галактике», если между ними конечное число элементов (они близки).
\end{definition}
\begin{proposition}
    Если $\boldsymbol{G}$ -- галактика в $\boldsymbol{M}$, $\boldsymbol{G} \neq \mathbb{N}$, то порядок $(\boldsymbol{G},<_M)$ изоморфен $(\mathbb{Z},<)$ 
\end{proposition}
    Пусть $\mathcal{G}$ есть множество всех галактик в $\boldsymbol{M}$. Определим $G_{1}<_{M} G_{2}$, если
для любых $x \in G_{1}, \ y \in G_{2}\: x<_{M} y$
\begin{theorem}
    Порядок  $\left(\mathcal{G},<_{M}\right)$ есть плотный порядок без наибольшего элемента, наименьшим элементом которого является $\mathbb{N}$
\end{theorem}
\begin{proof}
    Если $\boldsymbol{G_1}<\boldsymbol{G_2}$, возьмем четные $x_1 \in \boldsymbol{G_1},x_2 \in \boldsymbol{G2}$ и рассмотрим $y=(x1+x2)/2$\\
    Если $y \in \boldsymbol{G_1}$, то $(x1+x1)/2=x+\overline{n}, \ n\in \mathbb{N}$, тогда $2x_1+2\overline{n}=x_1+x_2$, откуда $x_1+\overline{n}=x_2$, то есть $x_2\in \boldsymbol{G_2}$ -- противоречие\\
    Аналогично, $y \notin \boldsymbol{G_2}$
\end{proof}