\section{Построение множеств целых, рациональных, вещественных чисел в теории множеств.}

\begin{definition}
	Если считать известным определение натурального ряда $\mathbb{N}$, то множество \textbf{целых чисел} $\mathbb{Z}$
	удобно рассматривать как фактормножество. Целое число можно представить разностью двух натуральных чисел $m-n$.
	При этом некоторые пары задают одно и то же число. Поэтому множество целых чисел Z определяется как

	$$
	\mathbb{Z}:=(\mathbb{N} \times \mathbb{N}) /=_{\mathbb{Z}},
	$$

	где отношение эквивалентности $=_{\mathbb{Z}}$ задаётся следующим образом:

	$$
	\left\langle m_{1}, n_{1}\right\rangle=_{\mathbb{Z}}\left\langle m_{2}, n_{2}\right\rangle \stackrel{\text { def
	}}{\Longleftrightarrow} m_{1}+n_{2}=n_{1}+m_{2}.
	$$
\end{definition}

\begin{definition}
	\textbf{Рациональное число} $q = \frac{m}{n}$ можно рассматривать как пару $\left\langle m, n\right\rangle$, где $m
	\in \mathbb{Z}$ и $n \in \mathbb{N} \backslash \{0\}$. Однако некоторые пары задают одно и то же рациональное
	число $q$. Поэтому мы вводим отношение эквивалентности $=_{\mathbb{Q}}$ на $\mathbb{Z} \times(\mathbb{N}
	\backslash\{0\})$ по правилу

	$$
	\left\langle m_1, n_1 \right \rangle=_{\mathbb{Q}} \left \langle m_{2}, n_{2}\right \rangle
	\overset{\underset{\mathrm{def}}{}}{\Longleftrightarrow} m_{1} n_{2}=n_{1} m_{2}.
	$$

	Проверим, что $=_{\mathbb{Q}}$ в самом деле есть отношение эквивалентности. Рефлексивность очевидна, симметричность
	и транзитивность следуют из таковых свойств для целых чисел. Две дроби равны тогда и только тогда, когда
	соответствующие пары эквивалентны. Поэтому рациональные числа можно отождествить с соответствующими классами
	эквивалентности и официальное определение множества рациональных чисел $\mathbb{Q}$ -- это

	$$
	\mathbb{Q}:=\mathbb{Z} \times(\mathbb{N} \backslash\{0\}) /=_{\mathbb{Q}}.
	$$
\end{definition}

Для определения вещественных чисел воспользуемся некоторыми определениями из следующего билета.

\begin{definition}
	Рассмотрим совокупность начальных отрезков в рациональных числах, каждый из которых непустой, собственный (не
	совпадает во всем множеством $\mathbb{Q}$) и не содержит максимума. Каждый такой отрезок назовем
	\textbf{вещественным числом}. На этом множестве отрезков есть естественный порядок (по включению). Такие
	отрезки называются \textbf{дедекиндовыми сечениями}.
\end{definition}
