\section{Рекурсивные определения, определение операций сложения и умножения.}

Функции натурального аргумента часто определяются по индукции (рекурсии). Для того, чтобы определить значение
функции на аргументе $n+1$ предполагается известным значение функции на предыдущем аргументе $n$. Простейшая схема
рекурсивного определения функции $f: \mathbb{N} \rightarrow Y$ сводится к следующей теореме. 

\begin{theorem}
	Пусть $Y$ — множество, $y_{0} \in Y$ и $h: Y \rightarrow Y$ — любая
	функция. Тогда существует единственная функция $f: \mathbb{N} \rightarrow Y$ такая,
	что для всех $n \in \mathbb{N}$

	\begin{equation}
		\left\{\begin{array}{l}
			{f(0)=y_{0}} \\
			{f(n+1)=h(f(n))}
		\end{array}\right.
	\end{equation}

\end{theorem}

\begin{proof}

	Пусть даны $Y, \ y_0$ и $h$ как в условии теоремы. Рассмотрим множество $F$ всех тех функций $f: m \rightarrow Y$,
	где $m \in \mathbb{N}$, для которых выполнены условия (1) для любого $n \in m$. Это множество непусто,
	поскольку содержит пустую функцию, а также функцию, состоящую из
	пары $\left\langle 0, y_{0}\right\rangle$.

	Утверждается, что любые две функции $f, g \in F$ совпадают на пересечении своих областей определения. В противном
	случае рассмотрим минимальный $k \in \mathbb{N}$ такой, что $f(k) \neq g(k)$. Мы имеем $k \neq 0$, поскольку
	$f(0)=y_{0}=g(0)$. Следовательно $k=s+1$, причем $f(s) = g(s)$, поскольку $k$ -- минимальный. Отсюда
	$f(k)=f(s+1)=h(f(s))=h(g(s))=g(s+1)=g(k)$, противоречие.

	Каждая $g: m \rightarrow Y$ есть подмножество $m \times Y \subset \mathbb{N} \times Y$. Рассмотрим множество
	$\bigcup F \subset \mathbb{N} \times Y$. Утверждается, что $f:=\cup F$ есть функция $\mathbb{N} \rightarrow Y$.
	Отношение $\bigcup F$ функционально, поскольку любые два элемента $F$ совпадают на общей области определения.
	Докажем тотальность, рассуждая от противного. Рассмотрим минимальное $m$ такое, что
	$m \notin \dom(f)$. Тогда $\forall g\in F m\notin\dom(g)$, при этом по условию $m\ne 0$. Значит, $\exists l=m-1,
	m=l+1$, и $l\in\dom(g)$. Тогда $g$ можно определить на $m$: $g(m)=h(g(l))\Rightarrow m\in\dom(f)$ --
	противоречие. Свойства (1) очевидно выполняются для $f$, тем самым существование $f$ доказано.
	Единственность $f$, как в рассуждении выше, легко следует по принципу наименьшего числа.
\end{proof}

Применяя эту теорему мы доказываем, например, существование и единственность функции $f(x) = 2^x$ (предполагая
известным определение сложения). Действительно, $f$ рекурсивно определяется равенствами $f(0) = 1$ и $ f(n+1) =
f(n) + f(n)$.

Заметим, что на натуральных числах уже определена функция последователя $S(n) = n+1$. Сложение и умножение можно
определить рекурсией по второму аргументу. Сложение удовлетворяет равенствам

\begin{equation}
	\left\{\begin{array}{l}
		{m + 0 = m} \\
		{m + (n + 1) = (m + n) + 1}
	\end{array}\right.
\end{equation}

Чтобы уложить эту схему в рамки теоремы 3.1 заметим, что функции $f : \mathbb{N} \times X \rightarrow Y$ можно
отождествить с функциями $\mathbb{N} \rightarrow Y^X$, то есть с
последовательностями функций $f_n : X \rightarrow Y$. Таким образом, с помощью
теоремы 3.1 надо построить последовательность функций $f_n : \mathbb{N} \rightarrow \mathbb{N}$ такую, что

\begin{equation}
	\left\{\begin{array}{l}
		{f_0 = id_\mathbb{N}} \\
		{f_{n+1} = s \circ f_n}
	\end{array}\right.
\end{equation}

Тогда $f_0 (m) = m$ и $f_{n+1} (m) = (s \circ f_n) (m) = s(f_n (m)) = f_n (m) + 1$. То есть, если положить $m+n :=
f_n (m)$, то выполняются равенства (2).
Аналогично определяется умножение, как единственная функция $\mathbb{N}^2 \rightarrow \mathbb{N}$ для которой

\begin{equation}
	\left\{\begin{array}{l}
		{m \cdot 0 = 0} \\
		{m \cdot (n+1) = (m \cdot n) + m}.
	\end{array}\right.
\end{equation}
