\section{Задача распознавания свойств вычислимых функций по их программам. Теорема Райса–Успенского.}

\begin{theorem}
Пусть U — произвольная главная универсальная функция. Тогда множество тех n, при которых функция U является нигде не определённой, неразрешимо.
\end{theorem}
\begin{proof}
Используем метод, называемый «сведением» — покажем, что если бы это множество было разрешимым, то и вообще любое перечислимое множество было бы разрешимым (что, неверно). Пусть K — произвольное перечислимое неразрешимое множество. Рассмотрим такую вычислимую функцию V двух аргументов:
\begin{equation*}
V(n,x) = 
 \begin{cases}
   0 &\text{если $n \in K$}\\
   \text{не определено если $n \notin K$}
 \end{cases}
\end{equation*} 
Как видно, второй аргумент этой функции фиктивен, и она по существу совпадает с полухарактеристической функцией множества K от первого аргумента. Очевидно, эта функция имеет сечения двух типов: при $n \in K$ сечение V появляется нулевой функцией, при $n \notin K$ нигде не определённой функцией. Так как функция U является главной, существует вычислимая всюду определённая функция s, для которой $V(n,x) = U(s(n),x)$ при всех n и x, т. е.$V_n = U_s(n)$. Поэтому при $n \in K$ значение $s(n)$ является U-номером нулевой функции, а при $n \notin K$ значение $s(n)$ является U-номером нигде не определённой функции. Поэтому если бы множество U-номеров нигде не определённой функции разрешалось бы некоторым алгоритмом, то мы бы могли применить этот алгоритм к $s(n)$ и узнать, принадлежит ли число n множеству K или нет. Таким образом, множество K было бы разрешимым в противоречии с нашим предположением.
\end{proof}

\begin{theorem}[Теорема Райса-Успенского]
Пусть $\alpha $ — произвольное нетривиальное свойство вычислимых функций (нетривиальность означает, что есть как функции, ему удовлетворяющие, так и функции, ему не удовлетворяющие). Пусть U — главная универсальная функция. Тогда не существует алгоритма, который по U-номеру вычислимой функции проверял бы, обладает ли она свойством $\alpha$. Другими словами, множество $\{n | U_n \in \alpha \}$ неразрешимо.
\end{theorem}
\begin{proof}
Посмотрим, принадлежит ли нигде не определённая функция (обозначим её $\zeta$) классу $A$, и возьмём произвольную функцию $\xi$ «с другой стороны» (если $\zeta \in A$, то $\xi \notin A$ и наоборот). Далее действуем как раньше, но только вместо нулевой функции возьмём функцию $\xi$: 
\begin{equation*}
V(n,x) = 
 \begin{cases}
   \xi (x) &\text{если $n \in K$}\\
   &\text{не определено если $n \notin K$}
 \end{cases}
\end{equation*} 
Как и раньше, функция V будет вычислимой (для данных n и x мы ожидаем появления n в множестве K, после чего вычисляем $\xi (x)$. При $n \in K$ функция $V_n$ совпадает с $\xi$, при $n \notin K$ -- с $\zeta$. Таким образом, проверяя свойство $V_n \in \alpha$, можно было бы узнать, принадлежит ли число n множеству K или нет (а это невозможно).
\end{proof}