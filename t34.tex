\section[Кодирование машин Тьюринга. Построение универсальной машины Тьюринга.]{\sloppy Кодирование машин Тьюринга.
Построение универсальной машины Тьюринга.}
\parПусть $M=(Q, \Sigma, P, q_{0}, q_{1})$ -- машина Тьюринга, где $Q = \{q_{0}, \dots, q_{s}\}$, $\Sigma =
\{a_{0}, \dots, a_{r}\}$ и $P = \{P_{0}, \dots,  P_{rs)}\}$ -- команды. 
\parМы хотим построить такую машину Тьюринга, которая будет эмулировать работу других машин. Для этого нам понадобятся:
\par1) $\Pi = \{\rightarrow, N, L, R, q, a, 1\}$ -- алфавит программ.
\parКак будем записывать символы команд? $q_{i} \rightarrow q\underbrace{11...1}_{i} = q1^i$, $a_{i} \rightarrow
a\underbrace{11...1}_{i} = a1^i$. Тогда слово $x \in \Sigma^*$ перейдёт в $Code(x)$ -- конкатенацию всех букв.
Пример: $Code(a_{2}, a_{0}, a_{1}) = a11aa1$.
\parКак будет выглядеть команда? $ P: q_{i}a_{k} \rightarrow q_{j}a_{l}\nu$, где $\nu = \{L, N, K \}$. $Code(P) =
q1^i a1^k \rightarrow q1^j a1^l \nu$
\parКак быть с программой? $Code(M) := Code(P_{0})Code(P_{1}) \dots Code(P_{rs})$.
\parПо коду $Code(M)$ однозначно распознаётся $M$: если есть код, то мы знаем все встречающиеся символы (записаны в
коде) и все их пары. Разобьём символы на команды -- вот и восстановили машину. Это значит, что отображение $M
\mapsto Code(M)$ инъективно.
\par2) Понять, что значит, что $M$ вычисляет функцию.
\begin{definition}$M$ \textbf{вычисляет}$f : \Delta^* \rightarrow \Delta^*, \Delta \subset \Sigma, \# \notin
	\Delta$, если $\forall x \in \Delta^*$:
	\par1) i$x\notin\dom(f)$, то из конфигурации $q_{1}x$ машина не останавливается.
	\par2) $x \in\dom(f)$, то останавливается и на месте $x$ написано $f(x) = y$, слева и справа от $y$ символы не
	из $\Delta$, а головка внутри или перед $y$.
\end{definition}
Наше вычисление всё-таки оставляет ''мусор'' -- символы других алфавитов, оставшиеся на ленте. Как получить
чистое вычисление? Давайте просто дойдём до разделяющего символа до $Code(M)$ и сотрём всё до этого, затем идём к
разделяющему символу после слова $y$ и стираем всё после него.
\begin{definition}\textbf{Универсальная машина Тьюринга} $U_{\Delta}$ с рабочим алфавитом
	$\Pi\cup\Delta\cup\{\$\} \forall x \in \Delta^*$ вычисляет результат работы машины $M$ на входе $x$, то есть
	функцию $Code(M)\$x \rightarrow M_{\Delta}(x)$.
\end{definition}
Как организована работа такой машины? Мы ходим по $Code(M)$ и изменяем слово $x$, в конечном итоге придём к
конфигурации $XqY$ -- конфигурации последнего состояния. 

\begin{tabular}{ c|c|c|c|c|c|c|c|c } 
	\hline
	$\#$ & $P_{0}$ & $...$  & $P_{rs}$ & $\$$ & $a$ & $...$ & $1$ & $\#$ \\  \hline
\end{tabular}
