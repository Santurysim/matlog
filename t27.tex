\section{Элементарная теория данной модели. Подмодель, элементарная подмодель данной модели. Теорема Лёвенгейма–Сколема (для счётной сигнатуры). Cуществование счётных моделей теории множеств ZFC, элементарных теорий полей $\mathbb{R}$ и $\mathbb{C}$ и элементарной геометрии.}
Пусть $St_\varSigma$ -- множество предложений сигнатуры $\varSigma$
\begin{definition}
Элементарная теория модели $M$ есть множество $T h(M) \rightleftharpoons \left\{A \in \mathrm{St}_{\Sigma}: M \vDash A\right\}$
\end{definition}
\begin{definition}
    Модели $M$ и $N$ сигнатуры $\varSigma$ элементарно эквивалентны ($M\equiv N$), если в $M$ и в $N$ истинны одни и те же предложения из $\varSigma$, т.е. $Th(M) \equiv Th(N)$.
\end{definition}
\begin{proposition}
    Если $M \cong N$, то $M \equiv N$
\end{proposition}
\begin{proof} \textcolor{mygray}{
\textbf{Sketch of the proof} \\(сереньким, потому что не было на лекциях)\\
Естественно доказывать это утверждение по индукции. Для
этого его надо обобщить на произвольные формулы (не только замкнутые). Вот это обобщение: пусть $\alpha: M \rightarrow N$ — изоморфизм, а $F$ — произвольная формула нашей сигнатуры. Тогда она истинна в $M$ при оценке $\pi$ тогда и только тогда, когда она истинна в $N$ при $\alpha \circ \pi$\\
$M  \models F\left(a_{1}, \ldots, a_{n}\right) \Leftrightarrow N  \models\left(\alpha\left(a_{1}\right), \ldots, \alpha\left(a_{n}\right)\right)$ для любых элементов $a_{1}, \dots, a_{n}$ множества $M$
После такого обобщения доказательство по индукции становится
очевидным
}
\end{proof}
\begin{definition}
    $(N ; \Sigma)$  есть подмодель модели $(M ; \Sigma)$ если $N \subseteq M$ и для всех $P \in \operatorname{Pred}_{\Sigma}, f \in \mathrm{Func}_{\Sigma}, c \in \text { Const }_{\Sigma}$ $P_{N}=P_{M}\: \lceil N, c_{M} \in N, N$ замкнуто относительно $f_{M}$ и $f_{N}=f_{M}\lceil N$
\end{definition}
\begin{example}
Если $\left(G ;=, \cdot, 1,(\cdot)^{-1}\right)$ — группа, то  подмодели $G$ суть подгруппы группы $G$. Если же $G$  рассматривается как модель $(G ;=, \cdot, 1)$, то её подмоделями будут подполугруппы с единицей группы $G$.
\end{example}
\begin{definition}
    Подмодель $(N ;\Sigma)$ модели $(M ;\Sigma)$ элементарна, (обозначение $N \preccurlyeq M$) если для всех $A \in \mathrm{Fm}_{\Sigma}$ $\forall \vec{x} \in N(N \vDash A[\vec{x}] \Longleftrightarrow M \vDash A[\vec{x}])$
\end{definition}
\begin{proposition}
    $N \preccurlyeq M$ влечет $N \equiv M$.
\end{proposition}
\begin{example}
Если M — модель $Th(\mathbb{N})$,то $\mathbb{N}$ изоморфна некоторой элементарной подмодели  $N \preccurlyeq M$.
\end{example}
\begin{proof}
    Вложение $\varphi: \mathbb{N} \rightarrow M$ действует по формуле $\varphi(n) \rightleftharpoons(\bar{n})_{M}$.
    Докажем, что $N \rightleftharpoons \varphi(\mathbb{N})$
    есть подмодель $M$
    \\Ясно, что подмножество $\varphi(\mathbb{N}) \subseteq M$ замкнуто относительно функции $S_M$ и $S_{M}(\varphi(n))=\varphi\left(S_{\mathrm{N}}(n)\right)$.Рассмотрим теперь функцию сложения $+$  Надо установить, что $\phi(\mathbb{N})$ замкнуто относительно $+_M$ и 
    \\$\varphi(n)+_{M} \varphi(m)=\varphi(n+m)$\\
    Это вытекает из равенства $\bar{n}+\bar{m}=\overline{n+m}$, которое является истинным в $\mathbb{N}$
    и потому входит в $Th(\mathbb{N})$ . Функция умножения рассматривается аналогично.
    Таким образом,$\varphi: \mathbb{N} \rightarrow N$  — изоморфизм.\\
    Элементарность вложения следует из цепочки эквивалентностей, верной для любых $n_{1}, \dots, n_{k} \in \mathbb{N}$:\\
    $N \vDash A\left[\varphi\left(n_{1}\right), \ldots, \varphi\left(n_{k}\right)\right] \Longleftrightarrow \mathbb{N} \vDash A\left[n_{1}, \ldots, n_{k}\right] \Longleftrightarrow \mathbb{N} \vDash A(\overline{n_{1}}, \ldots, \overline{n_{k}})\Longleftrightarrow M \vDash A(\overline{n_{1}}, \ldots, \overline{n_{k}}) \Longleftrightarrow M \vDash A\left[\varphi\left(n_{1}\right), \ldots, \varphi\left(n_{k}\right)\right]$\\
    Третья эквивалентность следует из $M \vDash T h(\mathbb{N})$
\end{proof}
Пусть $\Sigma$ - счётная сигнатура\\
\begin{theorem}
    Всякая модель  $(N ;\Sigma)$ имеет (конечную или) счётную элементарную подмодель.
\end{theorem}
\begin{proof}
    Построим последовательность счётных подмножеств модели $M$\\
    $N_{0} \subseteq N_{1} \subseteq N_{2} \subseteq \ldots$\\
    следующим образом:\\
    1)$N_0$— любое непустое счётное подмножество $M$\\
    2)Для каждой формулы $A[a, \vec{b}]$ и набора $\vec{y} \in N_{k}$,если $M \vDash \exists v A[v, \vec{y}]$, выберем $x \in M$
    такой, что $M \vDash A[x, \vec{y}]$. Добавим все такие x к $N_k$ и получим $N_{k+1}$ .\\
    Положим $N \rightleftharpoons \bigcup_{k \geq 0} N_{k}$.По построению множества $N$ получаем следующее
    свойство.
    \begin{lemma}
        Для любой формулы $a$ и всех $\vec{y} \in N$\\
        $M \vDash \exists v A[v, \vec{y}] \Longleftrightarrow \exists x \in N \quad M \vDash A[x, \vec{y}]$
    \end{lemma}
    \begin{lemma}
        $N$ есть подмодель $M$
    \end{lemma}
    \begin{proof}
        Пусть $\vec{x} \in N,\:f\in Func_\Sigma$ Поскольку $M \vDash \exists v \: f(\vec{x})=v$, имеем $y\in N$ такой, что $M \vDash f(\vec{x})=y$, т.е. $f_M(\vec{x})\in N$
    \end{proof}
    Индукцией по построению $A$ теперь покажем\\
    $\forall \: \vec{y}\in N (N \vDash A[\vec{y}] \Longleftrightarrow M \vDash A[\vec{y}])$\\
    1)Для атомарных формул $A$ следует из того, что $N$ — подмодель $M$\\
    \textcolor{mygray}{Если $P\in Pred_\Sigma$ валентности n и $t1, . . . ,tn$ —
    термы, то $P(t1, . . . ,tn)$ есть формула,называемая атомарной формулой}\\
    2)Для $A=\neg B, B \wedge C, B \vee C$ вытекает из предположения индукции.\\
    3) Допустим $A=\exists v B[a / v]$. Тогда\\
    $M \vDash \exists v B[a / v, \vec{y}] \Longleftrightarrow \exists x \in N \quad M \vDash B[x, \vec{y}]\Longleftrightarrow \exists x \in N \quad N \vDash B[x, \vec{y}] \Longleftrightarrow N \vDash \exists v B[a / v, \vec{y}]$
\end{proof}
\begin{corollary}
    Всякая непротиворечивая теория в счётной сигнатуре имеет (конечную или) счётную модель.\\
    1)Существуют счётные модели $Th(\mathbb{R})$ и $Th(\mathbb{C})$\\
    2)Существует счётная модель элементарной геометрии\\
    3)Если теория множеств $ZFC$ непротиворечива, то существует\\ и счётная модель $ZFC$.\\ \textcolor{mygray}{Система аксиом Цермело — Френкеля,см билет}
\end{corollary}
\begin{theorem}\textbf{Теорема Лёвенгейма–Сколема (для счётной сигнатуры)}
     Пусть $(M ;\Sigma)$— бесконечная модель в счётной сигнатуре и $\alpha \leq |M|$
    — бесконечная мощность. Тогда найдётся подмодель $N \preceq M$ такая,что $|N|=\alpha$.
\end{theorem}
\begin{proof}
    Та же конструкция, но начинаем с любого подмножества $N_0\subseteq M$
    мощности $\alpha$ . Поскольку сигнатура счётна, нетрудно показать по индукции, что все множеств $N_K$,так же как и их объединение $N$,имеют мощность $\alpha$ 
\end{proof}
