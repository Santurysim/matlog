\section{Равносильность формул языка первого порядка, важнейшие равносильности. Переименование связанных
переменных. Приведение формулы языка первого порядка к предварённой форм}
\begin{definition}
	Формулы $A$ и $B$ сигнатуры $\Sigma$ \textbf{равносильны} (обозначение $A\equiv B$), если для любой модели
	$(M,\Sigma)$ и оценки $f$ на $M$
    $$M \vDash f(A) \Longleftrightarrow M \vDash f(B).$$
\end{definition}
Пусть список $b_1 \dots b_n$ содержит все свободные переменные $A$,$B$.
\begin{proposition}
	$A\equiv B$, если и только если в любой модели $M$ формулы $A$ и $B$ определяют один и тот же предикат, то есть
	если $A_M=B_M$ (для данного набора переменных).
\end{proposition}
\begin{proposition}
1) Отношение $\equiv$ рефлексивно,симметрично и транзитивно.\\
2) $A\equiv B$, если и только если формула $A \leftrightarrow B$ общезначима.\\
3) Формула $A$ общезначима тогда и только тогда, когда $A \equiv$\textup{T}.
\end{proposition}
Перечислим основные равносильности с кванторами
\begin{lemma}[замена связанной переменной]
	Если $x,y \in BdVar$ не входят в формулу $A$, то $\forall \:x A[a / x] \equiv \forall y A[a / y]$ u $\exists x
	A[a / x] \equiv \exists y A[a / y] $
\end{lemma}
\begin{proof}
    Для квантора существования рассуждение аналогично.\\
    $\begin{aligned}
    M \vDash \forall x A[a / x] & \Longleftrightarrow M \vDash A[a / c] \text{для всех} c \in M\\
    & \Longleftrightarrow M \vDash \forall y A[a / y]
    \end{aligned}$
\end{proof}
\begin{lemma}
    Если $X\in BdVar$ не входит в формулы $A,B$, то $(\forall x A[a / x] \vee B) \equiv \forall x(A[a / x] \vee B)$
\end{lemma}
\begin{proof}
	 Прежде всего заметим, что правая часть эквивалентности, так же как и левая часть, является формулой. В самом
	 деле, выберем $a'\in FrVar$, не входящую в $A,B$.Тогда $B[a'/x]=B,A[a/x]=A[a/a'][a'/x]$ и тем самым $\forall x
	 \:(A[a/x]\lor B)$ совпадает с  $\forall x\:(A[a/a']\lor B)[a'/x]$ Получаем
     \begin{center}
         $\begin{aligned}
    M \vDash \forall x(A[a / x] \vee B) & \Longleftrightarrow M \vDash(A[a / c] \vee B) \ \text{для всех} \ c \in M \\
    & \Longleftrightarrow(M \vDash B \ \text{или для всех} \ c \in M \ M \vDash A[a / c]) \\
    & \Longleftrightarrow(M \vDash B \ \text{или} \ M \vDash \forall x \  A[a / x]) \\
    & \Longleftrightarrow M \vDash (\forall x \  A[a / x] \vee B)
    \end{aligned}$
     \end{center}
	 Аналогично обосновываются остальные равносильности, входящие в следующую таблицу (где предполагается, что
	 переменные $x, \ y$ не входят в формулы $A$, $B$).\\
$$
\begin{array}{|ccc|ccc|}
\hline \forall x A[a / x] & {\equiv} & {\forall y A[a / y]} & {\exists x A[a / x]} & {\equiv} & {\exists y A[a / y]} \\
{(\forall x A[a / x] \vee B)} & {\equiv} & {\forall x(A[a / x] \vee B)} & {(\exists x A[a / x] \vee B)} & {\equiv}
	& {\exists x(A[a / x] \vee B)} \\
{(\forall x A[a / x] \wedge B)} & {\equiv} & {\forall x(A[a / x] \wedge B)} & {(\exists x A[a / x] \wedge B)} &
	{\equiv} & {\exists x(A[a / x] \wedge B)} \\
{\neg \forall x A[a / x]} & {\equiv} & {\exists x \neg A[a / x]} & {\neg \exists x A[a / x]} & {\equiv} & {\forall
	x \neg A[a / x]} \\
\hline
\end{array}
$$
\end{proof}
\begin{definition}
	Обогатим язык логики первого порядка пропозициональной переменной $P$. Можно считать $P$ нульместным
	предикатным символом. Распостраним на расширенный язык все синтаксические понятия, включая понятие формулы ($P$
	считается атомарной формулой). Запись $C[P/A]$ означает результат замены всех вхождений $P$ в формулу $C$ на
	$A$.Заметим, что $C[P/A]$ всегда является формулой. Для этого достаточно, чтобы связанные переменные $A$ не
	входили в $C$.
\end{definition}
\begin{lemma}
	$C[P/A]$ -- формула, если и только если любое вхождение P в формулу C не находится в области действия квантора
	по переменной $x\in BdVar$, входящей в A.
\end{lemma}
\begin{proof}
     Необходимость этого условия очевидна. Достаточность доказывается простой индукцией по построению формулы $C$.
\end{proof}
\begin{definition}
    Говорим, что разрешена подстановка формулы $A$ вместо $P$ в $C$, если выполнено условие предыдущей леммы.
\end{definition}
\begin{lemma}
	1) Если $A\equiv B$,то $\neg A \equiv \neg B$ Если $A_1\equiv B_1$ и $A_2\equiv B_2$,то $A_1\wedge A_2\equiv
	B_1 \wedge B_2$,$A_1\lor A_2\equiv B_1 \lor B_2$,$A_1\rightarrow A_2\equiv B_1 \rightarrow B_2$\\
	2)Если $A\equiv B$ и $x \in BdVar$ не входит в $A,B$, то $\forall x A[a / x] \equiv \forall x B[a / x]$ и
	$\exists x A[a / x] \equiv \exists x B[a / x]$
\end{lemma}
\begin{theorem}
	\textbf{(замена подформулы на эквивалентную} Если $A \equiv B$ и разрешена подстановка формул $A,B$ вместо $P$
	в $C$ то $C[P/A]\equiv C[P/B]\\$
    доказательство тут \url{http://lpcs.math.msu.su/vml2019/2019_VML_Beklemishev_2_Logic.pdf} стр.31-32.
\end{theorem}
\begin{lemma}
    Пусть $y\in BdVar$ не входит в формулу $B$. Тогда $B[x/y]$ есть формула и $B[x/y] \equiv B$ \\
    д-во стр.32 \url{http://lpcs.math.msu.su/vml2019/2019_VML_Beklemishev_2_Logic.pdf} 
\end{lemma}
\begin{theorem}
	Пусть формула $A$ общезначима и разрешена подстановка формулы $C$ вместо $P$ в $A$, тогда общезначима формула
	$A[P/C]$.
     д-во стр.32 \url{http://lpcs.math.msu.su/vml2019/2019_VML_Beklemishev_2_Logic.pdf} 
\end{theorem}
\begin{definition}
	Формула $A$ называется \textbf{предварённой}, если $A$ имеет вид $$\mathrm{Q} x_{1} \mathrm{Q} x_{2} \ldots
	\mathrm{Q} x_{n} A_{0}\left[b_{1} / x_{1}, \ldots, b_{n} / x_{n}\right],$$где $Q$ означает квантор $\forall$ или
	$\exists$, а формула $A_0$ бескванторная.
\end{definition}
Далее тут доказательства см. стр.33 \url{http://lpcs.math.msu.su/vml2019/2019_VML_Beklemishev_2_Logic.pdf}
\begin{theorem}
    Для любых $\alpha,\beta$ и любых формул $A,B$, не содержащих переменных из $\alpha,\beta$ имеем:\\
    1)$\alpha A \wedge B \equiv \alpha(A \wedge B) ; \alpha A \vee B \equiv \alpha(A \vee B)$\\
	2)$\neg \alpha A \equiv \bar{\alpha} \neg A$, где $\bar{\alpha}$ получается их $\alpha$ заменой всех символов
	$\exists$ на $\forall$ и наоборот.
\end{theorem}
\begin{theorem}
	Для каждой формулы $A$ можно указать эквивалентную ей предварённую формулу $A'$ от тех же свободных переменных.
	Такую формулу $A'$ называем предварённой формой формулы $A$.
\end{theorem}
