\section{Основные понятия теории алгоритмов. Пошаговый характер выполнения алгоритма. Частичная функция, вычисляемая данным алгоритмом; область определения и область значений вычислимой функции.
}
\begin{definition}{
    \textbf{Алгоритм} -- это точное предписание, которое задает вычислительный процесс, начинающийся с произвольного исходного данного и направленный на получение полностью определяемого этим исходным данным результата. Множество возможных исходных данных всякого алгоритма есть некоторый тип конструктивных объектов. Главной особенностью любого типа конструктивных объектов является наличие вычислимой взаимнооднозначной нумерации. Исполнение алгоритма представляет собой конечную или бесконечную последовательность шагов, каждый из которых завершается за конечное время. Если на некотором шаге выполнение алгоритма на исходном данном завершается и получен результат, что алгоритм применим к cовокупности исходных данных, к которым применим данный алгоритм, называется \textbf{областью применимости этого алгоритма}. (Определение от Плиско)
  }
\end{definition}
Функция назывется частичной, если она, возможно, не тотальна (определена не везде). Частичная функция $f: x \rightarrow y$ называется \textbf{вычислимой}, если существует алгоритм, такой что $D(f)$ совпадает с областью применимости и $\forall \:x\in D(f)$ есть результат применения алгортима к x.\\
Множество $A\subseteq X$ называется \textbf{разрешимым}, если вычислима его \textbf{характеристическая} функция
\begin{center}
    $\chi_{A}(n)=\left\{\begin{array}{ll}
    {1,} & {n \in A} \\
    {0,} & {n \notin A}
    \end{array}\right.$
\end{center}
Множество $A\subseteq X$ называется \textbf{полуразрешимым}, если вычислима его полухарактеристическая функция 
    \begin{center}
        $\chi_{A}(n)=\left\{\begin{array}{ll}
        {1,} & {n \in A} \\
        \text{не определено иначе}
        \end{array}\right.$
    \end{center}
Множество $A\subseteq X$ называется \textbf{перечислимым}, если $A=\varnothing$ или $A$ -- множество значений всюду определенной вычислимой функции $f:\mathbb{N}\rightarrow X$\\.
\textcolor{mygray}{пример решения задач на эквивалентность определений \url{http://lpcs.math.msu.su/~zolin/vmlta/pdf/2019-2020_Logic_Zolin_Seminar_7_screen.pdf}}
