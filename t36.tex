\section{Пример вычислимой функции, не имеющей всюду определённого вычислимого продолжения. Пример неразрешимого
перечислимого множества. Алгоритмическая неразрешимость проблемы остановки машин Тьюринга.}
\begin{theorem}
	Существует вычислимая функция d с натуральными аргументами и значениями, от которой никакая вычислимая функция
	f не может всюду отличаться: для любой вычислимой функции f найдётся такое число n, что f(n) = d(n)(либо оба
	значения f(n) и d(n) не определены, либо оба определены и равны).
\end{theorem}
\begin{proof}
	Такова диагональная функция $d(n) = U(n,n)$ (здесь U — вычислимая функция двух аргументов, универсальная для
	класса вычислимых функций одного аргумента). Любая вычислимая функция f есть $U_n$ при некотором n и потому
	$f(n) = U_n(n) = U(n,n) = d(n)$.
\end{proof}
\begin{theorem}[Теорема-пример 1]
	Существует вычислимая функция, не имеющая всюду определённого вычислимого продолжения.
\end{theorem}
\begin{proof}
	Такова, например, функция $D(n) = d(n) + 1$, где d — функция из предыдущей теоремы. В самом деле, любое её
	всюду определённое продолжение всюду отличается от d (в тех местах, где функция d определена, функция D на
	единицу больше d и потому любое продолжение функции D отличается от d; там, где d не определена, любая всюду
	определённая функция отличается от d).
\end{proof}
\begin{theorem}[Теорема-пример 2]
	Существует перечислимое неразрешимое множество.
\end{theorem}
\begin{proof}
	Рассмотрим вычислимую функцию f(x), не имеющую всюду определённого вычислимого продолжения. Её область
	определения F будет искомым множеством. В самом деле, F перечислимо (по одному из определений перечислимости).
	Если бы F было разрешимо, то функция
	\begin{equation*}
		g(x) = 
		\begin{cases}
			f(x) &\text{если $x \in F$}\\
			0 &\text{если $x \notin F$}
		\end{cases}
	\end{equation*} 
	была бы вычислимым всюду определённым продолжением функции f (при вычислении g(x) мы сначала проверяем, лежит
	ли x в F, если лежит, то вычисляем f(x)).
\end{proof}

Полезно проследить, какое именно множество в итоге оказалось перечислимым и неразрешимым. Легко понять, что это
множество тех n, при которых U(n,n) определено. Если вспомнить конструкцию функции U, то это множество тех n, при
которых n-я программа останавливается на n. Поэтому говорят, что «проблема самоприменимости» (применимости
программы к своему номеру) неразрешима. Заметим, что отсюда следует, что и область определения всей универсальной
функции U является перечислимым неразрешимым множеством пар. (Если бы проблема выяснения применимости программы к
произвольному аргументу была бы разрешима, то и её частный случай — применимость программы к себе — был бы
разрешим.) Эту более общую задачу (узнать, остановится ли данная программа на данном входе) называют иногда
«проблемой остановки».
