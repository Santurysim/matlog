\section{Теории первого порядка с равенством. Нормальные модели. Теорема о существовании нормальной модели у
выполнимой теории с равенством.Примеры теорий с равенством: теория групп, формальная арифметика.}
Пусть $\Sigma$ -- сигнатура, содержащая выделенный предикатный символ <<$ = $>>.
\begin{definition}
	Нормальной моделью называем модель $(M,\Sigma)$, в которой «=» интерпретируется как равенство $\{\,\langle x,x
	\rangle\mid x\in M \,\}$.
\end{definition}
\begin{definition}
	Аксиомы равенства для $\Sigma$ суть универсальные замыкания следующих формул:
	\begin{enumerate}
		\item аксиомы отношения эквивалентности для <<=>>
		\item $a_{1}=b_{1} \wedge a_{2}=b_{2} \wedge \ldots \wedge a_{n}=b_{n} \rightarrow\left(P\left(a_{1}, \ldots,
	a_{n}\right) \leftrightarrow P\left(b_{1}, \ldots, b_{n}\right)\right)$
		\item $a_{1}=b_{1} \wedge a_{2}=b_{2} \wedge \ldots \wedge a_{n}=b_{n} \rightarrow\left(f\left(a_{1}, \ldots,
	a_{n}\right)=f\left(b_{1}, \ldots, b_{n}\right)\right)$ для всех $f\in Func_\Sigma,P\in Pred_\Sigma$
	\end{enumerate}
\end{definition}
\begin{proposition}
	Если $(M,\Sigma)$ -- нормальная модель, то в $M$ истинны все аксиомы равенства.
\end{proposition}
\begin{definition}
	\textbf{Теорией с равенством} называем теорию в языке с равенством, содержащую все аксиомы равенства.
\end{definition}
\begin{theorem}
	Пусть $T$ -- теория с равенством. Если $T$ выполнима, то $T$ имеет нормальную модель.
\end{theorem}
\begin{proof}
	Пусть $M \vDash T$. Предикат $=_M$ есть отношение эквивалентности на $M$.Положим $M'\rightleftharpoons M/_{=M}$
	-- множество классов эквивалентности и пусть $\varphi\colon M\to M'$сопоставляет любому $x\in M$ его класс
	эквивалентности $\varphi(x)\in M'$.Все функции и предикаты сигнатуры $\Sigma$ естественным образом переносятся с
	$M$ на $M'$. Полагаем
	\begin{center}
		$\begin{aligned}
			P_{M^{\prime}}\left(\varphi\left(x_{1}\right), \ldots, \varphi\left(x_{n}\right)\right) &
			\Longleftrightarrow P_{M}\left(x_{1}, \ldots, x_{n}\right) \\
			f_{M^{\prime}}\left(\varphi\left(x_{1}\right), \ldots, \varphi\left(x_{n}\right)\right) &
			\rightleftharpoons \varphi\left(f_{M}\left(x_{1}, \ldots, x_{n}\right)\right) \\
			c_{M^{\prime}} & \rightleftharpoons \varphi\left(c_{M}\right)
		\end{aligned}$
	\end{center}
	Заметим, что в силу истинности аксиом равенства в $M$ все функции и предикаты корректно определены на $M'$, и
	$M'$ -- нормальная модель.\\
	Индукцией по построению формулы $A$ проверяем
	\begin{center}
		$M \vDash A\left[x_{1}, \ldots, x_{n}\right] \Longleftrightarrow M^{\prime} \vDash
		A\left[\varphi\left(x_{1}\right), \ldots, \varphi\left(x_{n}\right)\right]$
	\end{center}
	Отсюда следует $M'\vDash T$
\end{proof}
\begin{example}
	Язык арифметики содержит один константный символ 0, один одноместный функциональный символ S и двухместные
	функциональные символы + и $\cdot$. Единственным (двухместным) предикатным символом языка является равенство. В
	стандартной интерпретации переменные принимают значения в множестве натуральных чисел N, символ 0
	интерпретируется как ноль, S как операция прибавления единицы, а + и $\cdot$ как сложение и умножение
	соответственно. Формула в языке арифметики со свободными переменными задаёт некоторый предикат. Если предикат
	можно выразить некоторой формулой в языке арифметики, то он называется арифметичным. 
\end{example}
\begin{example}
	$(M;=;\cdot;1)$ есть группа, если $M$ есть модель следующей теории (при условии что <<=>> в $M$ понимается как
	равенство):\\
	1) $\forall x,y,z: x\cdot(y\cdot z)=(x\cdot y)\cdot z$\\
	2) $\forall x \ (1 \cdot x=x \wedge x \cdot 1=x)$\\
	3) $\forall x \ \exists y \ (x \cdot y=1 \wedge y \cdot x=1)$\\
\end{example}
