\section{Вполне упорядоченные множества. Начальные отрезки. Теорема о сравнении вполне упорядоченных множеств.
Сумма, произведение вполне упорядоченных множеств.}

\begin{definition}
	\textbf{Строгим частичным порядком} на множестве $X$ называем бинарное отношение $<$ на $X$, удовлетворяющее
	свойствам:

	$\circ$ $x < y \land y<z \Rightarrow x<z$ (транзитивность);

	$\circ$ $x \not< x$ (иррефлексивность).

	Пару $(X, <)$ называем \textbf{частично упорядоченным множеством}. Элементы $x, y \in X$ называются
	\textbf{сравнимыми}, если $x < y$, или $x = y$, или $y < x$. Частично упорядоченное множество $(X, <)$
	называется \textbf{линейно упорядоченным}, или просто \textbf{упорядоченным}, если любые $x, y \in X$ сравнимы.

	Пусть $(X, <)$ -- частично упорядоченное множество и $Y \subset X$. 

	$\circ$ Элемент $y \in Y$ \textbf{максимальный} в $Y$, если $\forall x \in Y \ y \not< x$.

	$\circ$ Элемент $y \in Y$ \textbf{наибольший} в $Y$, если $\forall x \in Y \ x \leqslant y$.

	$\circ$ Элемент $x \in X$ есть \textbf{верхняя грань} $Y$, если $\forall y \in Y \ y \leqslant x$.
\end{definition}

\begin{example}
	Частично упорядоченные множества:

	$\circ$ $(\mathbb{R}, <)$, $(\mathbb{Q}, <)$, $(\mathbb{N}, <)$,

	$\circ$ $\mathbb{N}$ с отношением $x$ \textit{есть собственный делитель} $y$;

	$\circ$ $(\mathcal{P}(X), \varsubsetneqq)$;

	$\circ$ $\mathbb{N}^{*}$ с отношением \textit{последовательность $x$ -- собственное начало последовательности $y$}.

	Первые три примера — линейно упорядоченные множества, а последние три — нет.
\end{example}

\begin{definition}
	Всякое подмножество $Y \subset X$ частично упорядоченного множества $(X, <)$ можно также рассматривать как
	частично упорядоченное множество по отношению $<^\prime$ на $Y$:

	$$
	x<^{\prime} y \Longleftrightarrow(x, y \in Y \text { u } x<y)
	$$

	(Формально, можно было бы определить $<^\prime$ как $< \cap Y^2$.) В этом случае говорят, что порядок
	$<^\prime$ является \textbf{ограничением порядка} $<$ на множество $Y$ или \textbf{индуцирован} на $Y$ c $X$.

	Множество $Y \subset X$ называется \textbf{цепью}, если любые два элемента $Y$ сравнимы. Другими словами, $Y$
	-- цепь, если $Y$ линейно упорядочено в смысле индуцированного отношения порядка. Множество $Y \subset X$
	называется \textbf{антицепью}, если любые два элемента $Y$ \textbf{несравнимы}.
\end{definition}

\begin{definition}
	Пусть $(X, <_X)$ и $(Y, <_Y)$ -- линейно упорядоченные множества. Отображение $f : X \rightarrow Y$ называется
	\textbf{сохраняющим порядок} (или \textbf{возрастающим}), если

	$$
	\forall x_{1}, x_{2} \in X\left(x_{1}<_{X} x_{2} \Rightarrow f\left(x_{1}\right)<_{Y} f\left(x_{2}\right)\right)
	$$

	\textbf{Изоморфизмом} упорядоченных множеств $X$ и $Y$ называется биекция $f : X \rightarrow Y$, для которой
	$f$ и обратное отображение $f^{-1}$ сохраняют порядок. $X \cong Y$ означает, что упорядоченные множества $X$ и
	$Y$ изоморфны, то есть между ними существует изоморфизм.
\end{definition}

\begin{definition}
	\textbf{Суммой} $X+Y$ назовём упорядоченное множество $(Z, <_Z)$, где $Z =
	X \sqcup Y$ и для любых $z_1, z_2 \in Z$ соотношение $z_1 <_Z z_2$ имеет место в одном
	из трех случаев:

	$\circ$ $z_1, z_2 \in X$ и $z_1 <_X z_2$,

	$\circ$ $z_1, z_2 \in Y$ и $z_1 <_Y z_2$,

	$\circ$ $z_1 \in X$ и $z_1 \in Y$.

	\textbf{Произведением} $X \cdot Y$ назовём множество $(Z, <_Z)$, где $Z = Y \times X$ и для любых $z_1 = (y_1,
	x_1) \in Z$ и $z_2 = (y_2, x_2) \in Z$ соотношение $z_1 <_Z z_2$ имеет место, если и только если $y_1 <_Y y_2$
	или же $y_1 = y_2$ и $x_1 < x_2$. (Сравнение сначала элементов множества $Y$, а потом уже $X$, выражает ту
	идею, что $X \cdot Y$ состоит из копий множества $X$, упорядоченных между собой как $Y$, а не наоборот.)
\end{definition}

\begin{definition}
	Упорядоченное множество $(X, <)$ называем \textbf{вполне упорядоченным}, если любое непустое подмножество $Y
	\subset X$ имеет наименьший элемент $y \in Y$. Наименьший элемент $Y$ -- единственный и обозначается $min(Y)$.

	\textbf{Начальным отрезком} множества $(X, <)$ называем такое подмножество $Y \subset X$, для которого 

	$$
	\forall x, y(x \in Y, y<x \Rightarrow y \in Y).
	$$

	В частности, начальными отрезками $X$ считаем само $X$ и пустое множество.
\end{definition}

\begin{lemma}
	Любой собственный начальный отрезок $(X, <)$ имеет вид $\bar{a}=\{x \in X | \ x<a\}$ для некоторого $a \in X$.
\end{lemma}

\begin{proof}
	Пусть $Y$ -- собственный начальный отрезок $X$, и пусть $a = min ( X \backslash Y)$. Заметим, что $a \notin Y$
	и $\forall x<a \ x \in Y$. Второе влечёт $\bar{a} \subset Y$. С другой стороны, если $\exists y \in Y \ a
	\leqslant y$, то мы имеем $a \in Y$, поскольку $Y$ -- начальный отрезок. Этого не может быть, значит $Y \subset
	\bar{a}$.
\end{proof}

\begin{lemma}
	Пусть $(X, <)$ вполне упорядочено и $f : X \rightarrow X$ сохраняет порядок. Тогда $\forall x \in X \ f(x)
	\geqslant x$.
\end{lemma}

\begin{proof}
	В противном случае рассмотрим $a = min Y$, где $Y=\{x \in X | f(x) < x\}$. Поскольку $a \in Y$ мы имеем $f(a) <
	a$. Отсюда следует $f(f(a)) < f(a)$ по монотонности $f$. Но тогда $f(x) < x$ для некоторого $x < a$ (возьмём $x
	= f(a)$), что противоречит минимальности $a$.
\end{proof}

\begin{theorem}
	\begin{enumerate}
		\item Вполне упорядоченное множество не изоморфно никакому своему собственному начальному отрезку. 

		\item Для любых двух вполне упорядоченных множеств одно изоморфно начальному отрезку другого. 
	\end{enumerate}
\end{theorem}

\begin{proof}
	\begin{enumerate}
		\item Пусть $Y \subset X$ -- собственный начальный отрезок $X$, и $f : X \rightarrow Y$ -- изоморфизм.
			Тогда по лемме 5.2 имеем $f(x) \geqslant x$ для всех $x \in X$. Но если $a \in X \backslash Y$, то
			$f(a) \in Y$ и тем самым $f(a) < a$, поскольку $Y$ -- начальный отрезок $X$. Противоречие. 

		\item Рассмотрим бинарное отношение $R \subset X \times Y$ такое, что

			$$
			x R y \Longleftrightarrow \bar{x} \cong \bar{y}.
			$$

			Сначала докажем, что отношения $R$, $R^{-1}$ функциональны и сохраняют порядок.
			Действительно, если $xRy_1$ и $xRy_2$, то $\bar{x} \cong \bar{y_1}$ и $\bar{x} \cong \bar{y_2}$, значит
			$\bar{y_1} \cong \bar{y_2}$. Поскольку $Y$ линейно упорядочено, мы имеем $y_1 < y_2$ или $y_2 < y_1$
			или $y_1 = y_2$. Если $y_1 < y_2$, то $\bar{y_1}$ -- собственный начальный отрезок $\bar{y_2}$, что
			противоречит (1). Аналогично, не может быть $y_2 < y_1$, поэтому $y_1 = y_2$. 

			Докажем, что $R$ сохраняет порядок. Допустим, что $x_1 < x_2$, $\bar{x_1} \cong \bar{y_1}$ и $\bar{x_2}
			\cong \bar{y_2}$. Изоморфизм $f : \bar{x_2} \rightarrow \bar{y_2}$ переводит $\bar{x_1}$ в некоторый
			собственный начальный отрезок $f(\bar{x_1}) \subset \bar{y_2}$. Если при этом $y_2 \leqslant y_1$, то
			получаем, что $\bar{y_1}$ изоморфно собственному начальному отрезку $f(\bar{x_1}) \cong \bar{x_1}$, что
			невозможно. Значит, $y_1 < y_2$.

			Аналогично устанавливаем, что $x_1Ry$ и $x_2Ry$ влечёт $x_1 = x_2$, и что $R^{-1}$ сохраняет порядок.

			Осталось доказать, что хотя бы одна из функций $R$ и $R^{-1}$ определена на всём множестве $X$ или на
			всём множестве $Y$, соответственно. Предположим противное и рассмотрим наименьший $a \in X$ такой, что
			$\nexists y \in Y \ aRy$ и наименьший $b \in Y$ такой, что $\nexists x \in X \ xRb$. Тогда $R$ есть
			изоморфизм начального отрезка $\bar{a} \subset X$ на начальный отрезок $\bar{b} \subset Y$, поскольку
			на $\bar{a}$ функция $R$ всюду определена, сохраняет порядок, и то же верно для обратной функции
			$R^{-1}$. Но тогда по определению $R$ мы имеем $aRb$. Противоречие с минимальностью $a$ и $b$. 
	\end{enumerate}
\end{proof}
