\section{Перечислимые множества.Теорема об эквивалентных определениях перечислимого множества.}
\begin{definition}
	Множество $A \subset \Sigma^*$ \textbf{перечислимо}, если $A=rng(f)$, где $f$ вычислима.
\end{definition}
Интуитивно это значит следующее: $f\colon\mathbb{N}\to\Sigma^*$ переводит какие-то натуральные числа в $f(n)$, при
этом мы получаем перечисление $x_0, x_2, x_2,\ldots$ значений функции, которое и будет являться нашим
множеством $A$.

\begin{proposition}
	Из разрешимости следует перечислимость.
\end{proposition}
\begin{proof}
	Действительно, если $A$ разрешимо, то можно вычислить характеристическую функцию $\chi_{A}(x)$, тогда положим
	$\chi^*_{A}(x)$ := $1$, если вычислили  $\chi_{A}(x)$ и получили $1$. В другом случае зациклимся. Построенная
	функция является вычислимой, определена на всех элементах множества $A$ и не определена иначе. Значит, $A$
	перечислимо.
\end{proof}
\begin{theorem}[Эквивалентные определения перечислимого множества]
	Следующие утверждения равносильны:
	\begin{enumerate}
		\item $A$ перечислимо;
		\item $A = \varnothing $ или $A = rng(f)$, где f -- тотально вычислимая функция;
		\item $A = \dom(f)$, где $f$ - вычислимая;
		\item Вычислима функция
			\begin{equation*}
				\chi^*_{A}(x) = 
				\begin{cases}
					\text{1, если $x \in A$}\\
					\text{не определена, иначе}
				\end{cases}
			\end{equation*}
		\item $A = \{\,x\mid\exists y\,\langle x,y\rangle\in B\,\}$, где $B$ - разрешимо.
	\end{enumerate}
\end{theorem}
\begin{proof}
	2) $\Rightarrow$ 1), 4) $\Rightarrow$ 3) очевидны.

	3) $\Rightarrow$ 4): $A=\dom(f)$. Рассмотрим машину Тьюринга $M$, соответствующую $f$.Сделаем новую машину
	Тьюринга $M'$, которая делает следующее: $M'$ работает как $M$, если же $M$ останавлвается, то мы стираем то,
	что было на ленте и пишем там $f$. $M'$ вычислит $\chi^*_{A}(x)$.

	3) $\Rightarrow$ 5): Рассмотрим машину $M_f$, которая вычисляет $f, \dom(f)=A$ и рассмотрим множество $B =
	\{\,\langle x,y\rangle\in A\times\mathbb{N}\mid M_{f}(x)\mbox{ останавливается через}\,\leqslant y\mbox{ шагов
	}\,\}$. $B$ - разрешимо. Действительно, характеристическую функцию $B$ можно вычислить следующим образом:
	выполним $y+1$ шаг $f(x)$. Если при этом машина остановится, то выдаём $1$, иначе $0$. Требуемое множество
	построено.

	5) $\Rightarrow$ 2): Пусть $A = \{\,x\mid \exists y\,\langle x,y\rangle \in B\,\}$, $B$ - разрешимо.
	Допустим, что $A\neq\varnothing$. Выберем $a_0\in A$ и построим функцию
	$\varphi: \mathbb{N} \to \mathbb{N}$ такую, что 
	\begin{equation*}
		\varphi(n) = 
		\begin{cases}
			a_{0}\text{, если }n=\langle x,y\rangle\notin B&\\
			l(n)\text{, если }n=\langle x,y\rangle \in B.&
		\end{cases}
	\end{equation*}

	Она вычислима. $x \in A \iff \exists n (l(n)=x \land n \in B) \Rightarrow$ вычислима.
	
	1) $\Rightarrow$ 4) $A = rng(f)$, дана машина $M_f$. Для данного $n=0,1,2,\ldots$ выполним $l(n)$ шагов
	вычисления функции $f(r(n))$ (теперь проекция на вторую координату), то есть $M_{f}(r(n))$, где $n = \langle l(n),
	r(n)\rangle$. Если какое-то вычисление завершено и $f(r(n))=x$, то возьмём 1 на нём в качестве
	ответа. Схема обхода на картинке.
\end{proof}
