\section{Натуральные числа. Индуктивные множества, аксиома бесконечности. Формальное определение множества
натуральных чисел. Принципы индукции, порядковой индукции, наименьшего числа, их вывод из определения множества
натуральных чисел.}

По определению, число 0 отождествляется с пустым множеством $\varnothing$, число 1 с $\{ \varnothing \}$, число 2 с
$\{ \varnothing, \{ \varnothing \}\}$ и так далее. Требуется доказать, что это множество.

\textbf{Аксиома бесконечности}: Существует $S$ такое, что $\varnothing \in S$ и для любого $x \in S$ множество $x
\cup\{x\} \in S$.

\begin{definition}
	Множество $S$ называется \textbf{индуктивным}, если $\varnothing \in S$ и $\forall x \in S$ $x \cup\{x\}
	\in S$.
\end{definition}

Аксиома бесконечности утверждает, что существует хотя бы одно индуктивное множество.

\begin{lemma}
	Пересечение непустого множества индуктивных множеств индуктивно.
\end{lemma}

\begin{proof}{$\varnothing$ содержится в каждом из множеств, значит содержится и в пересечении. Если $x$ содержится
	в пересечении, то и $x \cup\{x\}$ содержится.}
\end{proof}

Пусть $S$ — некоторое индуктивное множество, существующее по аксиоме бесконечности. Рассмотрим множество
$\mathcal{I}$ всех индуктивных подмножеств $S$.
Обозначим через $\omega$ его пересечение: $\omega:=\cap \mathcal{I}$. Из леммы  получаем, что $\omega$ индуктивно.
Кроме того, $\omega$ содержится в любом индуктивном множестве: если $S^{\prime}$ индуктивно, то таковым является $S
\cap S^{\prime}$, откуда $S \cap S^{\prime} \in \mathcal{I}$ и следовательно $\omega \subset S \cap S^{\prime}$.
Таким образом, мы показали, что $\omega$ есть наименьшее (по включению) индуктивное множество. Такое множество
единственно по аксиоме объёмности.

\begin{definition}
	Наименьшее по включению индуктивное множество называется \textbf{множеством натуральных чисел}
	(обозначение $\mathbb{N}$ или $\omega$).
\end{definition}

Отметим, что $0 := \varnothing \in \mathbb{N}$ и из того, что $n \in \mathbb{N}$ следует $n \cup\{n\} \in
\mathbb{N}$. Натуральное число $n \cup\{n\}$ назовём \textbf{следующим} за $n$ и будем обозначать $S(n) = n + 1$.
Функция $n \mapsto n+1$ действует из $\mathbb{N}$ в $\mathbb{N}$. Для $n \in \mathbb{N}$ и любого $m$ определим $m
< n$, если и только если $m \in n$. (Ниже мы докажем, что из $m \in n \in \mathbb{N}$ следует $m \in \mathbb{N}$.)
Из этого определения мы видим, что для всех $m, n \in \mathbb{N}$ 

$$m<n+1 \leftrightarrow(m<n \vee m=n)$$

\subsection{Принцип математической индукции}

\begin{theorem}
	(принцип индукции) Допустим, что некоторое множество $A$ удовлетворяет условиям: $0 \in A$ и $\forall n \in
	\mathbb{N} \ (n \in A \rightarrow n+1 \in A)$. Тогда $\forall n \in \mathbb{N} \ n \in A$.
\end{theorem}

\begin{proof}{По условию теоремы множество $A \cap \mathbb{N}$ индуктивно. Поскольку $\mathbb{N}$ -- наименьшее
	индуктивное, мы имеем $\mathbb{N} \subset A$.}
\end{proof}

\begin{theorem}
	(порядковая индукция) Пусть множество $A$ удовлетворяет условию $\forall n \in \mathbb{N}(\forall m<n \ m \in A
	\rightarrow n \in A)$. Тогда $\forall n \in \mathbb{N} \ n \in A$.
\end{theorem}

\begin{proof}{Предположим, что $A$ удовлетворяет условию теоремы. Рассмотрим множество

	$$A^{\prime}:=\{x \in \mathbb{N}: \forall y<x \ y \in A\}$$

	Тогда $0 \in A$, поскольку $\neg \exists y$ $y<0$, то есть условие $\forall y<0$ $y \in A$ выполняется
	тривиально. Допустим $n \in \mathbb{N}$ и $n \in A^{\prime}$, тогда $\forall m<n$ $m \in A$. По условию теоремы
	отсюда следует $n \in A$. Мы утверждаем, что $\forall m < n + 1$ $m \in A$, то есть $n+1 \in A^\prime$. В самом
	деле, если $m < n + 1$, то $ m < n$ или $ m = n$. В каждом из этих случаев мы уже знаем, что $m \in A$. По
	предыдущей теореме мы заключаем $\forall n \in \mathbb{N}$ $n \in A^{\prime}$. Осталось вывести отсюда $\forall
	n \in \mathbb{N}$ $n \in A$. Рассмотрим любое $n \in \mathbb{N}$, тогда $n + 1 \in \mathbb{N}$ и по доказанному
	$n + 1 \in A^{\prime}$. Поскольку $n < n + 1$, по определению $A^\prime$ отсюда следует $n \in A$.}
\end{proof}

\begin{theorem}
	(принцип минимального элемента) Всякое непустое подмножество $A \subset \mathbb{N}$ имеет минимальный элемент.
\end{theorem}

\begin{proof} Допустим противное, то есть $A \neq \varnothing$ и

	$$\neg \exists n \in A \ \forall m<n \  m \notin A$$

	Рассмотрим $B:=\mathbb{N} \backslash A$ Докажем

	$$\forall n \in \mathbb{N}(\forall m<n \  m \in B \rightarrow n \in B)$$

	Допустим $n \in \mathbb{N}$ и $\forall m<n \  m \in B$. Тогда $\forall m<n \ m \notin A$. Значит $n \notin A$ в
	силу первого утверждения, то есть $n \in B$. По теореме 2.2 заключаем, что $\forall n \in \mathbb{N} \ n \in
	B$. Отсюда следует $A = \varnothing$, противоречие.
\end{proof}

\begin{corollary}
	$\forall n \in \mathbb{N} \ n \not<n$.
\end{corollary}

\begin{proof} 
	Применим порядковую индукцию. Допустим $\forall m < n \ m \not< m$. Если $ n < n$, то в качестве $m$ можно
	взять само $n$, тогда получим
	$n \not< n$, противоречие. Следовательно, $n \not< n$. 
\end{proof}

\begin{corollary} 
	$\forall n \in \mathbb{N} \ \forall x<n \ x \in \mathbb{N}$ 
\end{corollary}

\begin{proof} 
	Применим индукцию. $\forall m < 0 \ m \in \mathbb{N}$ тривиально.
	Допустим $n \in \mathbb{N}$ и $\forall x<n \ x \in \mathbb{N}$. Тогда очевидно $\forall x<n+1 \ x \in \mathbb{N}$. 
\end{proof}

\begin{corollary} 
	$\forall k, m, n \in \mathbb{N} \ (k<m<n \rightarrow k<n)$.
\end{corollary}

\begin{proof} 
	Индукция по $n$. Случай $n = 0$ тривиален. Допустим, что утверждение верно для $n$ и имеет место $k<m<n + 1$.
	Тогда $m < n$ или $m = n$. В первом случае по предположению индукции $k < n$. Во втором случае мы уже знаем,
	что $k < n = m$. Из $k < n$ следует и
	$k < n + 1$. 
\end{proof}

\begin{corollary} 
	$\forall m, n \in \mathbb{N} \ \neg(n<m \wedge m<n)$.
\end{corollary}

\begin{proof} 
	Если $n<m<n$, то $n<n$, что противоречит следствию 2.3.1. 
\end{proof}

\subsection{Аксиомы Дедекинда-Пеано}

\begin{enumerate}

	\item $ \forall n \in \mathbb{N} \  S(n) \not= 0$ 

	\item $ \forall n,m \in \mathbb{N} \ (S(n)=S(m) \leftrightarrow n = m)$ 

	\item $ \forall n \in \mathbb{N} \ (n = 0 \lor \exists m (S(m) = n))$ 

	\item Принцип индукции 

\end{enumerate}

Данные аксиомы выполняются для определения, данного выше.

\begin{proof}

	\begin{enumerate}

		\item Множество $S(n)=n+1$ непусто 

		\item Легко доказывается индукцией по $n$ 

		\item Достаточно доказать импликацию слева направо. Допустим $n \cup\{n\}=m \cup\{m\}$. Так как $m \in n
			\cup\{n\}$ имеем $m \in n$ или
			$m = n$. Во втором случае утверждение доказано. Допустим $m \in n$. Так
			как $n \in m \cup\{m\}$ имеем $n \in m$ или $n = m$. Первый случай невозможен
			по следствию 2.3.4. Остаётся второй. 

		\item По определению. 
	\end{enumerate}
\end{proof}
