\section{Разрешимые множества. Свойства объединения, пересечения и дополнения разрешимых множеств.}
\begin{definition}
	Множество $A \subset \Sigma^*$ \textbf{разрешимо}, если можно вычислить характеристическую функцию:
	\begin{equation*}
		\chi_{A}(x) = 
		\begin{cases}
			1 &\text{если $x \in A$}\\
			0 &\text{если $x \notin A$}
		\end{cases}
	\end{equation*}
\end{definition}
\begin{example}
	$\varnothing$ (тождественно нулевая функция), $\mathbb{N}$ (тождественная функция), $2\mathbb{N}$
\end{example}

\begin{proposition}
	Семейство разрешимых множеств замкнуто относительно объединения, пересечения и дополнения, то есть если $A, B$
	разрешимы, то $A \cup B$, $A \cap B$, $\mathbb{N} \setminus A$ разрешимы.
\end{proposition}
\begin{proof}
	Если $A$ разрешимо, то можно вычислить характеристическую функцию $\chi_{A}(x)$, тогда
	функция для дополнения $\mathbb{N} \setminus A$ есть инвертированная $\chi_{A}(x)$.
\end{proof}
