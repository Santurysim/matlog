\section{Дизъюнктивные и конъюнктивные нормальные формы. Приведение формул логики высказываний к совершенной
дизъюнктивной (конъюнктивной) нормальной форме. Единственность совершенной дизъюнктивной
нормальной формы.}

\definition{ \textbf{Литералами} называются переменные и их отрицания.}
\begin{example}
	Формула А является литералом, а $ \neg\neg A$ - нет.
\end{example}

\definition{  \textbf{Элементарной конъюнкцией} называем формулу вида
$\bigwedge^{n}_{i=1}L_{i}$, где $ L_{i} $ - литералы.

\textbf{Дизъюнктивной нормальной формой (ДНФ)} называем формулу вида $\vee^{m}_{j=1}C_{j}$ , где $ C_{j} $ -
элементарные коньюнкции.

\textbf{Элементарной дизъюнкцией} называем формулу вида
$\vee^{n}_{i=1}L_{i}$, где $ L_{i} $ - литералы.  

\textbf{Конъюнктивной нормальной формой (КНФ)} называем формулу вида $\wedge^{m}_{j=1}C_{j}$ , где $ C_{j} $ -
элементарные дизъюнкции.}


\definition{ Формула A называется \textbf{совершенной ДНФ}, если A— ДНФ и 
\begin{itemize}
	\item Каждая элементарная конъюнкция имеет вид 
		$ A_{\vec{x}} \rightleftarrows \bigwedge_{i=1}^{n} P_{i}^{x_{i}} $
		для некоторого $\vec{x} = (x_1,\ldots,x_n) \in \mathbb{B}^n$ 
	\item $A=\bigvee_{j=1}^m A_{\vec{x}_j}$, где $\vec{x}_1,\ldots,\vec{x}_m \in \mathbb{B}^{n}$ попарно различны и
		взяты в лексикографическом порядке.
\end{itemize}
Аналогично определяется \textbf{совершенная КНФ}, с заменой дизъюнкций
на конъюнкции и наоборот.}
\begin{remark} 
	Удобно расширить множество формул константами
	$\perp$ (ложь) и $\top$ (истина). Тем самым, формулами считаются и все выражения, построенные с помощью булевых
	связок из переменных и этих констант. Считаем $\perp$ совершенной ДНФ, а $\top$ -- совершенной КНФ.\\
	Совершенные ДНФ и КНФ перестают быть совершенными, если рассматривать их как формулы от более широкого набора
	переменных. Поэтому имеет смысл говорить о совершенных ДНФ и КНФ лишь относительно некоторого фиксированного
	набора переменных.
\end{remark}

\begin{theorem}
	Каждая пропозициональная формула A равносильна некоторой совершенной дизъюнктивной нормальной форме. Причем СДНФ
	любой формулы A единственна.
\end{theorem}

\begin{proof}
	\textbf{Существование.} Если формула A тождественно ложна, в качестве её ДНФ можно взять $\perp$. В противном
	случае достаточно заметить, что формула, построенная в доказательстве теоремы о функциональной полноте для
	функции $\varphi_{A}$ есть совершенная ДНФ.

	\textbf{Единственность.}
	Заметим, что совершенные ДНФ эквивалентных формул (графически) совпадают. И правда, для совершенной ДНФ каждая
	элементарная конъюнкция определяет некоторую выполняющую оценку, а сама ДНФ -- множество всех таких оценок.
	Отсюда и следует единственность.
\end{proof}
{Аналогичная теорема и для СКНФ.}
\begin{theorem}
	Каждая пропозициональная формула A равносильна некоторой совершенной коньюктивной нормальной форме. Причем СКНФ
	любой формулы A единственна.
\end{theorem}

\begin{proof}
	\textbf{Существование.} Если формула A тождественно истинна, в качестве её СКНФ можно взять $\top$. В противном
	случае, построим СДНФ для $\neg A$. Берем отрицание этого СДНФ. Нетрудно заметить, что это СКНФ для нашей
	функции.  

	\textbf{Единственность.}
	Заметим, что совершенные КНФ эквивалентных формул (графически) совпадают.И правда, для совершенной СКНФ каждая
	элементарная дизъюнкция определяет некоторую невыполняющую оценку, а сама КНФ -- множество всех таких оценок.
	Отсюда и следует единственность. (Или просто из единственность СДНФ для отрицания A).
\end{proof}
