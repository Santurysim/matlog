\section{Универсальные функции. Построение универсальной вычислимой функции для класса всех одноместных вычислимых функций $\mathbb{N} \rightarrow \mathbb{N}$}

\begin{definition}
Функция $U$ двух натуральных аргументов является универсальной для класса вычислимых функций одного аргумента, если для каждого n функция $U_n: x \mapsto U(n,x)$ («сечение» функции U при фиксированном n) является вычислимой и если все вычислимые функции одного аргумента встречаются среди $U_n$. 
\end{definition}

\begin{theorem}
Существует вычислимая функция двух аргументов, являющаяся универсальной функцией для класса вычислимых функций одного аргумента.
\end{theorem}

\begin{proof}
Запишем все программы, вычисляющие функции одного аргумента, в вычислимую последовательность $p_0, \ p_1$ и т.д. (например, в порядке возрастания их длины). Положим $U(i,x)$ равным результату работы i-ой программы на входе x. Тогда функция U и будет искомой вычислимой универсальной функцией. Сечение $U_i$ будет вычислимой функцией, вычисляемой программой $p_i$. Алгоритм, вычисляющий саму функцию U, есть по существу интерпретатор для используемого языка программирования (он применяет первый аргумент ко второму, если отождествить программу и её номер).
\end{proof}

\begin{theorem}
Не существует вычислимой всюду определённой функции двух аргументов, универсальной для класса всех вычислимых всюду определённых функций одного аргумента.
\end{theorem}

\begin{proof}
Воспользуемся «диагональной конструкцией». Пусть U -- произвольная вычислимая всюду определённая функция двух аргументов. Рассмотрим диагональную функцию $u(n) = U(n,n)$. Очевидно, на аргументе n функция u совпадает с функцией $U_n$, а функция $d(n) = u(n) + 1$ отличается от $U_n$. Таким образом, вычислимая всюду определённая функция d(n) отличается от всех сечений $U_n$, и потому функция U не является универсальной.
\end{proof}